%
% Copyright � 2017 Peeter Joot.  All Rights Reserved.
% Licenced as described in the file LICENSE under the root directory of this GIT repository.
%
\makeproblem{Proving convexity-preserving operations}{convex-optimization:problemSet1:7}{
\makesubproblem{}{convex-optimization:problemSet1:7a}
Prove that the set \( \calS \) resulting from taking the intersection of a set of convex sets \( \calS_\alpha \) is itself a convex set.
I.e., \( \calS = \cap_\alpha \calS_\alpha \) is a convex set when all the \( \calS_\alpha \) are convex sets.

\makesubproblem{}{convex-optimization:problemSet1:7b}
Consider any affine function \( f : \Rm{n} \rightarrow \Rm{m} \) and convex set \( \calS \subseteq \Rm{n}\).
Prove that the image of \( \calS \) under \( f \), i.e., \( f(\calS) = \setlr{f(\Bx)| \Bx \in \calS}\), is a convex set.

\makesubproblem{}{convex-optimization:problemSet1:7c}
Consider any affine function \( f : \Rm{n} \rightarrow \Rm{m} \) and convex set \( \calS \subseteq \Rm{m} \).
Prove that the inverse (or pre-) image of \( \calS \) under \(f\), i.e., \( f^{-1}(\calS) = \setlr{\Bx|f(\Bx) \in \calS}\), is a convex set.
} % makeproblem

\makeanswer{convex-optimization:problemSet1:7}{
\makeSubAnswer{}{convex-optimization:problemSet1:7a}

It is sufficient to
consider the non-empty intersection of two convex sets \( A,B \), such as the intersection of polygons crudely sketched in \cref{fig:ps1p7:ps1p7Fig1}.

\imageFigure{../figures/ece1505-convex-optimization/ps1p7Fig1}{Intersection of plane polygons.}{fig:ps1p7:ps1p7Fig1}{0.3}

Any two points \( \Bx, \By \) in the intersection, are also contained in \( A, B \) and by convexity of \( A,B\) we must have both

\begin{dmath}\label{eqn:ProblemSet1Problem7:20}
\begin{aligned}
\setlr{ \theta \Bx + (1-\theta) \By | \theta \in [0,1] } &\subseteq A \\
\setlr{ \theta \Bx + (1-\theta) \By | \theta \in [0,1] } &\subseteq B,
\end{aligned}
\end{dmath}

so for any \( \theta \in [0,1] \), and any points \( \Bx, \By \in A \cap B \), we have

\begin{dmath}\label{eqn:ProblemSet1Problem7:40}
\theta \Bx + (1-\theta) \By \in A \cap B,
\end{dmath}

which means that \( A \cap B \) is convex.

\makeSubAnswer{}{convex-optimization:problemSet1:7b}

An affine function (linear function, plus translation), has the general form

\begin{dmath}\label{eqn:ProblemSet1Problem7:60}
f(\Bx) = A \Bx + \Bb,
\end{dmath}

Given a point \( \Bz \in S \), and a \( \theta \in [0,1] \)

\begin{dmath}\label{eqn:ProblemSet1Problem7:80}
\Bz = \theta \Bx + (1-\theta) \By,
\end{dmath}

the image of that point is

\begin{dmath}\label{eqn:ProblemSet1Problem7:100}
f( \theta \Bx + (1-\theta)\By )
=
A ( \theta \Bx + (1-\theta)\By ) + \Bb
=
A ( \theta (\Bx - \By) + \By ) + \Bb
=
\theta ( A \Bx + \Bb - A \By - \Bb ) + A \By + \Bb
=
\theta ( A \Bx + \Bb ) + (1 -\theta) (A \By + \Bb)
=
\theta f(\Bx) + (1-\theta) f(\By),
\end{dmath}

which is an affine combination.  This means that the set of such points, the image of \( f \) is a convex set.

\makeSubAnswer{}{convex-optimization:problemSet1:7c}

Running the steps above backwards, given \( \setlr{ f(\Bx) } \in S \), and \( \theta \in [0,1] \) then

\begin{dmath}\label{eqn:ProblemSet1Problem7:120}
\theta f(\Bx) + (1-\theta) f(\By)
=
\theta ( A \Bx + \Bb ) + (1 -\theta) (A \By + \Bb)
=
\theta ( A \Bx + \Bb - A \By - \Bb ) + A \By + \Bb
=
A ( \theta (\Bx - \By) + \By ) + \Bb
=
A ( \theta \Bx + (1-\theta)\By ) + \Bb
=
f( \theta \Bx + (1-\theta)\By ).
\end{dmath}

This shows that the set of points \( \setlr{ \theta \Bx + (1-\theta)\By } \) is a convex set, proving the result.
}
