%
% Copyright � 2017 Peeter Joot.  All Rights Reserved.
% Licenced as described in the file LICENSE under the root directory of this GIT repository.
%
\makeproblem{Convex, affine, and conic hulls}{convex-optimization:problemSet1:4}{
\makesubproblem{}{convex-optimization:problemSet1:4a}
Consider the set

\begin{equation}\label{eqn:ProblemSet1Problem4:20}
\calS = \setlr{
\begin{bmatrix}
1 \\
1
\end{bmatrix},
\begin{bmatrix}
1 \\
2
\end{bmatrix}
}
\subseteq \Rm{2}.
\end{equation}

Sketch conv(\(\calS\)), affine(\(\calS\)) and conic(\(\calS\)), respectively the convex, affine, and conic hulls of the set \(\calS\). Each is the union of all combinations of the respective type (convex, affine or conic).

\makesubproblem{}{convex-optimization:problemSet1:4b}

Repeat \partref{convex-optimization:problemSet1:4a} for the set

\begin{equation}\label{eqn:ProblemSet1Problem4:40}
\calS = \setlr{
\begin{bmatrix}
1 \\
1
\end{bmatrix},
\begin{bmatrix}
1 \\
2
\end{bmatrix},
\begin{bmatrix}
0.5 \\
0.25
\end{bmatrix}
}
.
\end{equation}

\makesubproblem{}{convex-optimization:problemSet1:4c}
Consider a set \(\calS\). What are the respective inclusion relations between the convex hull, the affine hull, and the conic hull of \(\calS\). I.e., which of these three sets are always subsets of the other, regardless of the original \(\calS\)?
} % makeproblem

\makeanswer{convex-optimization:problemSet1:4}{
\makeSubAnswer{}{convex-optimization:problemSet1:4a}

TODO.
\makeSubAnswer{}{convex-optimization:problemSet1:4b}

TODO.
\makeSubAnswer{}{convex-optimization:problemSet1:4c}

TODO.
}
