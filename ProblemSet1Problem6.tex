%
% Copyright � 2017 Peeter Joot.  All Rights Reserved.
% Licenced as described in the file LICENSE under the root directory of this GIT repository.
%
\makeproblem{Ellipses, eigenvalues, eigenvectors, and volume}{convex-optimization:problemSet1:6}{
Make neat and clearly-labelled sketches of the ellipsoid \( \calE = \setlr{\Bx | (\Bx - \Bx_c)^\T P^{-1} (\Bx - \Bx_c) = 1} \) for the following sets of parameters:
\makesubproblem{}{convex-optimization:problemSet1:6a}

Center \( \Bx_c =
\begin{bmatrix}
0 \\ 0
\end{bmatrix} \)
and \( P =
\begin{bmatrix}
1.5 & -0.5 \\
-0.5 & 1.5
\end{bmatrix} \).

\makesubproblem{}{convex-optimization:problemSet1:6b}
Center \( \Bx_c =
\begin{bmatrix}
1 \\ -2
\end{bmatrix} \)
and \( P =
\begin{bmatrix}
3 & 1 \\
1 & 3
\end{bmatrix} \).

\makesubproblem{}{convex-optimization:problemSet1:6c}
Center \( \Bx_c =
\begin{bmatrix}
-2 \\ 1
\end{bmatrix} \)
and \( P =
\begin{bmatrix}
9 & -2 \\
-2 & 6
\end{bmatrix} \).

For each part (a)-(c) also compute each pair of eigenvalues and corresponding eigenvectors.

\makesubproblem{}{convex-optimization:problemSet1:6d}
Recall that the most geometrically meaningful property of the determinant of a square real matrix \(A\) is that its magnitude \( \Abs{\det A} \) is equal to the volume of the parallelepiped \( \calP \) formed by applying \( A \) to the unit cube \( \calC = \setlr{x|0 \le x \le 1} \).
(Recall that since \( x \in \Rm{n} \) we interpret the inequalities coordinate-wise, i.e., \( 0 \le x_i \le 1 \) for all \( i = 1, \cdots, n \).)
In other words, if \( \calP = \setlr{ A x| x \in \calC} \) then \( \Abs{\det(A)}\) is equal to the volume of \( \calP\).
Furthermore, recall that the determinant of a matrix is zero if any of its eigenvalues are zero.
Explain how to interpret this latter fact in terms of the interpretation of \( \Abs{det(A)} \) as the volume of \( \calP\).
} % makeproblem

\makeanswer{convex-optimization:problemSet1:6}{
\makeSubAnswer{}{convex-optimization:problemSet1:6a}

TODO.
\makeSubAnswer{}{convex-optimization:problemSet1:6b}

TODO.
\makeSubAnswer{}{convex-optimization:problemSet1:6c}

TODO.
\makeSubAnswer{}{convex-optimization:problemSet1:6d}

TODO.
}
