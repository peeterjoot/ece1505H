%
% Copyright � 2017 Peeter Joot.  All Rights Reserved.
% Licenced as described in the file LICENSE under the root directory of this GIT repository.
%
%{
\input{../latex/blogpost.tex}
\renewcommand{\basename}{intro}
%\renewcommand{\dirname}{notes/phy1520/}
\renewcommand{\dirname}{notes/ece1228-electromagnetic-theory/}
%\newcommand{\dateintitle}{}
%\newcommand{\keywords}{}

\input{../latex/peeter_prologue_print2.tex}

\usepackage{peeters_layout_exercise}
\usepackage{peeters_braket}
\usepackage{peeters_figures}
\usepackage{siunitx}
%\usepackage{mhchem} % \ce{}
%\usepackage{macros_bm} % \bcM
%\usepackage{macros_qed} % \qedmarker
%\usepackage{txfonts} % \ointclockwise

\beginArtNoToc

\generatetitle{XXX}
%\chapter{XXX}
%\label{chap:intro}
\section{What's this course about}

\begin{itemize}
\item Science of optimization.
\item problem formulation, design, analysis of eng systems
\end{itemize}

\section{Basic concepts}

\begin{itemize}
\item Basic concepts.  convex sets, functions, problems.
\item Theory (40 percent).  Lagrangian duality.
\item Algorithms: gradient descent, newton's, interior point.
\end{itemize}

Homework will involve computational work (solving problems, ...)

\section{goals}

\begin{itemize}
\item Recognize and formulate engineering problems as convex optimization problems.
\item To develop (Matlab) code to solve problems numerically.
\item To characterize the solutions via duality theory
\item NOT a math course, but lots of proofs.
\item NOT a communications course, but lots of ex?.
\item NOT a CS course, but lots of useful ideas.
\end{itemize}

Mathematical program:

\begin{dmath}\label{eqn:intro:n}
\min_\Bx F_0(\Bx)
\end{dmath}

where \( \Bx = (x_1, x_2, \cdots, x_m) \el \R{m} \) is subject to

\begin{dmath}\label{eqn:intro:n}
F_i(\Bx) \le 0, \qquad i = 1, \cdots, m
\end{dmath}

The functions:

\begin{dmath}\label{eqn:intro:n}
F_0 : \R{m} \rightarrow \R{1}
\end{dmath}

is the ``objective function'', and

\begin{dmath}\label{eqn:intro:n}
F_i : \R{m} \rightarrow \R{1}
\end{dmath}

are the constraints.

Solving a problem produces:

An optimal \(\Bx^\conj\) is an \( \Bx \) that gives the smallest value among all the feasible \( \Bx \).

F1: F_0(\Bx) vs x_1, x_2

\begin{itemize}
\item A convex objective looks like a bowl, ``holds water''.
\item If connect two feasible points line segment in the ? above bottom of the bowl.
\end{itemize}

F2: Non-convex: wavy figure with a number of local minimums

Example:

Line fitting.

F3: points in an x,y plane with various points and a line \( y = a x + b \) passing through them.  Here \( a, b \) are the optimization variables \( \Bx = (a, b) \).

Idea1.

Describe an error function, describing how far from the line a given point is.

\begin{dmath}\label{eqn:intro:n}
y_i - (a x_i + b),
\end{dmath}

Because this can be positive or negative, we can define a squared variant of this, and then sum over all data points.

\begin{dmath}\label{eqn:intro:n}
F_0 = \sum_{i=1}^N \lr{ y_i - (a x_i + b) }^2
\end{dmath}

One way to solve (for \( a, b \)): Take the derivatives

\begin{dmath}\label{eqn:intro:n}
\begin{aligned}
\PD{a}{F_0} &= \sum_{i=1}^N 2 ( y_i - (a x_i + b) )(-x_i) = 0
\PD{b}{F_0} &= \sum_{i=1}^N 2 ( y_i - (a x_i + b) )(-1) = 0
\end{aligned}
\end{dmath}

Yields

\begin{dmath}\label{eqn:intro:n}
\begin{aligned}
\sum_{i = 1}^N y_i     &= \lr{\sum_{i = 1}^N x_i} a + \lr{\sum_{i = 1}^N 1} b \\
\sum_{i = 1}^N x_i y_i &= \lr{\sum_{i = 1}^N x_i^2} a + \lr{\sum_{i = 1}^N x_i} b
\end{aligned}
\end{dmath}

In matrix form
\begin{dmath}\label{eqn:intro:n}
\begin{bmatrix}
\sum x_i y_i \\
\sum y_i
\end{bmatrix}
=
\begin{bmatrix}
\sum x_i^2 & \sum x_i \\
\sum x_i & n 
\end{bmatrix}
\begin{bmatrix}
a \\
b
\end{bmatrix}.
\end{dmath}

If invertible, have an analytic solution for \( (a^\conj, b^\conj) \).

This is a convex optimization problem because \( F(x) = x^2 \) is a convex ``quadratic program''.  

A quadratic program is

\begin{dmath}\label{eqn:intro:n}
F(a, b) = (\cdots) a^2 + (\cdots) a b + (\cdots) b^2.
\end{dmath}

\paragraph{Alternate approach}

\begin{dmath}\label{eqn:intro:n}
\begin{bmatrix}
y_1 \\
\vdots
y_N
\end{bmatrix}
=
\begin{bmatrix}
x_1 1 \\
\vdots
x_N 1
\end{bmatrix}
\begin{bmatrix}
a \\
b
\end{bmatrix}
+
\begin{bmatrix}
z_1 \\
\vdots
z_N 
\end{bmatrix}
,
\end{dmath}

or
\begin{dmath}\label{eqn:intro:n}
\By = H \Bv + \BZ.
\end{dmath}

here \( \BZ \) is the error vector.

The problem is now to:

Fit \( \By \) to be as close to \( H v + Z \) as possible, or to minimize the norm of the error vector

\begin{dmath}\label{eqn:intro:n}
\min_\Bv \Norm{ \By - H \Bv }^2_2 
= \min_\Bv \lr{ \By - H \Bv }^\T \lr{ \By - H \Bv }
= \min_\Bv 
\lr{ \By^\T \By - 2 \By^\T H \Bv + \Bv^\T H^\T H \Bv }.
\end{dmath}

Can now take the derivative with respect to the \( \Bv \) vector

\begin{dmath}\label{eqn:intro:n}
\PD{\Bv}{} 
\lr{ \By^\T \By - 2 \By^\T H \Bv + \Bv^\T H^\T H \Bv }
=
- 2 \By^\T H + 2 \Bv^\T H^\T H
= 0,
\end{dmath}

or

\begin{dmath}\label{eqn:intro:n}
(H^\T H) \Bv = H^\T \By,
\end{dmath}

so, assuming that \( H^\T H \) is invertible

\begin{dmath}\label{eqn:intro:n}
\Bv^\conj = 
(H^\T H)^{-1} H^\T \By,
\end{dmath}

where

\begin{dmath}\label{eqn:intro:n}
H^\T H = 
\begin{bmatrix}
x_1 & \cdots & x_N \\
 1  & \cdots & 1   \\
\end{bmatrix}
\begin{bmatrix}
x_1 1 \\
\vdots
x_N 1
\end{bmatrix}.
\end{dmath}

\section{Why pick the 2-norm for the objective function?}

\begin{itemize}
\item We do this in calculus, because we can solve the derivative equation.
\item One justification: In statistics the error vector \( \Bz = \By - H \Bv \) can be modelled as an independently and identically distributed Gaussian random variable (i.e. noise).  Under this model, we've solved an ML estimation problem (see chapter 7 citep...).
\end{itemize}

Model

\begin{dmath}\label{eqn:intro:n}
y_i = a x_i + b
\end{dmath}

\begin{dmath}\label{eqn:intro:n}
z_i = y_i - a x_i -b \sim N(O, O^2)
\end{dmath}

\begin{dmath}\label{eqn:intro:n}
P_Z(z) = \inv{\sqrt{2 \pi \sigma}} \exp\lr{ -\inv{2} z^2/\sigma^2 }.
\end{dmath}

\paragraph{MLE: Maximum Likelyhood Estimator}

Pick \( (a,b) \) to maximize the probability of observed data.

\begin{dmath}\label{eqn:intro:n}
(a^\conj, b^\conj) 
= \arg \max P( x, y ; a, b )
= \arg \max P_Z( y - (a x + b) )
= \arg \max \product_{i = 1}^n 
= \arg \max \inv{\sqrt{2 \pi \sigma}} \exp\lr{ -\inv{2} (y_i - a x_i - b)^2/\sigma^2 }.
\end{dmath}

Taking logs gives
\begin{dmath}\label{eqn:intro:n}
(a^\conj, b^\conj) 
= \arg \max 
\lr{ 
\textrm{constant}
   -\inv{2} \sum_i (y_i - a x_i - b)^2/\sigma^2 
}
= \arg \min 
   \inv{2} \sum_i (y_i - a x_i - b)^2/\sigma^2 
= \arg \min 
   \sum_i (y_i - a x_i - b)^2/\sigma^2 
\end{dmath}

Here \( \arg \max \) is not the maximum of the function, but the value of the parameter (the argument) that maximizes the function.

\paragraph{Double sides exponential noise}

F5: %abs(e^{-x^2}}

\begin{dmath}\label{eqn:intro:n}
P_Z(z) = \inv{2 c} \exp\lr{ -\inv{c} \Abs{z} }.
\end{dmath}

\max_{a,b} \product_{i = 1}^n P_z(z_i)
=
\max_{a,b} \product_{i = 1}^n 
\inv{2 c} \exp\lr{ -\inv{c} \Abs{z_i} }.
=
\max_{a,b} \product_{i = 1}^n 
\inv{2 c} \exp\lr{ -\inv{c} \Abs{y_i - a x_i - b} }
=
\max_{a,b} 
\inv{2 c}^n \exp\lr{ -\inv{c} \Sum \Abs{y_i - a x_i - b} }

This is a L1 norm problem

\begin{dmath}\label{eqn:intro:n}
\min_{a,b} \sum_{i = 1}^n \Abs{ y_i - a x_i - b }.
\end{dmath}

i.e.

\begin{dmath}\label{eqn:intro:n}
\Norm{ \By = H \Bv }_1.
\end{dmath}

This is still convex, but has no analytic solution

F6: %F(x) = \Abs{x}

This is an example of a linear program.

\section{Solution of linear program}

Introduce helper variables \( t_1, \cdots, t_n \), and 
minimize \( \sum_i t_i \), such that

\begin{dmath}\label{eqn:intro:n}
\Abs{ y_i - a x_i - b } \le t_i,
\end{dmath}

This is now an optimization problem for \( a, b, t_1, \cdots t_n \).  A linear program is defined as

\begin{dmath}\label{eqn:intro:n}
\min_{a, b, t_1, \cdots t_n} \sum_i t_i 
\end{dmath}

such that
\begin{dmath}\label{eqn:intro:n}
\begin{aligned}
y_i - a x_i - b \le t_i
y_i - a x_i - b \ge -t_i
\end{aligned}
\end{dmath}

\paragraph{Single sided exponential}

What if your noise doesn't look double sided, with only noise for values \( x > 0 \)

F7: % \theta(x) e^-x/c

Can define a single sided probability distribution

\begin{dmath}\label{eqn:intro:n}
P_Z(z) = 
\left\{
\begin{array}{l l}
\inv{c} e^{-z/c} & \quad \mbox{\( z \ge 0 \)} \\
0 & \quad \mbox{\( z < 0 \)} 
\end{array}
\right.
\end{dmath}

i.e. all \( z_i \) error values are always non-negative.

\begin{dmath}\label{eqn:intro:n}
\log P_z(z) = 
\left\{
\begin{array}{l l}
\textrm{const} - z/c & \quad \mbox{\( z > 0\)} \\
-\infty & \quad \mbox{\( z< 0\)} 
\end{array}
\right.
\end{dmath}

Problem becomes

\begin{dmath}\label{eqn:intro:n}
\min_{a, b} \sum_i \lr{ y_i - a x_i - b)
\end{dmath}

such that
\begin{dmath}\label{eqn:intro:n}
\begin{aligned}
y_i - a x_i - b \ge t_i \forall i
\end{aligned}
\end{dmath}

\paragraph{Uniform noise}

F8: constant in range -c,c, zero outside that range.

\begin{dmath}\label{eqn:intro:n}
P_Z(z) = 
\left\{
\begin{array}{l l}
\inv{2 c} & \quad \mbox{\( \Abs{z} \le c\)} \\
0 & \quad \mbox{\( \Abs{z} > c. \)} 
\end{array}
\right.
\end{dmath}

or

\begin{dmath}\label{eqn:intro:n}
\log P_Z(z) = 
\left\{
\begin{array}{l l}
\textrm{const} & \quad \mbox{\( \Abs{z} \le c\)} \\
-\infty & \quad \mbox{\( \Abs{z} > c. \)} 
\end{array}
\right.
\end{dmath}

MLE solution

\max_{a,b} \product_{i = 1}^n P(x, y; a, b) 
=
\max_{a,b} \sum_{i = 1}^n \log P_Z( y_i - a x_i - b )

Here the argument is constant if \( -c \le y_i - a x_i - b \le c \), so an ML solution is \underline{any} \( (a,b) \) such that

\begin{dmath}\label{eqn:intro:n}
\Abs{ y_i - a x_i - b } \le c \forall i \in 1, \cdots, n.
\end{dmath}

This is a linear program known as a ``feasibility problem''.

\begin{dmath}\label{eqn:intro:n}
\min d
\end{dmath}

such that

\begin{dmath}\label{eqn:intro:n}
\begin{aligned}
y_i - a x_i - b &\le d \\
y_i - a x_i - b &\ge -d
\end{aligned}
\end{dmath}

If \( d^\conj \le c \), then the problem is feasible, however, if \( d^\conj > c \) it is infeasible.

Compare three columns: 
- two sided exp
- one sided exp
- constant range

three point plots respectively
- points on both sides of line
- points only above the line
- error bars passing through the line (or lines)

Friday: review of multivariable calc and algebra.

%}
%\EndArticle
\EndNoBibArticle
