%
% Copyright � 2017 Peeter Joot.  All Rights Reserved.
% Licenced as described in the file LICENSE under the root directory of this GIT repository.
%
\section{Matrix inner product}

Given real matrices \( X, Y \in \bbR^{m\times n} \), one possible matrix inner product definition is

\begin{dmath}\label{eqn:convexOptimizationLecture3:20}
\innerprod{X}{Y}
= \Tr( X^\T Y)
= \Tr \lr{ \sum_{k = 1}^m X_{ki} Y_{kj} }
= \sum_{k = 1}^m \sum_{j = 1}^n X_{kj} Y_{kj}
= \sum_{i = 1}^m \sum_{j = 1}^n X_{ij} Y_{ij}.
\end{dmath}

This inner product induces a norm on the (matrix) vector space, called the Frobenius norm

\begin{dmath}\label{eqn:convexOptimizationLecture3:40}
\Norm{X }_F
= \Tr( X^\T X)
= \sqrt{
\innerprod{X}{X} }
=
\sum_{i = 1}^m \sum_{j = 1}^n X_{ij}^2.
\end{dmath}

\section{Range, nullspace.}

\makedefinition{Range.}{dfn:convexOptimizationLecture3:10}{
Given \( A \in \bbR^{m \times n} \), the range of A is the set:

\begin{equation*}
\calR(A) = \setlr{ A \Bx | \Bx \in \bbR^n }.
\end{equation*}
} % definition

\makedefinition{Nullspace.}{dfn:convexOptimizationLecture3:20}{
Given \( A \in \bbR^{m \times n} \),
the nullspace of A is the set:

\begin{equation*}
\calN(A) = \setlr{ \Bx | A \Bx = 0 }.
\end{equation*}
} % definition

\section{SVD.}

To understand operation of \( A \in \bbR^{m \times n} \), a representation of a linear transformation from \R{n} to \R{m},
decompose \( A \) using the singular value decomposition (SVD).

\makedefinition{SVD.}{dfn:convexOptimizationLecture3:40}{
Given \( A \in \bbR^{m \times n} \),
an operator on \( \Bx \in \bbR^n \),
a decomposition of the following form is always possible

\begin{equation*}
\begin{aligned}
A &= U \Sigma V^\T \\
U &\in \bbR^{m \times r} \\
V &\in \bbR^{n \times r},
\end{aligned}
\end{equation*}

where
\( r \) is the rank of \(A\), and
both \( U \) and \( V \) are orthogonal

\begin{equation*}
\begin{aligned}
U^\T U &= I \in \bbR^{r \times r} \\
V^\T V &= I \in \bbR^{r \times r}.
\end{aligned}
\end{equation*}

Here \( \Sigma = \diag( \sigma_1, \sigma_2, \cdots, \sigma_r ) \), is a
diagonal matrix of ``singular'' values, where

\begin{equation*}
\sigma_1 \ge \sigma_2 \ge \cdots \ge \sigma_r.
\end{equation*}
} % definition

For simplicity consider square case \( m = n \)

\begin{dmath}\label{eqn:convexOptimizationLecture3:100}
A \Bx = \lr{ U \Sigma V^\T } \Bx.
\end{dmath}

The first product \( V^\T \Bx \) is a rotation, which can be checked by looking at the length

\begin{dmath}\label{eqn:convexOptimizationLecture3:120}
\Norm{ V^\T \Bx}_2
= \sqrt{ \Bx^\T V V^\T \Bx }
= \sqrt{ \Bx^\T \Bx }
= \Norm{ \Bx }_2,
\end{dmath}

which shows that the length of the vector is unchanged after application of the linear transformation represented by \( V^\T \) so that operation must be a rotation.

Similarly the operation of \( U \) on \( \Sigma V^\T \Bx \) also must be a rotation.  The operation \( \Sigma = [\sigma_i]_i \) applies a scaling operation to each component of the vector \( V^\T \Bx \).

All linear (square) transformations can therefore be thought of as a rotate-scale-rotate operation.  Often the \( A \) of interest will be symmetric \( A = A^\T \).

\section{Set of symmetric matrices}

Let \( S^n \) be the set of real, symmetric \( n \times n \) matrices.

\maketheorem{Spectral theorem.}{thm:convexOptimizationLecture3:50}{
When \( A \in S^n \) then it is possible to factor \( A \) as

\begin{equation*}
A = Q \Lambda Q^\T,
\end{equation*}

where \( Q \) is an orthogonal matrix, and \( \Lambda = \diag( \lambda_1, \lambda_2, \cdots \lambda_n)\).  Here \( \lambda_i \in \bbR \, \forall i \) are the (real) eigenvalues of \( A \).

A real symmetric matrix \( A \in S^n\)
is ``positive semi-definite'' if

\begin{equation*}
\Bv^\T A \Bv \ge 0 \qquad\forall \Bv \in \bbR^n, \Bv \ne 0,
\end{equation*}
and is ``positive definite'' if

\begin{equation*}
\Bv^\T A \Bv > 0 \qquad\forall \Bv \in \bbR^n, \Bv \ne 0.
\end{equation*}

The set of such matrices is denoted \( S^n_{+} \), and \( S^n_{++} \) respectively.
} % theorem

Consider \( A \in S^n_{+} \) (or \( S^n_{++} \) )

\begin{dmath}\label{eqn:convexOptimizationLecture3:200}
A = Q \Lambda Q^\T,
\end{dmath}

possible since the matrix is symmetric.  For such a matrix

\begin{dmath}\label{eqn:convexOptimizationLecture3:220}
\Bv^\T A \Bv
=
\Bv^\T Q \Lambda A^\T \Bv
=
\Bw^\T \Lambda \Bw,
\end{dmath}

where \( \Bw = A^\T \Bv \).  Such a product is

\begin{dmath}\label{eqn:convexOptimizationLecture3:240}
\Bv^\T A \Bv
=
\sum_{i = 1}^n \lambda_i w_i^2.
\end{dmath}

So, if \( \lambda_i \ge 0 \) (\(\lambda_i > 0 \) ) then \( \sum_{i = 1}^n \lambda_i w_i^2 \) is non-negative (positive) \( \forall \Bw \in \bbR^n, \Bw \ne 0 \).  Since \( \Bw \) is just a rotated version of \( \Bv \) this also holds for all \( \Bv \).  A necessary and sufficient condition for \( A \in S^n_{+} \) (\( S^n_{++} \) ) is \( \lambda_i \ge 0 \) (\(\lambda_i > 0\)).

\section{Square root of positive semi-definite matrix}

Real symmetric matrix power relationships such as

\begin{dmath}\label{eqn:convexOptimizationLecture3:260}
A^2
=
Q \Lambda Q^\T
Q \Lambda Q^\T
=
Q \Lambda^2
Q^\T
,
\end{dmath}

or more generally \( A^k = Q \Lambda^k Q^\T,\, k \in \bbZ \), can be further generalized to non-integral powers.  In particular, the square root (non-unique) of a square matrix can be written

\begin{dmath}\label{eqn:convexOptimizationLecture3:280}
A^{1/2} = Q
\begin{bmatrix}
\sqrt{\lambda_1} &                  &        &  \\
                 & \sqrt{\lambda_2} &        & \\
                 &                  & \ddots & \\
                 &                  &        & \sqrt{\lambda_n} \\
\end{bmatrix}
Q^\T,
\end{dmath}

since \( A^{1/2} A^{1/2} = A \), regardless of the sign picked for the square roots in question.

\section{Functions of matrices}

Consider \( F : S^n \rightarrow \bbR \), and define

\begin{dmath}\label{eqn:convexOptimizationLecture3:300}
F(X) = \log \det X,
\end{dmath}

Here \( \dom F = S^n_{++} \).
The task is to find \( \spacegrad F \), which can be done by looking at the perturbation
\( \log \det ( X + \Delta X ) \)

\begin{dmath}\label{eqn:convexOptimizationLecture3:320}
\log \det ( X + \Delta X )
=
\log \det ( X^{1/2} (I + X^{-1/2} \Delta X X^{-1/2}) X^{1/2} )
=
\log \det ( X (I + X^{-1/2} \Delta X X^{-1/2}) )
=
\log \det X  + \log \det (I + X^{-1/2} \Delta X X^{-1/2}).
\end{dmath}

Let \( X^{-1/2} \Delta X X^{-1/2} = M \) where \( \lambda_i \) are the eigenvalues of \( M : M \Bv = \lambda_i \Bv \) when \( \Bv \) is an eigenvector of \( M \).  In particular

\begin{dmath}\label{eqn:convexOptimizationLecture3:340}
(I + M) \Bv =
(1 + \lambda_i) \Bv,
\end{dmath}

where \( 1 + \lambda_i \) are the eigenvalues of the \( I + M \) matrix.  Since the determinant is the product of the eigenvalues, this gives

\begin{dmath}\label{eqn:convexOptimizationLecture3:360}
\log \det ( X + \Delta X )
=
\log \det X +
\log \prod_{i = 1}^n (1 + \lambda_i)
=
\log \det X +
\sum_{i = 1}^n \log (1 + \lambda_i).
\end{dmath}

If \( \lambda_i \) are sufficiently ``small'', then \( \log ( 1 + \lambda_i ) \approx \lambda_i \), giving

\begin{dmath}\label{eqn:convexOptimizationLecture3:380}
\log \det ( X + \Delta X )
=
\log \det X +
\sum_{i = 1}^n \lambda_i
\approx
\log \det X +
\Tr( X^{-1/2} \Delta X X^{-1/2} ).
\end{dmath}

Since
\begin{dmath}\label{eqn:convexOptimizationLecture3:400}
\Tr( A B ) = \Tr( B A ),
\end{dmath}

% also used above:
%%%\det (A B) = \det (BA) = \det A \det B

this trace operation can be written as

\begin{dmath}\label{eqn:convexOptimizationLecture3:420}
\log \det ( X + \Delta X )
\approx
\log \det X +
\Tr( X^{-1} \Delta X )
=
\log \det X +
\innerprod{ X^{-1}}{\Delta X},
\end{dmath}

so
\begin{dmath}\label{eqn:convexOptimizationLecture3:440}
\spacegrad F(X) = X^{-1}.
\end{dmath}

%%This trace is the ``matrix inner product''.

To check this, consider the simplest example with \( X \in \bbR^{1 \times 1} \), where we have

\begin{dmath}\label{eqn:convexOptimizationLecture3:460}
\frac{d}{dX} \lr{ \log \det X } = \frac{d}{dX} \lr{ \log X } = \inv{X} = X^{-1}.
\end{dmath}

This is a nice example demonstrating how the gradient can be obtained by performing a
first order perturbation of the function.  The gradient can then be read off from the result.

\section{Second order perturbations}

\begin{itemize}
\item To get first order approximation found the part that varied linearly in \( \Delta X \).
\item To get the second order part, perturb \( X^{-1} \) by \( \Delta X \) and see how that perturbation varies in \( \Delta X \).
\end{itemize}

For \( G(X) = X^{-1} \), this is

\begin{dmath}\label{eqn:convexOptimizationLecture3:480}
(X + \Delta X)^{-1}
=
\lr{ X^{1/2} (I + X^{-1/2} \Delta X X^{-1/2} ) X^{1/2} }^{-1}
=
X^{-1/2} (I + X^{-1/2} \Delta X X^{-1/2} )^{-1} X^{-1/2}
\end{dmath}

To be proven in the homework (for ``small'' A)

\begin{dmath}\label{eqn:convexOptimizationLecture3:500}
(I + A)^{-1} \approx I - A.
\end{dmath}

This gives

\begin{dmath}\label{eqn:convexOptimizationLecture3:520}
(X + \Delta X)^{-1}
=
X^{-1/2} (I - X^{-1/2} \Delta X X^{-1/2} ) X^{-1/2}
=
X^{-1} - X^{-1} \Delta X X^{-1},
\end{dmath}

or

\begin{dmath}\label{eqn:convexOptimizationLecture3:800}
G(X + \Delta X)
= G(X) + (D G) \Delta X
= G(X) + (\spacegrad G)^\T \Delta X,
\end{dmath}

so
\begin{dmath}\label{eqn:convexOptimizationLecture3:820}
(\spacegrad G)^\T \Delta X
=
- X^{-1} \Delta X X^{-1}.
\end{dmath}

The Taylor expansion of \( F \) to second order is

\begin{dmath}\label{eqn:convexOptimizationLecture3:840}
F(X + \Delta X)
=
F(X)
+
\Tr \lr{ (\spacegrad F)^\T \Delta X}
+
\inv{2}
\lr{ (\Delta X)^\T (\spacegrad^2 F) \Delta X}.
%=
%F(X)
%+
%\Tr \lr{ (\spacegrad F)^\T \Delta X}
%+
%\inv{2}
%\Tr \lr{ (\Delta X)^\T (\spacegrad^2 F) \Delta X},
\end{dmath}

The first trace can be expressed as an inner product

\begin{dmath}\label{eqn:convexOptimizationLecture3:860}
\Tr \lr{ (\spacegrad F)^\T \Delta X}
=
\innerprod{ \spacegrad F }{\Delta X}
=
\innerprod{ X^{-1} }{\Delta X}.
\end{dmath}

The second trace also has the structure of an inner product

\begin{dmath}\label{eqn:convexOptimizationLecture3:880}
(\Delta X)^\T (\spacegrad^2 F) \Delta X
=
\Tr \lr{ (\Delta X)^\T (\spacegrad^2 F) \Delta X}
=
\innerprod{ (\spacegrad^2 F)^\T \Delta X }{\Delta X},
\end{dmath}

where a no-op trace could be inserted in the second order term since that quadratic form is already a scalar.
This \( (\spacegrad^2 F)^\T \Delta X \) term has essentially been found implicitly by performing the linear variation of \( \spacegrad F \) in \( \Delta X \), showing that we must have

\begin{dmath}\label{eqn:convexOptimizationLecture3:900}
\Tr \lr{ (\Delta X)^\T (\spacegrad^2 F) \Delta X}
=
\innerprod{ - X^{-1} \Delta X X^{-1} }{\Delta X},
\end{dmath}

%Since the term \( X^{-1} \Delta X X^{-1} \) varies linearly in \( \Delta X \) the first and second order changes are respectively
so
\begin{dmath}\label{eqn:convexOptimizationLecture3:560}
F( X + \Delta X) = F(X) +
\innerprod{X^{-1}}{\Delta X}
+\inv{2} \innerprod{-X^{-1} \Delta X X^{-1}}{\Delta X},
\end{dmath}

or
\begin{dmath}\label{eqn:convexOptimizationLecture3:580}
\log \det ( X + \Delta X) = \log \det X +
\Tr( X^{-1} \Delta X )
- \inv{2} \Tr( X^{-1} \Delta X X^{-1} \Delta X ).
\end{dmath}

%FIXME: The trace operations in
%\cref{eqn:convexOptimizationLecture3:840} weren't in the
%Jacobian and Hessian expansion in lecture 2, nor in the text?

\section{Convex Sets}

\begin{itemize}
\item Types of sets: Affine, convex, cones
\item Examples: Hyperplanes, polyhedra, balls, ellipses, norm balls, cone of PSD matrices.
\end{itemize}

\makedefinition{Affine set}{dfn:convexOptimizationLecture3:1}{

A set \( C \subseteq \bbR^n \) is affine if \( \forall \Bx_1, \Bx_2 \in C \) then

\begin{equation*}
\theta \Bx_1 + (1 -\theta) \Bx_2 \in C, \qquad \forall \theta \in \bbR.
\end{equation*}
} % definition

The affine sum above can
be rewritten as

\begin{dmath}\label{eqn:convexOptimizationLecture3:600}
\Bx_2 + \theta (\Bx_1 - \Bx_2).
\end{dmath}

Since \( \theta \) is a scaling, this is the line containing \( \Bx_2 \) in the direction between \( \Bx_1 \) and \( \Bx_2 \).

Observe that the solution to a set of linear equations

\begin{equation}\label{eqn:convexOptimizationLecture3:620}
C = \setlr{ \Bx | A \Bx = \Bb },
\end{equation}

is an affine set.  To check, note that

\begin{dmath}\label{eqn:convexOptimizationLecture3:640}
A (\theta \Bx_1 + (1 - \theta) \Bx_2)
=
\theta A \Bx_1 + (1 - \theta) A \Bx_2
=
\theta \Bb + (1 - \theta) \Bb
= \Bb.
\end{dmath}

\makedefinition{Affine combination.}{dfn:convexOptimizationLecture3:71}{
An affine combination of points \( \Bx_1, \Bx_2, \cdots \Bx_n \) is

\begin{equation*}
\sum_{i = 1}^n \theta_i \Bx_i,
\end{equation*}

such that for \( \theta_i \in \bbR \)

\begin{equation*}
\sum_{i = 1}^n \theta_i = 1.
\end{equation*}
} % definition

An affine set contains all affine combinations of points in the set.  Examples of a couple
affine sets are sketched in
\cref{fig:l3AffineFig1and2}.

\imageTwoFigures
{../figures/ece1505-convex-optimization/l3AffineFig1}
{../figures/ece1505-convex-optimization/l3AffineFig2}
{Affine.}
{fig:l3AffineFig1and2}
{scale=0.1}

For comparison, a couple of non-affine sets are sketched in \cref
{fig:l3NotAffineFig3and4}.

\imageTwoFigures
{../figures/ece1505-convex-optimization/l3NotAffineFig3}
{../figures/ece1505-convex-optimization/l3NotAffineFig4}
{Not affine.}
{fig:l3NotAffineFig3and4}
{scale=0.1}

\makedefinition{Convex set}{dfn:convexOptimizationLecture3:2}{

A set \( C \subseteq \bbR^n \) is convex if \( \forall \Bx_1, \Bx_2 \in C \) and \( \forall \theta \in \bbR, \theta \in [0,1] \), the combination

\begin{dmath}\label{eqn:convexOptimizationLecture3:700}
\theta \Bx_1 + (1 - \theta) \Bx_2 \in C.
\end{dmath}
} % definition

\makedefinition{Convex combination}{dfn:convexOptimizationLecture3:3}{

A convex combination of \( \Bx_1, \Bx_2, \cdots \Bx_n \) is

\begin{equation*}
\sum_{i = 1}^n \theta_i \Bx_i,
\end{equation*}

such that \( \forall \theta_i \ge 0 \)

\begin{equation*}
\sum_{i = 1}^n \theta_i = 1
\end{equation*}

} % definition

\makedefinition{Convex hull.}{dfn:convexOptimizationLecture3:3a}{

Convex hull of a set \( C \) is a set of all convex combinations of points in \(C\), denoted

\begin{equation*}
\conv(C) = \setlr{ \sum_{i=1}^n \theta_i \Bx_i | \Bx_i \in C, \theta_i \ge 0, \sum_{i=1}^n \theta_i = 1 }.
\end{equation*}
} % definition

A non-convex set can be converted into a convex hull by filling in all the combinations of points connecting points in the set, as sketched in \cref
{fig:l3NotAffineHull:l3NotAffineHullFig6}.

\imageTwoFigures{../figures/ece1505-convex-optimization/l3NotAffineHullFig5}
{../figures/ece1505-convex-optimization/l3NotAffineHullFig6}
{Convex hulls.}
{fig:l3NotAffineHull:l3NotAffineHullFig6}
{scale=0.1}

\makedefinition{Cones.}{dfn:convexOptimizationLecture3:4}{

A set \(C\) is a cone if \( \forall \Bx \in C \) and \( \forall \theta \ge 0 \) we have \( \theta \Bx \in C\).
} % definition

This scales out if \(\theta > 1\) and scales in if \(\theta < 1\).

A convex cone is a cone that is also a convex set.  A conic combination is

\begin{equation*}
\sum_{i=1}^n \theta_i \Bx_i, \theta_i \ge 0.
\end{equation*}

A convex and non-convex 2D cone is sketched in \cref{fig:ConesFig8}

\imageTwoFigures{../figures/ece1505-convex-optimization/l3ConvexConeFig7}
{../figures/ece1505-convex-optimization/l3NotConvexConeFig8}
{Convex and non-convex cone.}
{fig:ConesFig8}
{scale=0.1}

Like the convex null, it is possible to define affine and conic hulls.  These are

\makedefinition{Affine hull.}{dfn:convexOptimizationLecture3:920}{

Affine hull of a set \( C \) is a set of all affine combinations of points in \(C\), denoted

\begin{equation*}
\affine(C) = \setlr{ \sum_{i=1}^n \theta_i \Bx_i | \Bx_i \in C, \theta_i \in \bbR, \sum_{i=1}^n \theta_i = 1 }.
\end{equation*}
} % definition

\makedefinition{Conic hull.}{dfn:convexOptimizationLecture3:940}{

Conic hull of a set \( C \) is a set of all conic combinations of points in \(C\), denoted

\begin{equation*}
\conic(C) = \setlr{ \sum_{i=1}^n \theta_i \Bx_i | \Bx_i \in C, \theta_i \ge 0 }.
\end{equation*}
} % definition

% Fixme: this prevents the table from being embedded in a definition.  Is there a way to
% force the table to be emmitted after the definition without clearpage?
\clearpage

A comparison of these three types of hulls are tabulated in \cref{tab:affineConvexConic:1}.

\captionedTable{Affine, Convex, and Conic properties.}{tab:affineConvexConic:1}{
\centering
\begin{tabular}{l|l|l|}
\cline{2-3}
                             & \( \theta_i \ge 0 \) & \( \sum \theta_i = 1 \) \\ \hline
\multicolumn{1}{|l|}{Affine} & No                   & Yes                     \\ \hline
\multicolumn{1}{|l|}{Convex} & Yes                  & Yes                     \\ \hline
\multicolumn{1}{|l|}{Conic}  & Yes                  & No                      \\ \hline
\end{tabular}
}

\section{Hyperplanes and half spaces}

\makedefinition{Hyperplane.}{dfn:convexOptimizationLecture3:99}{
A hyperplane is defined by

\begin{equation*}
\setlr{ \Bx | \Ba^\T \Bx = \Bb, \Ba \ne 0 }.
\end{equation*}
} % definition

A line and plane are examples of this general construct as sketched in
\cref{fig:l3hyperPlaneLine:l3hyperPlanesFig9}.

\imageTwoFigures{../figures/ece1505-convex-optimization/l3hyperPlaneLineFig9}
{../figures/ece1505-convex-optimization/l3hyperplanePlaneFig10}
{Hyperplanes.}
{fig:l3hyperPlaneLine:l3hyperPlanesFig9}
{scale=0.1}

An alternate view is possible should one
find any specific \( \Bx_0 \) such that \( \Ba^\T \Bx_0 = \Bb \)

\begin{dmath}\label{eqn:convexOptimizationLecture3:740}
\setlr{\Bx | \Ba^\T \Bx = b }
=
\setlr{\Bx | \Ba^\T (\Bx -\Bx_0) = 0 }
\end{dmath}

This shows that \( \Bx - \Bx_0 = \Ba^\perp \) is perpendicular to \( \Ba \), or

\begin{dmath}\label{eqn:convexOptimizationLecture3:780}
\Bx
=
\Bx_0 + \Ba^\perp.
\end{dmath}

This is the subspace perpendicular to \( \Ba \) shifted by \(\Bx_0\), subject to \( \Ba^\T \Bx_0 = \Bb \).  As a set

\begin{equation}\label{eqn:convexOptimizationLecture3:760}
\Ba^\perp = \setlr{ \Bv | \Ba^\T \Bv = 0 }.
\end{equation}

\section{Half space}

\makedefinition{Half space.}{dfn:convexOptimizationLecture3:101}{

The half space is defined as
\begin{equation*}
\setlr{ \Bx | \Ba^\T \Bx = \Bb }
= \setlr{ \Bx | \Ba^\T (\Bx - \Bx_0) \le 0 }.
\end{equation*}
} % definition

This can also be expressed as \( \setlr{ \Bx | \innerprod{ \Ba }{\Bx - \Bx_0 } \le 0 } \).
