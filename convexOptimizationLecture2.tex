%
% Copyright � 2016 Peeter Joot.  All Rights Reserved.
% Licenced as described in the file LICENSE under the root directory of this GIT repository.
%
\input{../latex/blogpost.tex}
\renewcommand{\basename}{convex-optimization2}
\renewcommand{\dirname}{notes/ece1505/}
\newcommand{\keywords}{ECE1505H}
\input{../latex/peeter_prologue_print2.tex}

\usepackage{ece1505}
\usepackage{peeters_braket}
%\usepackage{peeters_layout_exercise}
\usepackage{peeters_figures}
\usepackage{mathtools}
\usepackage{siunitx}
\usepackage{peeters_layout_exercise}

\beginArtNoToc
\generatetitle{ECE1505H Convex Optimization.  Lecture 2: Mathematical background.  Taught by Prof.\ Stark Draper}
%\chapter{Mathematical background}
\label{chap:convex-optimization2}

\paragraph{Disclaimer}

Peeter's lecture notes from class.  These may be incoherent and rough.

These are notes for the UofT course ECE1505H, Convex Optimization, taught by Prof. Stark Draper, covering \textchapref{{1}} \citep{boyd2004convex} content.

\section{Mathematical background}

\begin{itemize}
\item Calculus: Derivatives and Jacobians, Gradients, Hessians, approximation functions.
\item Linear algebra, Matrices, decompositions, ...
\end{itemize}

\subsection{Norms}
\paragraph{Vector space}

\makedefinition{Vector space}{dfn:convex-optimizationLecture2:1}{
A set of elements (vectors) that is closed under vector addition and scaling.
} % definition

\makedefinition{Normed vector spaces}{dfn:convex-optimizationLecture2:2}{
A vector space with a notion of lenght of any signle vector, the ``norm''.
} % definition

\makedefinition{Inner product space.}{dfn:convex-optimizationLecture2:3}{
A normed vector space with a notion of a real angle vetween any pair of vectors.
} % definition

This course has a focus on optimization in \R{n}.  Complex spaces in the context of this course can be considered with a mapping \( \text{\C{n}} \rightarrow \Rm{2 n} \).

\makedefinition{Norm.}{dfn:convex-optimizationLecture2:4}{
A norm is a function operating on a vector

\begin{dmath*}
\Bx = ( x_1, x_2, \cdots, x_n )
\end{dmath*}

that provides a mapping

\begin{equation*}
\Norm{ \cdot } : \Rm{n} \rightarrow \bbR,
\end{equation*}

where

\begin{itemize}
\item \( \Norm{ \Bx } \ge 0 \)
\item \( \Norm{ \Bx } = 0 \qquad \iff \Bx = 0 \)
\item \( \Norm{ t \Bx } = \Abs{t} \Norm{ \Bx } \)
\item \( \Norm{ \Bx + \By } \le \Norm{ \Bx } + \Norm{\By} \).  This is the triangle inequality.
\end{itemize}
} % definition

\paragraph{Example: Euclidean norm}

\begin{dmath}\label{eqn:convex-optimizationLecture2:24}
\Norm{\Bx} = \sqrt{ \sum_{i = 1}^n x_i^2 }
\end{dmath}

\paragraph{Example: \(l_p\)-norms}

\begin{dmath}\label{eqn:convex-optimizationLecture2:44}
\Norm{\Bx}_p = \lr{ \sum_{i = 1}^n \Abs{x_i}^p }^{1/p}.
\end{dmath}

For \( p = 1 \), this is

\begin{dmath}\label{eqn:convex-optimizationLecture2:64}
\Norm{\Bx}_1 = \sum_{i = 1}^n \Abs{x_i},
\end{dmath}

For \( p = 2 \), this is the Euclidean norm \cref{eqn:convex-optimizationLecture2:24}.
For \( p = \infty \), this is

\begin{dmath}\label{eqn:convex-optimizationLecture2:324}
\Norm{\Bx}_\infty = \max_{i = 1}^n \Abs{x_i}.
\end{dmath}

\makedefinition{Unit ball}{dfn:convex-optimizationLecture2:10}{

\begin{dmath*}
\setlr{ \Bx | \Norm{\Bx} \le 1 }
\end{dmath*}

} % definition

The \( l_2 \) norm is not only familiar, but can be ``induced'' by an inner product

\begin{equation}\label{eqn:convex-optimizationLecture2:84}
\innerproduct{\Bx}{\By} = \Bx^\T \By = \sum_{i = 1}^n x_i y_i,
\end{equation}

which is not true for all norms.  The norm induced by this inner product is

\begin{dmath}\label{eqn:convex-optimizationLecture2:104}
\Norm{\Bx}_2 = \sqrt{ \innerproduct{\Bx}{\By} }
\end{dmath}

Inner product spaces have a notion of angle given by

\begin{dmath}\label{eqn:convex-optimizationLecture2:124}
\innerproduct{\Bx}{\By} = \Norm{\Bx} \Norm{\By} \cos \theta,
\end{dmath}

F3

and always satify the Cauchy-Schwartz inequality

\begin{dmath}\label{eqn:convex-optimizationLecture2:144}
\innerproduct{\Bx}{\By} \le \Norm{\Bx}_2 \Norm{\By}_2.
\end{dmath}

In an inner product space we say \( \Bx \) and \( \By \) are orthogonal vectors \( \Bx \perp \By \) if
\( \innerproduct{\Bx}{\By} = 0 \).

F4

\makedefinition{Dual norm}{dfn:convex-optimizationLecture2:20}{
Let \( \Norm{ \cdot } \) be a norm in \R{n}.  The ``dual'' norm \( \Norm{ \cdot }_\conj \) is defined as
\begin{equation*}
\Norm{\Bz}_\conj = \sup_\Bx \setlr{ \Bz^\T \Bx | \Norm{\Bx} \le 1 }.
\end{equation*}

where \( \sup \) is roughly the ``least upper bound''.
\index{sup}

This is a limit over the unit ball of \( \Norm{\cdot} \).
} % definition

\paragraph{\( l_2 \) dual}.

Dual of the \( l_2 \) is the \( l_2 \) norm.

F5

Proof:

\begin{dmath}\label{eqn:convex-optimizationLecture2:164}
\Norm{\Bz}_\conj
= \sup_\Bx \setlr{ \Bz^\T \Bx | \Norm{\Bx}_2 \le 1 }
= \sup_\Bx \setlr{ \Norm{\Bz}_2 \Norm{\Bx}_2 \cos\theta | \Norm{\Bx}_2 \le 1 }
\le \sup_\Bx \setlr{ \Norm{\Bz}_2 \Norm{\Bx}_2 | \Norm{\Bx}_2 \le 1 }
\le
\Norm{\cancel{\Bz}}_2
\Norm{
\frac{\Bz}{ \cancel{\Norm{\Bz}_2} }
}_2
=
\Norm{\Bz}_2.
\end{dmath}

\paragraph{\( l_1 \) dual}.
For \( l_1 \), the dual is the \( l_\infty \) norm.  Proof:

\begin{dmath}\label{eqn:convex-optimizationLecture2:184}
\Norm{\Bz}_\conj
=
\sup_\Bx \setlr{ \Bz^\T \Bx | \Norm{\Bx}_1 \le 1 },
\end{dmath}

but
\begin{dmath}\label{eqn:convex-optimizationLecture2:204}
\Bz^\T \Bx
=
\sum_{i=1}^n z_i x_i \le
\Abs{
\sum_{i=1}^n z_i x_i
}
\le
\sum_{i=1}^n \Abs{z_i x_i },
\end{dmath}

so
\begin{dmath}\label{eqn:convex-optimizationLecture2:224}
\Norm{\Bz}_\conj
=
\sum_{i=1}^n \Abs{z_i}\Abs{ x_i }
\le \lr{ \max_{j=1}^n \Abs{z_j} }
\sum_{i=1}^n \Abs{ x_i }
\le \lr{ \max_{j=1}^n \Abs{z_j} }
=
\Norm{\Bz}_\infty.
\end{dmath}

F6

\paragraph{\( l_\infty \) dual}.

F7
\begin{dmath}\label{eqn:convex-optimizationLecture2:244}
\Norm{\Bz}_\conj
=
\sup_\Bx \setlr{ \Bz^\T \Bx | \Norm{\Bx}_\infty \le 1 }.
\end{dmath}

Here
\begin{dmath}\label{eqn:convex-optimizationLecture2:264}
\Bz^\T \Bx
=
\sum_{i=1}^n z_i x_i
\le
\sum_{i=1}^n \Abs{z_i}\Abs{ x_i }
\le
\lr{ \max_j \Abs{ x_j } }
\sum_{i=1}^n \Abs{z_i}
=
\Norm{\Bx}_\infty
\sum_{i=1}^n \Abs{z_i}.
\end{dmath}

So
\begin{dmath}\label{eqn:convex-optimizationLecture2:284}
\Norm{\Bz}_\conj
\le
\sum_{i=1}^n \Abs{z_i}
=
\Norm{\Bz}_1.
\end{dmath}

\begin{dmath}\label{eqn:convex-optimizationLecture2:304}
x_i^\conj
=
\left\{
\begin{array}{l l}
+1 & \quad \mbox{\( z_i \ge 0 \)} \\
-1 & \quad \mbox{\( z_i \le 0 \)}
\end{array}
\right.
\end{dmath}

\subsection{Calculus}

\subsubsection{Gradient}

Consider a scalar function

\begin{equation}\label{eqn:convex-optimizationLecture2:344}
F: \Rm{n} \rightarrow \bbR
\end{equation}

where for \( \Bx \in \Rm{n} \)

\begin{dmath}\label{eqn:convex-optimizationLecture2:364}
\spacegrad F(\Bx)
=
\begin{bmatrix}
\PD{x_1}{F(\Bx)} \\
\PD{x_2}{F(\Bx)} \\
\vdots \\
\PD{x_n}{F(\Bx)}
\end{bmatrix}
\end{dmath}

Example:

\begin{dmath}\label{eqn:convex-optimizationLecture2:384}
F(\Bx)
= \Bx^\T P \Bx
=
\begin{bmatrix}
x_1 & x_2 & \cdots & x_n
\end{bmatrix}
\begin{bmatrix}
P_{11} & P_{12} & \cdots & P_{1n} \\
P_{21} & P_{22} & \cdots & P_{2n} \\
\cdots
P_{n1} & P_{n2} & \cdots & P_{nn} \\
\end{bmatrix}
\begin{bmatrix}
x_1 \\ x_2 \\ \vdots \\ x_n
\end{bmatrix}
=
\begin{bmatrix}
x_1 & x_2 & \cdots & x_n
\end{bmatrix}
\lr{
x_1
\begin{bmatrix}
P_{11} \\
P_{21} \\
\vdots \\
P_{n1} \\
\end{bmatrix}
+
x_2
\begin{bmatrix}
P_{12} \\
P_{22} \\
\vdots \\
P_{n2} \\
\end{bmatrix}
+
\cdots
}
=
\sum x_j
\begin{bmatrix}
x_1 & x_2 & \cdots & x_n
\end{bmatrix}
\begin{bmatrix}
P_{1j} \\
P_{2j} \\
\vdots \\
P_{nj} \\
\end{bmatrix}
=
\sum_j x_j \sum_i x_i P_{ij}
=
\sum_{i,j = 1}^n x_i x_j P_{ij},
\end{dmath}

We want to show that
\begin{dmath}\label{eqn:convex-optimizationLecture2:404}
\spacegrad F(\Bx) = \lr{P + P^\T } \Bx.
\end{dmath}

Consider the k-th derivative

\begin{dmath}\label{eqn:convex-optimizationLecture2:424}
\PD{x_k}{} F(\Bx)
=
\PD{x_k}{}
\lr{
P_{kk} x_k^2
+
\sum_{i \ne k} x_i x_k \lr{ P_{ik} + P_{ki} }
}
=
2 P_{kk} x_k + 2
\sum_{i \ne k} x_i \frac{\lr{ P_{ik} + P_{ki} }}{2}
=
\sum_{i}^n x_i \frac{\lr{ P_{ik} + P_{ki} }}{2}
=
\sum_{i}^n \lr{ P_{ik} + P_{ki} } x_i
,
\end{dmath}

which proves \cref{eqn:convex-optimizationLecture2:404}.

\paragraph{Symmetric matrices}

Let \( S \) be the set of symmetric matrices

\begin{equation}\label{eqn:convex-optimizationLecture2:444}
S = \setlr{ P \in \Rm{n\times n} | P = P^\T},
\end{equation}

then
%In particular, for a symmetric matrix \( P \), this means

\begin{dmath}\label{eqn:convex-optimizationLecture2:464}
\spacegrad \Bx^\T P \Bx = 2 P \Bx.
\end{dmath}

\paragraph{Gradient}

The gradient provides a linear approximation of a function about a point \( \Bx_0 \in \Rm{n} \).

\begin{dmath}\label{eqn:convex-optimizationLecture2:484}
F(\Bx)
\approx F(\Bx_0) + \spacegrad F(\Bx_0)^\T \lr{ \Bx - \Bx_0 }.
=
F(\Bx_0) + \innerproduct{ \spacegrad F(\Bx_0)}{ \Bx - \Bx_0 },
\end{dmath}

or
\begin{equation}\label{eqn:convex-optimizationLecture2:504}
F(\Bx + \Delta \Bx)
=
F(\Bx) + \innerproduct{ \spacegrad F(\Bx)}{ \Delta \Bx }.
\end{equation}

This can be thought of as the definition of the gradient in an inner product space.  It will be possible to find the structure of the gradient by considering a pertubation of a function about a point.

\subsubsection{Chain rule}

Gradients for compositions of functions.

\paragraph{Example 1:}

\begin{dmath}\label{eqn:convex-optimizationLecture2:524}
\begin{aligned}
F &: \Rm{n} \rightarrow \bbR \\
g &: \bbR \rightarrow \bbR,
\end{aligned}
\end{dmath}

and let

\begin{dmath}\label{eqn:convex-optimizationLecture2:544}
h(\Bx) = g(F(\Bx)),
\end{dmath}

for \( \Bx \in \Rm{n} \), then

\begin{dmath}\label{eqn:convex-optimizationLecture2:564}
\spacegrad h(\Bx)
=
g'(F(\Bx)) \spacegrad F(\Bx).
\end{dmath}

\paragraph{Example 2:}

\begin{dmath}\label{eqn:convex-optimizationLecture2:584}
\begin{aligned}
F &: \Rm{n} \rightarrow \Rm{n} \\
g &: \Rm{n} \rightarrow \bbR,
\end{aligned}
\end{dmath}

and let

\begin{dmath}\label{eqn:convex-optimizationLecture2:604}
h(\Bx)
= g(F(\Bx))
= g\lr{
\begin{bmatrix}
F_1(\Bx) \\
F_2(\Bx) \\
\vdots \\
F_n(\Bx) \\
\end{bmatrix}
}
\end{dmath}

for \( \Bx \in \Rm{n} \), then

\begin{dmath}\label{eqn:convex-optimizationLecture2:624}
\PD{x_k}{h(\Bx)}
=
\PD{F_1}{g}
\PD{x_k}{F_1}
+
\PD{F_2}{g}
\PD{x_k}{F_2}
+
\cdots
\end{dmath}

Let

\begin{dmath}\label{eqn:convex-optimizationLecture2:644}
D F(\Bx)
=
\begin{bmatrix}
\PD{x_1}{F_1} & \PD{x_2}{F_1} & \PD{x_n}{F_1} \\
\PD{x_1}{F_2} & \PD{x_2}{F_2} & \PD{x_n}{F_2} \\
\vdots \\
\PD{x_1}{F_n} & \PD{x_2}{F_n} & \PD{x_n}{F_n} \\
\end{bmatrix}
\end{dmath}

so

\begin{dmath}\label{eqn:convex-optimizationLecture2:664}
\spacegrad h(\Bx)
=
\lr{ D F(\Bx) }^\T
\spacegrad g(F(\Bx)).
\end{dmath}

\paragraph{In general}

\begin{dmath}\label{eqn:convex-optimizationLecture2:684}
\begin{aligned}
F &: \Rm{n} \rightarrow \Rm{m} \\
g &: \Rm{m} \rightarrow \Rm{p},
\end{aligned}
\end{dmath}

With \( h(\Bx) = g(F(\Bx) \),

\begin{dmath}\label{eqn:convex-optimizationLecture2:704}
D h(\Bx) = D g(F(\Bx)) D F(\Bx),
\end{dmath}

and
\begin{dmath}\label{eqn:convex-optimizationLecture2:724}
\spacegrad h(\Bx) =
\lr{
D F(\Bx)
}^\T
\spacegrad g(F(\Bx))
\end{dmath}

Note: \( \spacegrad h(\Bx) = D h(\Bx)^\T \) if \( p = 1 \).

EXERSIZE: work this out for a general non-square case such as \( n = 4, m = 3, p = 2 \).

\paragraph{Affine functions}

An important example are affine functions of \( \Bx \)

\begin{dmath}\label{eqn:convex-optimizationLecture2:744}
\begin{aligned}
F &: \Rm{n} \rightarrow \Rm{n} \\
g &: \Rm{n} \rightarrow \bbR,
\end{aligned}
\end{dmath}

\begin{dmath}\label{eqn:convex-optimizationLecture2:764}
F(\Bx) = A \Bx + \Bb,
\end{dmath}

where \( A \) is an \( n \times n \) matrix and \( \Bb \) is an \( n \times 1 \) column vector.

Given a function

\begin{equation}\label{eqn:convex-optimizationLecture2:784}
h(\Bx) = g(F(\Bx)) = g(A \Bx + \Bb).
\end{equation}

\begin{dmath}\label{eqn:convex-optimizationLecture2:804}
F(\Bx)
= A \Bx + \Bb
=
\begin{bmatrix}
a_1^\T \\
a_2^\T \\
\vdots \\
a_n^\T \\
\end{bmatrix}
\Bx
+
\Bb
=
\begin{bmatrix}
\innerproduct{a_1}{\Bx} \\
\innerproduct{a_2}{\Bx} \\
\vdots \\
\innerproduct{a_n}{\Bx}
\end{bmatrix}
+
\Bb
=
\begin{bmatrix}
\sum_{i = 1}^n a_{1i} x_i \\
\sum_{i = 1}^n a_{2i} x_i \\
\vdots \\
\sum_{i = 1}^n a_{ni} x_i
\vdots \\
\end{bmatrix}
+
\Bb
=
\begin{bmatrix}
F_1(\Bx) \\
F_2(\Bx) \\
\vdots \\
F_n(\Bx) \\
\end{bmatrix}
+
\Bb,
\end{dmath}

so

\begin{dmath}\label{eqn:convex-optimizationLecture2:824}
D F(\Bx) = A.
\end{dmath}

\subsubsection{Second derivative.}

The second derivative for \( \Rm{n} \rightarrow \bbR \), the ``Hessian'' is

\begin{dmath}\label{eqn:convex-optimizationLecture2:844}
\spacegrad^2 F =
\begin{bmatrix}
\frac{\partial^2 F}{\partial x_i \partial x_j }
\end{bmatrix}
=
\begin{bmatrix}
\frac{\partial^2 F}{\partial x_1^2 } & \frac{\partial^2 F}{\partial x_1 \partial x_2} & \cdots & \frac{\partial^2 F}{\partial x_1 \partial x_n} \\
\frac{\partial^2 F}{\partial x_2 \partial x_1 } & \frac{\partial^2 F}{\partial x_2^2 } & \cdots & \frac{\partial^2 F}{\partial x_2 \partial x_n} \\
\vdots \\
\end{bmatrix}
\end{dmath}

\paragraph{Example}

Given

\begin{dmath}\label{eqn:convex-optimizationLecture2:864}
F(\Bx)
= \inv{2} \Bx^\T P \Bx
+ \Bq^\T \Bx
+ \Bc,
\end{dmath}

where \( P = P^\T \), then

\begin{dmath}\label{eqn:convex-optimizationLecture2:884}
\spacegrad F
=
\inv{2} \lr{ P + P^\T } \Bx + \Bq
=
P \Bx + \Bq,
\end{dmath}

and

\begin{dmath}\label{eqn:convex-optimizationLecture2:904}
\spacegrad^2 F  P.
\end{dmath}

\EndArticle
%\EndNoBibArticle
