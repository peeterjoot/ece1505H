%
% Copyright © 2017 Peeter Joot.  All Rights Reserved.
% Licenced as described in the file LICENSE under the root directory of this GIT repository.
%
\paragraph{Topics}

\begin{itemize}
\item Calculus: Derivatives and Jacobians, Gradients, Hessians, approximation functions.
\item Linear algebra, Matrices, decompositions, ...
\end{itemize}

\section{Norms}
%\paragraph{Vector space}

\makedefinition{Vector space}{dfn:convex-optimizationLecture2:1}{
A set of elements (vectors) that is closed under vector addition and scaling.
} % definition

This generalizes the directed arrow concept of vector space (
\cref{fig:L2XPlusY:L2XPlusYFig1})
that is familiar from geometry.

\imageFigure{../figures/ece1505-convex-optimization/L2XPlusYFig1}{Vector addition.}{fig:L2XPlusY:L2XPlusYFig1}{0.2}

\makedefinition{Normed vector spaces}{dfn:convex-optimizationLecture2:2}{
A vector space with a notion of length of any single vector, the ``norm''.
} % definition

\makedefinition{Inner product space.}{dfn:convex-optimizationLecture2:3}{
A normed vector space with a notion of a real angle between any pair of vectors.
} % definition

This course has a focus on optimization in \R{n}.  Complex spaces in the context of this course can be considered with a mapping \( \text{\C{n}} \rightarrow \Rm{2 n} \).

\makedefinition{Norm.}{dfn:convex-optimizationLecture2:4}{
A norm is a function operating on a vector

\begin{dmath*}
\Bx = ( x_1, x_2, \cdots, x_n )
\end{dmath*}

that provides a mapping

\begin{equation*}
\Norm{ \cdot } : \Rm{n} \rightarrow \bbR,
\end{equation*}

where

\begin{itemize}
\item \( \Norm{ \Bx } \ge 0 \)
\item \( \Norm{ \Bx } = 0 \qquad \iff \Bx = 0 \)
\item \( \Norm{ t \Bx } = \Abs{t} \Norm{ \Bx } \)
\item \( \Norm{ \Bx + \By } \le \Norm{ \Bx } + \Norm{\By} \).  This is the triangle inequality.
\end{itemize}
} % definition

\paragraph{Example: Euclidean norm}

\begin{dmath}\label{eqn:convex-optimizationLecture2:24}
\Norm{\Bx} = \sqrt{ \sum_{i = 1}^n x_i^2 }
\end{dmath}

\paragraph{Example: \(l_p\)-norms}

\begin{dmath}\label{eqn:convex-optimizationLecture2:44}
\Norm{\Bx}_p = \lr{ \sum_{i = 1}^n \Abs{x_i}^p }^{1/p}.
\end{dmath}

For \( p = 1 \), this is

\begin{dmath}\label{eqn:convex-optimizationLecture2:64}
\Norm{\Bx}_1 = \sum_{i = 1}^n \Abs{x_i},
\end{dmath}

For \( p = 2 \), this is the Euclidean norm \cref{eqn:convex-optimizationLecture2:24}.
For \( p = \infty \), this is

\begin{dmath}\label{eqn:convex-optimizationLecture2:324}
\Norm{\Bx}_\infty = \max_{i = 1}^n \Abs{x_i}.
\end{dmath}

\makedefinition{Unit ball}{dfn:convex-optimizationLecture2:10}{

\begin{dmath*}
\setlr{ \Bx | \Norm{\Bx} \le 1 }
\end{dmath*}

} % definition

The regions of the unit ball under the \( l_1, l_2, and l_\infty \) norms are plotted in \cref{fig:L2Norms:L2NormsFig2}.

\imageFigure{../figures/ece1505-convex-optimization/L2NormsFig2}{Some unit ball regions.}{fig:L2Norms:L2NormsFig2}{0.3}

The \( l_2 \) norm is not only familiar, but can be ``induced'' by an inner product

\begin{equation}\label{eqn:convex-optimizationLecture2:84}
\innerprod{\Bx}{\By} = \Bx^\T \By = \sum_{i = 1}^n x_i y_i,
\end{equation}

which is not true for all norms.  The norm induced by this inner product is

\begin{dmath}\label{eqn:convex-optimizationLecture2:104}
\Norm{\Bx}_2 = \sqrt{ \innerprod{\Bx}{\By} }
\end{dmath}

Inner product spaces have a notion of angle
(\cref{fig:L2XYTheta:L2XYThetaFig3})
given by

\begin{dmath}\label{eqn:convex-optimizationLecture2:124}
\innerprod{\Bx}{\By} = \Norm{\Bx} \Norm{\By} \cos \theta,
\end{dmath}

\imageFigure{../figures/ece1505-convex-optimization/L2XYThetaFig3}{Inner product induced angle.}{fig:L2XYTheta:L2XYThetaFig3}{0.2}
and always satisfy the Cauchy-Schwartz inequality

\begin{dmath}\label{eqn:convex-optimizationLecture2:144}
\innerprod{\Bx}{\By} \le \Norm{\Bx}_2 \Norm{\By}_2.
\end{dmath}

In an inner product space we say \( \Bx \) and \( \By \) are orthogonal vectors \( \Bx \perp \By \) if
\( \innerprod{\Bx}{\By} = 0 \), as sketched in
\cref{fig:L2XYRightAngle:L2XYRightAngleFig4}.

\imageFigure{../figures/ece1505-convex-optimization/L2XYRightAngleFig4}{Orthogonality.}{fig:L2XYRightAngle:L2XYRightAngleFig4}{0.2}

\section{Dual norm}

\makedefinition{Dual norm}{dfn:convex-optimizationLecture2:20}{
Let \( \Norm{ \cdot } \) be a norm in \R{n}.  The ``dual'' norm \( \Norm{ \cdot }_\conj \) is defined as
\begin{equation*}
\Norm{\Bz}_\conj = \sup_\Bx \setlr{ \Bz^\T \Bx | \Norm{\Bx} \le 1 }.
\end{equation*}

where \( \sup \) is roughly the ``least upper bound''.
\index{sup}

This is a limit over the unit ball of \( \Norm{\cdot} \).
} % definition

\paragraph{\( l_2 \) dual}.

Dual of the \( l_2 \) is the \( l_2 \) norm.

%The calculation procedure is sketched in \cref{fig:unitBallL2:unitBallL2Fig5}.
\imageFigure{../figures/ece1505-convex-optimization/unitBallL2Fig5}{\(l_2\) dual norm determination.}{fig:unitBallL2:unitBallL2Fig5}{0.2}

Proof:

\begin{dmath}\label{eqn:convex-optimizationLecture2:164}
\begin{aligned}
\Norm{\Bz}_\conj
&= \sup_\Bx \setlr{ \Bz^\T \Bx | \Norm{\Bx}_2 \le 1 } \\
&= \sup_\Bx \setlr{ \Norm{\Bz}_2 \Norm{\Bx}_2 \cos\theta | \Norm{\Bx}_2 \le 1 } \\
&\le \sup_\Bx \setlr{ \Norm{\Bz}_2 \Norm{\Bx}_2 | \Norm{\Bx}_2 \le 1 } \\
&\le
\cancel{\Norm{\Bz}_2}
\Norm{
\frac{\Bz}{ \cancel{\Norm{\Bz}_2} }
}_2 \\
&=
\Norm{\Bz}_2.
\end{aligned}
\end{dmath}

\paragraph{\( l_1 \) dual}.
For \( l_1 \), the dual is the \( l_\infty \) norm.  Proof:

\begin{equation}\label{eqn:convex-optimizationLecture2:184}
\Norm{\Bz}_\conj
=
\sup_\Bx \setlr{ \Bz^\T \Bx | \Norm{\Bx}_1 \le 1 },
\end{equation}

but
\begin{dmath}\label{eqn:convex-optimizationLecture2:204}
\Bz^\T \Bx
=
\sum_{i=1}^n z_i x_i \le
\Abs{
\sum_{i=1}^n z_i x_i
}
\le
\sum_{i=1}^n \Abs{z_i x_i },
\end{dmath}

so
\begin{dmath}\label{eqn:convex-optimizationLecture2:224}
\Norm{\Bz}_\conj
=
\sum_{i=1}^n \Abs{z_i}\Abs{ x_i }
\le \lr{ \max_{j=1}^n \Abs{z_j} }
\sum_{i=1}^n \Abs{ x_i }
\le \lr{ \max_{j=1}^n \Abs{z_j} }
=
\Norm{\Bz}_\infty.
\end{dmath}

%\cref{fig:unitBallL1:unitBallL1Fig6}.
\imageFigure{../figures/ece1505-convex-optimization/unitBallL1Fig6}{\(l_1\) dual norm determination.}{fig:unitBallL1:unitBallL1Fig6}{0.2}

\paragraph{\( l_\infty \) dual}.

%\cref{fig:unitBallLinf:unitBallLinfFig7}.
\imageFigure{../figures/ece1505-convex-optimization/unitBallLinfFig7}{\(l_\infty\) dual norm determination.}{fig:unitBallLinf:unitBallLinfFig7}{0.2}

\begin{equation}\label{eqn:convex-optimizationLecture2:244}
\Norm{\Bz}_\conj
=
\sup_\Bx \setlr{ \Bz^\T \Bx | \Norm{\Bx}_\infty \le 1 }.
\end{equation}

Here
\begin{dmath}\label{eqn:convex-optimizationLecture2:264}
\Bz^\T \Bx
=
\sum_{i=1}^n z_i x_i
\le
\sum_{i=1}^n \Abs{z_i}\Abs{ x_i }
\le
\lr{ \max_j \Abs{ x_j } }
\sum_{i=1}^n \Abs{z_i}
=
\Norm{\Bx}_\infty
\sum_{i=1}^n \Abs{z_i}.
\end{dmath}

So
\begin{equation}\label{eqn:convex-optimizationLecture2:284}
\Norm{\Bz}_\conj
\le
\sum_{i=1}^n \Abs{z_i}
=
\Norm{\Bz}_1.
\end{equation}

Statement from the lecture: I'm not sure where this fits:

\begin{dmath}\label{eqn:convex-optimizationLecture2:304}
x_i^\conj
=
\left\{
\begin{array}{l l}
+1 & \quad \mbox{\( z_i \ge 0 \)} \\
-1 & \quad \mbox{\( z_i \le 0 \)}
\end{array}
\right.
\end{dmath}

%
% Copyright © 2017 Peeter Joot.  All Rights Reserved.
% Licenced as described in the file LICENSE under the root directory of this GIT repository.
%
\section{Multivariable Taylor approximation}

The Taylor series expansion for a scalar function \( g : \bbR \rightarrow \bbR \) about the origin is just

\begin{dmath}\label{eqn:jacobianAndHessian:20}
g(t) = g(0) + t g'(0) + \frac{t^2}{2} g''(0) + \cdots
\end{dmath}

In particular

\begin{dmath}\label{eqn:jacobianAndHessian:40}
g(1) = g(0) + g'(0) + \frac{1}{2} g''(0) + \cdots
\end{dmath}

Now consider \( g(t) = f( \Bx + \Ba t ) \), where \( f : \bbR^n \rightarrow \bbR \), \( g(0) = f(\Bx) \), and \( g(1) = f(\Bx + \Ba) \).
This trick, from
\citep{hestenes1999nfc}
allows for a direct expansion of the multivariable Taylor series of a scalar function

\begin{dmath}\label{eqn:jacobianAndHessian:60}
f( \Bx + \Ba)
= f(\Bx)
+ \evalbar{\frac{df(\Bx + \Ba t)}{dt}}{t = 0} + \frac{1}{2} \evalbar{\frac{d^2f(\Bx + \Ba t)}{dt^2}}{t = 0} + \cdots
\end{dmath}

The first order term is

\begin{dmath}\label{eqn:jacobianAndHessian:80}
\evalbar{\frac{df(\Bx + \Ba t)}{dt}}{t = 0}
=
\sum_{i = 1}^n
\frac{d( x_i + a_i t)}{dt}
\evalbar{\PD{(x_i + a_i t)}{f(\Bx + \Ba t)}}{t = 0}
=
\sum_{i = 1}^n
a_i
\PD{x_i}{f(\Bx)}
= \Ba \cdot \spacegrad f.
\end{dmath}

Similarly, for the second order term

\begin{dmath}\label{eqn:jacobianAndHessian:100}
\evalbar{\frac{d^2 f(\Bx + \Ba t)}{dt^2}}{t = 0}
=
\evalbar{\lr{
   \frac{d}{dt}
   \lr{
   \sum_{i = 1}^n
   a_i
   \PD{(x_i + a_i t)}{f(\Bx + \Ba t)}
   }
   }
}{t = 0}
=
\evalbar{
   \lr{
      \sum_{j = 1}^n
      \frac{d(x_j + a_j t)}{dt}
         \sum_{i = 1}^n
         a_i
      \frac{\partial^2 f(\Bx + \Ba t)}{\partial (x_j + a_j t) \partial (x_i + a_i t) }
   }
}{t = 0}
=
\sum_{i,j = 1}^n a_i a_j \frac{\partial^2 f}{\partial x_i \partial x_j}
=
(\Ba \cdot \spacegrad)^2 f.
\end{dmath}

The complete Taylor expansion of a scalar function \( f : \bbR^n \rightarrow \bbR \) is therefore

\begin{dmath}\label{eqn:jacobianAndHessian:120}
f(\Bx + \Ba)
= f(\Bx) +
\Ba \cdot \spacegrad f +
\inv{2} \lr{ \Ba \cdot \spacegrad}^2 f + \cdots,
\end{dmath}

so the Taylor expansion has an exponential structure

\begin{equation}\label{eqn:jacobianAndHessian:140}
f(\Bx + \Ba) = \sum_{k = 0}^\infty \inv{k!} \lr{ \Ba \cdot \spacegrad}^k f = e^{\Ba \cdot \spacegrad} f.
\end{equation}

Should an approximation of a vector valued function \( \Bf : \bbR^n \rightarrow \bbR^m \) be desired it is only required to form a matrix of the components

\begin{equation}\label{eqn:jacobianAndHessian:160}
\Bf(\Bx + \Ba)
= \Bf(\Bx) +
[\Ba \cdot \spacegrad f_i]_i +
\inv{2} [\lr{ \Ba \cdot \spacegrad}^2 f_i]_i + \cdots,
\end{equation}

where \( [.]_i \) denotes a column vector over the rows \( i \in [1,m] \), and \( f_i \) are the coordinates of \( \Bf \).

\section{The Jacobian matrix}

In \citep{boyd2004convex} the Jacobian \( D \Bf \) of a function \( \Bf : \bbR^n \rightarrow \bbR^m \) is defined in terms of the limit of the \( l_2 \) norm ratio

\begin{dmath}\label{eqn:jacobianAndHessian:180}
\frac{\Norm{\Bf(\Bz) - \Bf(\Bx) - (D \Bf) (\Bz - \Bx)}_2 }{ \Norm{\Bz - \Bx}_2 },
\end{dmath}

with the statement that the function \( \Bf \) has a derivative if this limit exists.  Here the Jacobian \( D \Bf \in \bbR^{m \times n} \) must be matrix valued.

Let \( \Bz = \Bx + \Ba \), so the first order expansion of \cref{eqn:jacobianAndHessian:160} is

\begin{dmath}\label{eqn:jacobianAndHessian:200}
\Bf(\Bz)
= \Bf(\Bx) + [\lr{ \Bz - \Bx } \cdot \spacegrad f_i]_i
.
%+ \inv{2} [(\lr{ \Bz - \Bx } \cdot \spacegrad)^2 f_i]_i.
\end{dmath}

With the (unproven) assumption that this Taylor expansion satisfies the norm limit criteria of \cref{eqn:jacobianAndHessian:180}, it is possible to extract the structure of the Jacobian by comparison

\begin{dmath}\label{eqn:jacobianAndHessian:220}
(D \Bf)
(\Bz - \Bx)
=
{\begin{bmatrix}
\lr{ \Bz - \Bx } \cdot \spacegrad f_i
\end{bmatrix}}_i
=
{\begin{bmatrix}
\sum_{j = 1}^n (z_j - x_j) \PD{x_j}{f_i}
\end{bmatrix}}_i
=
{\begin{bmatrix}
\PD{x_j}{f_i}
\end{bmatrix}}_{ij}
(\Bz - \Bx),
\end{dmath}

so
%\begin{dmath}\label{eqn:jacobianAndHessian:240}
\boxedEquation{eqn:jacobianAndHessian:240}{
(D \Bf)_{ij} = \PD{x_j}{f_i}
}
%\end{dmath}

Written out explicitly as a matrix the Jacobian is

\begin{equation}\label{eqn:jacobianAndHessian:320}
D \Bf
=
\begin{bmatrix}
\PD{x_1}{f_1} & \PD{x_2}{f_1} & \cdots & \PD{x_n}{f_1} \\
\PD{x_1}{f_2} & \PD{x_2}{f_2} & \cdots & \PD{x_n}{f_2} \\
\vdots & \vdots & & \vdots \\
\PD{x_1}{f_m} & \PD{x_2}{f_m} & \cdots & \PD{x_n}{f_m} \\
\end{bmatrix}
=
\begin{bmatrix}
(\spacegrad f_1)^\T \\
(\spacegrad f_2)^\T \\
\vdots \\
(\spacegrad f_m)^\T
\end{bmatrix}.
\end{equation}

In particular, when the function is scalar valued
\begin{dmath}\label{eqn:jacobianAndHessian:360}
D f = (\spacegrad f)^\T.
\end{dmath}

With this notation, the first Taylor expansion, in terms of the Jacobian matrix is

%\begin{dmath}\label{eqn:jacobianAndHessian:260}
\boxedEquation{eqn:jacobianAndHessian:260}{
\Bf(\Bz)
\approx \Bf(\Bx) + (D \Bf) \lr{ \Bz - \Bx }.
}
%\end{dmath}

\paragraph{Gradient}

The gradient provides a linear approximation of a function about a point \( \Bx_0 \in \Rm{n} \).

\begin{dmath}\label{eqn:jacobianAndHessian:484}
F(\Bx)
\approx F(\Bx_0) + \spacegrad F(\Bx_0)^\T \lr{ \Bx - \Bx_0 }.
=
F(\Bx_0) + \innerproduct{ \spacegrad F(\Bx_0)}{ \Bx - \Bx_0 },
\end{dmath}

or
\begin{equation}\label{eqn:jacobianAndHessian:504}
F(\Bx + \Delta \Bx)
=
F(\Bx) + \innerproduct{ \spacegrad F(\Bx)}{ \Delta \Bx }.
\end{equation}

This can be thought of as the definition of the gradient in an inner product space.  It will be possible to find the structure of the gradient by considering a perturbation of a function about a point.

When \( g \) is a scalar function, the chain rule can be expressed in terms of the gradient

\begin{dmath}\label{eqn:jacobianAndHessian:724}
\spacegrad
(g(F(\Bx)))
%h(\Bx)
=
\evalbar{\lr{
D F
}^\T}{\Bx}
\evalbar{\spacegrad g}{F(\Bx)}.
\end{dmath}

%Note: \( \spacegrad h(\Bx) = D h(\Bx)^\T \) if \( p = 1 \).
%Also,
%if \( p = 1 \),
%\( \spacegrad g(F(\Bx)) = (D g(F(\Bx)))^\T \).

\paragraph{Example 1:}

\begin{dmath}\label{eqn:jacobianAndHessian:524}
\begin{aligned}
F &: \Rm{n} \rightarrow \bbR \\
g &: \bbR \rightarrow \bbR,
\end{aligned}
\end{dmath}

and let

\begin{dmath}\label{eqn:jacobianAndHessian:544}
h(\Bx) = g(F(\Bx)),
\end{dmath}

for \( \Bx \in \Rm{n} \), then

\begin{dmath}\label{eqn:jacobianAndHessian:564}
\spacegrad h(\Bx)
=
g'(F(\Bx)) \spacegrad F(\Bx).
\end{dmath}

%
% Copyright © 2017 Peeter Joot.  All Rights Reserved.
% Licenced as described in the file LICENSE under the root directory of this GIT repository.
%
\makeexample{Quadratic form}{example:quadraticForm:1}{

\begin{equation}\label{eqn:convex-optimizationLecture2:384}
F(\Bx)
= \Bx^\T P \Bx
%=
%\begin{bmatrix}
%x_1 & x_2 & \cdots & x_n
%\end{bmatrix}
%\begin{bmatrix}
%P_{11} & P_{12} & \cdots & P_{1n} \\
%P_{21} & P_{22} & \cdots & P_{2n} \\
%\vdots & & & \\
%P_{n1} & P_{n2} & \cdots & P_{nn} \\
%\end{bmatrix}
%\begin{bmatrix}
%x_1 \\ x_2 \\ \vdots \\ x_n
%\end{bmatrix}
%=
%\begin{bmatrix}
%x_1 & x_2 & \cdots & x_n
%\end{bmatrix}
%\lr{
%x_1
%\begin{bmatrix}
%P_{11} \\
%P_{21} \\
%\vdots \\
%P_{n1} \\
%\end{bmatrix}
%+
%x_2
%\begin{bmatrix}
%P_{12} \\
%P_{22} \\
%\vdots \\
%P_{n2} \\
%\end{bmatrix}
%+
%\cdots
%}
%=
%\sum x_j
%\begin{bmatrix}
%x_1 & x_2 & \cdots & x_n
%\end{bmatrix}
%\begin{bmatrix}
%P_{1j} \\
%P_{2j} \\
%\vdots \\
%P_{nj} \\
%\end{bmatrix}
%=
%\sum_j x_j \sum_i x_i P_{ij}
=
\sum_{i,j = 1}^n x_i x_j P_{ij},
\end{equation}

We want to show that
\begin{dmath}\label{eqn:convex-optimizationLecture2:404}
\spacegrad F(\Bx) = \lr{P + P^\T } \Bx.
\end{dmath}

Consider the k-th derivative

\begin{dmath}\label{eqn:convex-optimizationLecture2:424}
\PD{x_k}{} F(\Bx)
=
\PD{x_k}{}
\lr{
P_{kk} x_k^2
+
\sum_{i \ne k} x_i x_k \lr{ P_{ik} + P_{ki} }
}
=
2 P_{kk} x_k + 2
\sum_{i \ne k} x_i \frac{\lr{ P_{ik} + P_{ki} }}{2}
=
\sum_{i}^n x_i \frac{\lr{ P_{ik} + P_{ki} }}{2}
=
\sum_{i}^n \lr{ P_{ik} + P_{ki} } x_i
,
\end{dmath}

which proves \cref{eqn:convex-optimizationLecture2:404}.
} % example


\paragraph{Symmetric matrices}

Let \( S^n \) be the set of symmetric matrices

\begin{equation}\label{eqn:jacobianAndHessian:444}
S^n = \setlr{ P \in \Rm{n\times n} | P = P^\T},
\end{equation}

then
%In particular, for a symmetric matrix \( P \), this means

\begin{dmath}\label{eqn:jacobianAndHessian:464}
\spacegrad (\Bx^\T P \Bx) = 2 P \Bx.
\end{dmath}

\section{Chain rule}

The gradients or Jacobians for compositions of functions can also be calculated

\maketheorem{Chain rule}{thm:jacobianAndHessian:1}{

Given functions

\begin{dmath}\label{eqn:jacobianAndHessian:684}
\begin{aligned}
F &: \Rm{n} \rightarrow \Rm{m} \\
g &: \Rm{m} \rightarrow \Rm{p},
\end{aligned}
\end{dmath}

%With \( h(\Bx) = g(F(\Bx) \),

\begin{dmath}\label{eqn:jacobianAndHessian:704}
D (g(F(\Bx)))
%h(\Bx)
= \evalbar{D g}{F(\Bx)} \evalbar{D F}{\Bx}.
\end{dmath}
} % theorem

\paragraph{Scalar valued composition}

To illustrate this, first consider a scalar valued composition

\begin{dmath}\label{eqn:jacobianAndHessian:584}
\begin{aligned}
F &: \Rm{n} \rightarrow \Rm{n} \\
g &: \Rm{n} \rightarrow \bbR,
\end{aligned}
\end{dmath}

and let

\begin{dmath}\label{eqn:jacobianAndHessian:604}
h(\Bx)
= g(F(\Bx))
= g\lr{
\begin{bmatrix}
F_1(\Bx) \\
F_2(\Bx) \\
\vdots \\
F_n(\Bx) \\
\end{bmatrix}
}
\end{dmath}

for \( \Bx \in \Rm{n} \), then

\begin{dmath}\label{eqn:jacobianAndHessian:624}
\PD{x_k}{h(\Bx)}
=
\PD{F_1}{g}
\PD{x_k}{F_1}
+
\PD{F_2}{g}
\PD{x_k}{F_2}
+
\cdots
\end{dmath}

With

\begin{dmath}\label{eqn:jacobianAndHessian:644}
D F(\Bx)
=
\begin{bmatrix}
\PD{x_1}{F_1} & \PD{x_2}{F_1} & \cdots & \PD{x_n}{F_1} \\
\PD{x_1}{F_2} & \PD{x_2}{F_2} & \cdots & \PD{x_n}{F_2} \\
\vdots & & \vdots & \\
\PD{x_1}{F_n} & \PD{x_2}{F_n} & \cdots & \PD{x_n}{F_n} \\
\end{bmatrix}
\end{dmath}

the gradient \( \spacegrad g = (D g)^\T \) is

\begin{dmath}\label{eqn:jacobianAndHessian:664}
\spacegrad h(\Bx)
=
\evalbar{ (D F)^\T }{\Bx}
\spacegrad g(F(\Bx)),
\end{dmath}

or

\begin{dmath}\label{eqn:jacobianAndHessian:665}
D (g(F(\Bx)))
=
\evalbar{D g}{F(\Bx)}
\evalbar{D F}{\Bx}.
\end{dmath}

\paragraph{Affine functions}

An important example are affine functions of \( \Bx \)

\begin{dmath}\label{eqn:jacobianAndHessian:744}
\begin{aligned}
F &: \Rm{n} \rightarrow \Rm{n} \\
g &: \Rm{n} \rightarrow \bbR,
\end{aligned}
\end{dmath}

\begin{dmath}\label{eqn:jacobianAndHessian:764}
F(\Bx) = A \Bx + \Bb,
\end{dmath}

where \( A \) is an \( n \times n \) matrix and \( \Bb \) is an \( n \times 1 \) column vector.

Given a function

\begin{equation}\label{eqn:jacobianAndHessian:784}
h(\Bx) = g(F(\Bx)) = g(A \Bx + \Bb).
\end{equation}

\begin{dmath}\label{eqn:jacobianAndHessian:804}
F(\Bx)
= A \Bx + \Bb
%=
%\begin{bmatrix}
%a_1^\T \\
%a_2^\T \\
%\vdots \\
%a_n^\T \\
%\end{bmatrix}
%\Bx
%+
%\Bb
%=
%\begin{bmatrix}
%\innerproduct{a_1}{\Bx} \\
%\innerproduct{a_2}{\Bx} \\
%\vdots \\
%\innerproduct{a_n}{\Bx}
%\end{bmatrix}
%+
%\Bb
=
\begin{bmatrix}
\sum_{i = 1}^n a_{1i} x_i \\
\sum_{i = 1}^n a_{2i} x_i \\
\vdots \\
\sum_{i = 1}^n a_{ni} x_i
\vdots \\
\end{bmatrix}
+
\Bb
%=
%\begin{bmatrix}
%F_1(\Bx) \\
%F_2(\Bx) \\
%\vdots \\
%F_n(\Bx) \\
%\end{bmatrix}
%+
%\Bb
,
\end{dmath}

so

\begin{dmath}\label{eqn:jacobianAndHessian:824}
D F(\Bx) = A,
\end{dmath}

and
\begin{dmath}\label{eqn:jacobianAndHessian:1064}
\spacegrad (g(F(\Bx)))
=
(A \Bx)^\T
\evalbar{\spacegrad g}{F(\Bx)}.
\end{dmath}

\paragraph{General case}

The proof of the general case
can be essentially be performed by example, provided that example is sufficiently non-trivial, such as
a non-square case such as \( n = 4, m = 3, p = 2 \)

\begin{dmath}\label{eqn:jacobianAndHessian:924}
F(\Bx) =
\begin{bmatrix}
F_1(\Bx) \\
F_2(\Bx) \\
F_3(\Bx) \\
\end{bmatrix},
\end{dmath}

and
\begin{dmath}\label{eqn:jacobianAndHessian:944}
g(\By)
=
\begin{bmatrix}
g_1(\By) \\
g_2(\By) \\
\end{bmatrix}.
\end{dmath}

For such a function

\begin{dmath}\label{eqn:jacobianAndHessian:964}
\PD{x_1}{g(F(\Bx))}
=
\begin{bmatrix}
\PDi{x_1}{g_1(F(\Bx))} \\
\PDi{x_1}{g_2(F(\Bx))}
\end{bmatrix},
\end{dmath}

so

\begin{dmath}\label{eqn:jacobianAndHessian:984}
D g(F(\Bx))
=
\begin{bmatrix}
\PDi{x_1}{g_1(F(\Bx))} & \PDi{x_2}{g_1(F(\Bx))} & \cdots & \PDi{x_4}{g_1(F(\Bx))} \\
\PDi{x_1}{g_2(F(\Bx))} & \PDi{x_2}{g_2(F(\Bx))} & \cdots & \PDi{x_4}{g_2(F(\Bx))} \\
\end{bmatrix}
=
\begin{bmatrix}
D (g_1(F(\Bx))) \\
D (g_2(F(\Bx))) \\
\end{bmatrix}.
\end{dmath}

This reduces the problem to the composition of a scalar and vector function, such as

\begin{dmath}\label{eqn:jacobianAndHessian:1004}
D (g_1(F(\Bx)))
=
\sum_{i=1}^3 \sum_{j= 1}^4 \evalbar{\PD{y_i}{g_1}}{y_i = F_i(\Bx)} \PD{x_j}{F_i}
=
\evalbar{
   \lr{
      \begin{bmatrix}
      \PD{y_1}{g_1} & \PD{y_2}{g_1} & \PD{y_3}{g_1}
      \end{bmatrix}
   }
}{\By = F(\Bx)}
{\begin{bmatrix}
\PD{x_j}{F_i}
\end{bmatrix}}_{ij}
=
\evalbar{D g_1}{F(\Bx)}
D F(\Bx).
\end{dmath}

The total Jacobian is
\begin{dmath}\label{eqn:jacobianAndHessian:1024}
D g(F(\Bx))
=
\begin{bmatrix}
\evalbar{D g_1}{F(\Bx)} D F(\Bx) \\
\evalbar{D g_2}{F(\Bx)} D F(\Bx)
\end{bmatrix},
\end{dmath}

which can be factored as
\begin{dmath}\label{eqn:jacobianAndHessian:1044}
D (g(F(\Bx)))
=
\evalbar{D g}{F(\Bx)} D F(\Bx).
\end{dmath}.

\section{The Hessian matrix}

For scalar valued functions, the text expresses the second order expansion of a function in terms of the Jacobian and Hessian matrices

\begin{dmath}\label{eqn:jacobianAndHessian:380}
f(\Bz)
\approx f(\Bx) + (D f) \lr{ \Bz - \Bx }
+ \inv{2} \lr{ \Bz - \Bx }^\T (\spacegrad^2 f) \lr{ \Bz - \Bx }.
\end{dmath}

Because \( \spacegrad^2 \) is the usual notation for a Laplacian operator, this \( \spacegrad^2 f \in \bbR^{n \times n}\) notation for the Hessian matrix is not ideal in my opinion.  Ignoring that notational objection for this class, the structure of the Hessian matrix can be extracted by comparison with the coordinate expansion

\begin{dmath}\label{eqn:jacobianAndHessian:300}
\Ba^\T (\spacegrad^2 f) \Ba
=
\sum_{r,s = 1}^n a_r a_s \frac{\partial^2 f}{\partial x_r \partial x_s}
\end{dmath}

so
%\begin{dmath}\label{eqn:jacobianAndHessian:280}
\boxedEquation{eqn:jacobianAndHessian:280}{
(\spacegrad^2 f)_{ij}
=
\frac{\partial^2 f_i}{\partial x_i \partial x_j}.
}
%\end{dmath}

In explicit matrix form the Hessian is

\begin{dmath}\label{eqn:jacobianAndHessian:340}
\spacegrad^2 f
=
\begin{bmatrix}
\frac{\partial^2 f}{\partial x_1 \partial x_1} & \frac{\partial^2 f}{\partial x_1 \partial x_2} & \cdots &\frac{\partial^2 f}{\partial x_1 \partial x_n} \\
\frac{\partial^2 f}{\partial x_2 \partial x_1} & \frac{\partial^2 f}{\partial x_2 \partial x_2} & \cdots &\frac{\partial^2 f}{\partial x_2 \partial x_n} \\
\vdots & \vdots & & \vdots \\
\frac{\partial^2 f}{\partial x_n \partial x_1} & \frac{\partial^2 f}{\partial x_n \partial x_2} & \cdots &\frac{\partial^2 f}{\partial x_n \partial x_n}
\end{bmatrix}.
\end{dmath}

Is there a similar nice matrix structure for the Hessian of a function \( f : \bbR^n \rightarrow \bbR^m \)?

\makeexample{Second order scalar function}{example:jacobianAndHessian:2}{
Given

\begin{dmath}\label{eqn:jacobianAndHessian:864}
F(\Bx)
= \inv{2} \Bx^\T P \Bx
+ \Bq^\T \Bx
+ \Bc,
\end{dmath}

where \( P \) is a symmetric matrix \( P = P^\T \), then

\begin{dmath}\label{eqn:jacobianAndHessian:884}
\spacegrad F
=
\inv{2} \lr{ P + P^\T } \Bx + \Bq
=
P \Bx + \Bq,
\end{dmath}

and

\begin{dmath}\label{eqn:jacobianAndHessian:904}
\spacegrad^2 F
=
P.
\end{dmath}
} % example

