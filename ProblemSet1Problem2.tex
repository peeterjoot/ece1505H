%
% Copyright � 2017 Peeter Joot.  All Rights Reserved.
% Licenced as described in the file LICENSE under the root directory of this GIT repository.
%
\makeproblem{Inversion formula for ``small'' matrices}{convex-optimization:problemSet1:2}{
Prove the relation

\begin{dmath}\label{eqn:ProblemSet1Problem2:20}
(I + A)^{-1} = I - A,
\end{dmath}

for \(A\) ``small''.  We used this in class to derive the second order expansion of

\begin{dmath}\label{eqn:ProblemSet1Problem2:40}
\log \det(I + A).
\end{dmath}

Prove this result in two ways:
\makesubproblem{}{convex-optimization:problemSet1:2a}
First, prove this for the special case of \( A \in S^n_{++} \) where the eigenvalues are small.
This is what we needed in class. Use a decomposition of \( A \) and Taylor approximation of the eigenvalues.

\makesubproblem{}{convex-optimization:problemSet1:2b}
Next prove the general relation: If \( A \in \Rm{n \times n} \) and \( \Norm{A}_p < 1 \) then \( I - A \) is non-singular, and

\begin{dmath}\label{eqn:ProblemSet1Problem2:60}
\lr{ I - A }^{-1} = \sum_{k=0}^\infty A^k
\end{dmath}

where

\begin{dmath}\label{eqn:ProblemSet1Problem2:80}
\Norm{(I - A)^{-1}}_p \le \inv{1 - \Norm{A}_p}.
\end{dmath}

The p-th matrix norm \( \Norm{A}_p \) is defined in terms of the vector p-norm as

\begin{dmath}\label{eqn:ProblemSet1Problem2:100}
\Norm{A}_p = \sup_{\Bx \ne 0} \frac{\Norm{A \Bx }_p}{\Norm{\Bx}_p}
\end{dmath}

which, using the scaling property of a norm, can be seen to be equivalent to

\begin{dmath}\label{eqn:ProblemSet1Problem2:120}
\Norm{A}_p = \max_{\Norm{\Bx}_p = 1} \Norm{A \Bx}_p.
\end{dmath}

In our derivation in class we used only the zeroth and first-order terms of the expansion.

Some hints that outline one approach to the above result:

\begin{enumerate}[(i)]
\item
One approach to proving the first statement (about non-singularity) is by contradiction: note
that if \( I - A \) is singular then there exists a vector \( \Bv \) such that \((I - A) \Bv = 0 \) and work from
there.
\item Next, consider the telescoping sum

\begin{dmath}\label{eqn:ProblemSet1Problem2:140}
\sum_{k=0}^N A^k (I - A) = I - A^{N+1},
\end{dmath}

and show
\begin{dmath}\label{eqn:ProblemSet1Problem2:160}
\lim_{k \rightarrow \infty } A^k = 0.
\end{dmath}

\item
To show that \( \lim_{k \rightarrow \infty } A^k = 0\), it is helpful first to prove that

\begin{dmath}\label{eqn:ProblemSet1Problem2:180}
\Norm{ A^{k+1} }_p \le \Norm{ A }_p \norm{ A^k }_p.
\end{dmath}

\item
Finally, combine your above results and the properties of a norm to show the desired result.
\end{enumerate}
} % makeproblem

\makeanswer{convex-optimization:problemSet1:2}{
\makeSubAnswer{}{convex-optimization:problemSet1:2a}

A matrix \( A \in S^n_{++} \) admits a representation

\begin{dmath}\label{eqn:ProblemSet1Problem2:200}
A = Q \Lambda Q^\T,
\end{dmath}

so

\begin{dmath}\label{eqn:ProblemSet1Problem2:220}
\lr{I + A}^{-1}
=
\lr{ I + Q \Lambda Q^\T }^{-1}
=
\lr{ Q Q^\T + Q \Lambda Q^\T }^{-1}
=
\lr{ Q ( I + \Lambda ) Q^\T }^{-1}
=
Q ( I + \Lambda)^{-1} Q^\T.
\end{dmath}

This reduces the problem to the computation of \( ( I + \Lambda)^{-1} \), where \( \Lambda \) is a diagonal matrix.  Such a matrix inverts trivially, provided all the eigenvalues are sufficiently small \( \lambda_i \in [0, 1) \)

\begin{dmath}\label{eqn:ProblemSet1Problem2:240}
\inv{ I + \Lambda }
=
{\begin{bmatrix}
\frac{\delta_{ij}}{ 1 + \lambda_i }
\end{bmatrix}}_{ij}
\approx
[ \delta_{ij}( 1 - \lambda_i ) ]_{ij}
=
1 - \Lambda.
\end{dmath}

This gives

\begin{dmath}\label{eqn:ProblemSet1Problem2:260}
\lr{I + A}^{-1}
=
Q ( I + \Lambda)^{-1} Q^\T
\approx
Q ( I - \Lambda) Q^\T
=
I - A. \qedmarker
\end{dmath}

\makeSubAnswer{}{convex-optimization:problemSet1:2b}

Removing the requirement that the matrix be symmetric, let

\begin{dmath}\label{eqn:ProblemSet1Problem2:280}
S_N
= \sum_{k = 0}^{N-1} A^k.
\end{dmath}

As with a scalar geometric series the sum can be found by subtracting the difference \( S_N - S_{N+1} \)

\begin{equation}\label{eqn:ProblemSetIProblem2:300}
S_N - S_{N+I}
=
S_N(I - A)
=
I - A^N.
\end{equation}

If it can be shown that \( A^N \rightarrow 0 \), then this shows that \( I - A \) must be non-singular.  Additionally, rearranging for \( S^N \), this is

\begin{dmath}\label{eqn:ProblemSetIProblem2:320}
S_N = (I - A^N)(I-A)^{-1},
\end{dmath}

so if \( I - A^N \rightarrow I \), then \cref{eqn:ProblemSet1Problem2:60} has been proven.  Intuitively, if \( A^N \rightarrow 0 \), then \( I - A^N \rightarrow I \), but to make this more meaningful, we want to express this limiting statement in terms of a norm.  The matrix p-norm of this difference is

\begin{dmath}\label{eqn:ProblemSet1Problem2:340}
\Norm{I - A^N}_p
= \sup_{\Bx \ne 0} \frac{ \Norm{ (I - A^N) \Bx }_p }{\Norm{\Bx}_p}
= \max_{\Norm{\Bx}_p = 1} \Norm{ (I - A^N) \Bx }_p
= \max_{\Norm{\Bx}_p = 1} \Norm{ \Bx - A^N \Bx }_p
\le \max_{\Norm{\Bx}_p = 1} \Norm{ \Bx }_p + \Norm{ A^N \Bx }_p
= 1 + \max_{\Norm{\Bx}_p = 1} \Norm{ A^N \Bx }_p.
\end{dmath}

The second last step uses the triangle law property of the norm.  Note that we did not prove that the p-norm has this property in class (called the Minkowski inequality for p-norms), but a nice proof can be found in \citep{tisdellMinkowski}.

\Cref{eqn:ProblemSet1Problem2:340} shows that if \( \Norm{A^N}_p \rightarrow 0 \), that \( I - A \) is non-singular, and the inverse is given by \cref{eqn:ProblemSet1Problem2:60}.

To examine this limit,
first consider the p-norm of a matrix-vector product

\begin{dmath}\label{eqn:ProblemSet1Problem2:360}
\Norm{A \Bx}_p
=
\lr{ \sum_{i = 1}^n \Abs{a_{ij} x_j}^p }^{1/p}
\le
\lr{ \sum_{i = 1}^n \Abs{a_{ij}}^p \Abs{x_j}^p }^{1/p}
\le
\lr{ \sum_{i = 1}^n \Abs{a_{ij}}^p }^{1/p} \lr{ \sum_{i = 1}^n \Abs{x_j}^p }^{1/p}
=
\Norm{A}_p \Norm{\Bx}_p.
\end{dmath}

For the norm of a product of matrices acting on a vector, this means that

\begin{dmath}\label{eqn:ProblemSet1Problem2:380}
\Norm{ A B \Bx }_p \le
\Norm{ A }_p \Norm{B \Bx }_p
\le
\Norm{ A }_p \Norm{B}_p \Norm{ \Bx }_p,
\end{dmath}

in particular

\begin{dmath}\label{eqn:ProblemSet1Problem2:400}
\Norm{A^N \Bx}_p \le \lr{ \Norm{A}_p }^N \Norm{\Bx}_p,
\end{dmath}

and finally
\begin{dmath}\label{eqn:ProblemSet1Problem2:420}
\Norm{I - A^N}_p
\le
1 + \max_{\Norm{\Bx}_p = 1} \lr{ \Norm{ A }_p }^N \Norm{\Bx}_p
= 1 + \lr{ \Norm{ A }_p }^N.
\end{dmath}

This tends to \( 1 \) as \( N \rightarrow \infty \) provided \( \lr{ \Norm{ A }_p } < 1\), proving the inverse power series identity given this constraint.  The only remaining thing to proof is to find the bound on the p-norm of \( (1-A)^{-1} \).  That is

\begin{dmath}\label{eqn:ProblemSet1Problem2:440}
\Norm{ (I - A)^{-1} }_p
=
\Norm{
\sum_{k=0}^\infty A^k
}_p
\le
\sum_{k=0}^\infty
\Norm{
A^k
}_p
\le
\sum_{k=0}^\infty
\lr{ \Norm{ A }_p }^k
= \inv{ 1 - \Norm{A}_p }. \qedmarker
\end{dmath}
}
