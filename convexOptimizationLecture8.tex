%
% Copyright � 2017 Peeter Joot.  All Rights Reserved.
% Licenced as described in the file LICENSE under the root directory of this GIT repository.
%
\input{../latex/blogpost.tex}
\renewcommand{\basename}{convexOptimization8}
\renewcommand{\dirname}{notes/ece1505/}
\newcommand{\keywords}{ECE1505H}
\input{../latex/peeter_prologue_print2.tex}

\usepackage{ece1505}
\usepackage{peeters_braket}
%\usepackage{peeters_layout_exercise}
\usepackage{peeters_figures}
\usepackage{mathtools}
\usepackage{siunitx}
\usepackage{enumerate}

\beginArtNoToc
\generatetitle{ECE1505H Convex Optimization.  Lecture 8: Local vs. Global, and composition of functions.  Taught by Prof.\ Stark Draper}
%\chapter{Local vs. Global, and composition of functions}
\label{chap:convexOptimization8}

\section{Disclaimer}

Peeter's lecture notes from class.  These may be incoherent and rough.

These are notes for the UofT course ECE1505H, Convex Optimization, taught by Prof. Stark Draper, from \citep{boyd2004convex}.

\section{Today}

\begin{itemize}
\item Finish local vs global.
\item Compositions of functions.
\item Introduction to convex optimization problems.
\end{itemize}

\section{Continuing proof:}

We want to prove that if

\begin{equation*}
\begin{aligned}
\spacegrad F(\Bx^\conj) &= 0 \\
\spacegrad^2 F(\Bx^\conj) \ge 0
\end{aligned},
\end{equation*}

then \( \Bx^\conj\) is a local optimum.

Proof:

Again, using Taylor approximation

\begin{dmath}\label{eqn:convexOptimizationLecture8:20}
F(\Bx^\conj + \Bv) = F(\Bx^\conj) + \lr{ \spacegrad F(\Bx^\conj)}^\T \Bv + \inv{2} \Bv^\T \spacegrad^2 F(\Bx^\conj) \Bv + o(\Norm{\Bv}^2)
\end{dmath}

The linear term is zero by assumption, whereas the Hessian term is given as \( > 0 \).  Any direction that you move in, if your move is small enough, this is going uphill at a local optimum.

\section{Summarize:}

For twice continuously differentiable functions, at a local optimum \( \Bx^\conj \), then

\begin{dmath}\label{eqn:convexOptimizationLecture8:40}
\begin{aligned}
\spacegrad F(\Bx^\conj) &= 0 \\
\spacegrad^2 F(\Bx^\conj) \ge 0
\end{aligned}
\end{dmath}

If, in addition, \( F \) is convex, then \( \spacegrad F(\Bx^\conj) = 0 \) implies that \( \Bx^\conj \) is a global optimum.  i.e. for (unconstrained) convex functions, local and global optimums are equivalent.

\begin{itemize}
\item It is possible that a convex function does not have a global optimum.  Examples are \( F(x) = e^x \)
(\cref{fig:l8exponential:l8exponentialFig1})
, which has an \( \inf \), but no lowest point.

\imageFigure{../figures/ece1505-convex-optimization/l8exponentialFig1}{Exponential has no global optimum.}{fig:l8exponential:l8exponentialFig1}{0.2}

\item Our discussion has been for unconstrained functions.  For constrained problems (next topic) is not not necessarily true that \( \spacegrad F(\Bx) = 0 \) implies that \( \Bx \) is a global optimum, even for \( F \) convex.

As an example of a constrained problem consider

\begin{dmath}\label{eqn:convexOptimizationLecture8:n}
\begin{aligned}
\min &2 x^2 + y^2 \\
x &\ge 3 \\
y &\ge 5.
\end{aligned}
\end{dmath}

The level sets of this objective function are plotted in \cref{fig:l8constrainedMin:l8constrainedMinFig2}.  The optimal point is at \( \Bx^\conj = (3,5) \), where \( \spacegrad F \ne 0 \).

\imageFigure{../figures/ece1505-convex-optimization/l8constrainedMinFig2}{Constrained problem with optimum not at the zero gradient point.}{fig:l8constrainedMin:l8constrainedMinFig2}{0.3}
\end{itemize}

\section{Projection}

Given \( \Bx \in \bbR^n, \By \in \bbR^p \), if \( h(\Bx,\By) \) is convex in \( \Bx, \By \), then

\begin{dmath}\label{eqn:convexOptimizationLecture8:60}
F(\Bx_0) = \inf_\By h(\Bx_0,\By)
\end{dmath}

is convex in \( \Bx\), as sketched in \cref{fig:l8convexProjection:l8convexProjectionFig3}.

%F3 : epigraph of \( h \) is a filled in bowl.
\imageFigure{../figures/ece1505-convex-optimization/l8convexProjectionFig3}{Epigraph of \( h \) is a filled bowl.}{fig:l8convexProjection:l8convexProjectionFig3}{0.4}

The intuition here is that shining light on the (filled) ``bowl''.  That is, the image of \( \epi h \) on the \( \By = 0 \) screen which we will show is a convex set.

Proof:

Since \( h \) is convex in \( \begin{bmatrix} \Bx \\ \By \end{bmatrix} \in \dom h \), then

\begin{equation}\label{eqn:convexOptimizationLecture8:80}
\epi h = \setlr{ (\Bx,\By,t) | t \ge h(\Bx,\By), \begin{bmatrix} \Bx \\ \By \end{bmatrix} \in \dom h },
\end{equation}

is a convex set.

We also have to show that the domain of \( F \) is a convex set.  To show this note that

\begin{dmath}\label{eqn:convexOptimizationLecture8:100}
\begin{aligned}
\dom F
&= \setlr{ \Bx | \exists \By s.t. \begin{bmatrix} \Bx \\ \By \end{bmatrix} \in \dom h } \\
&= \setlr{
\begin{bmatrix}
I_{n\times n} & 0_{n \times p}
\end{bmatrix}
\begin{bmatrix}
\Bx \\
\By
\end{bmatrix}
| \begin{bmatrix} \Bx \\ \By \end{bmatrix} \in \dom h
}.
\end{aligned}
\end{dmath}

This is an affine map of a convex set.  Therefore \( \dom F \) is a convex set.

\begin{dmath}\label{eqn:convexOptimizationLecture8:120}
\begin{aligned}
\epi F
&=
\setlr{ \begin{bmatrix} \Bx \\ \By \end{bmatrix} | t \ge \inf h(\Bx,\By), \Bx \in \dom F, \By: \begin{bmatrix} \Bx \\ \By \end{bmatrix} \in \dom h } \\
&=
\setlr{
\begin{bmatrix}
I & 0 & 0 \\
0 & 0 & 1
\end{bmatrix}
\begin{bmatrix}
\Bx \\
\By \\
t
\end{bmatrix}
|
t \ge h(\Bx,\By), \begin{bmatrix} \Bx \\ \By \end{bmatrix} \in \dom h
}.
\end{aligned}
\end{dmath}

\paragraph{Example:}

The function

\begin{dmath}\label{eqn:convexOptimizationLecture8:140}
F(\Bx) = \inf_{\By \in C} \Norm{ \Bx - \By },
\end{dmath}

over \( \Bx \in \bbR^n, \By \in C \), ,is convex if \( C \) is a convex set.  Reason:

\begin{itemize}
\item \( \Bx - \By \) is linear in \((\Bx, \By)\).
\item \( \Norm{ \Bx - \By } \) is a convex function if the domain is a convex set
\item The domain is \( \bbR^n \times C \).  This will be a convex set if \( C \) is.
\item \( h(\Bx, \By) = \Norm{\Bx -\By} \) is a convex function if \( \dom h \) is a convex set.  By setting \( \dom h = \bbR^n \times C \), if \( C \) is convex, \( \dom h \) is a convex set.
\item \( F() \)
\end{itemize}

\section{Composition of functions}

Consider

\begin{dmath}\label{eqn:convexOptimizationLecture8:160}
\begin{aligned}
F(\Bx) &= h(g(\Bx)) \\
\dom F &= \setlr{ \Bx \in \dom g | g(\Bx) \in \dom h } \\
F &: \bbR^n \rightarrow \bbR \\
g &: \bbR^n \rightarrow \bbR \\
h &: \bbR \rightarrow \bbR.
\end{aligned}
\end{dmath}

Cases:
\begin{enumerate}[(a)]
\item \( g \) is convex, \( h \) is convex and non-decreasing.
\item \( g \) is convex, \( h \) is convex and non-increasing.
\end{enumerate}

Show for 1D case ( \( n = 1 \)).  Get to \( n > 1 \) by applying to all lines.

\begin{enumerate}[(a)]
\item
\begin{dmath}\label{eqn:convexOptimizationLecture8:180}
\begin{aligned}
F'(x) &= h'(g(x)) g'(x)
F''(x) &=
h''(g(x)) g'(x) g'(x)
+
h'(g(x)) g''(x)
=
h''(g(x)) (g'(x))^2
+
h'(g(x)) g''(x)
= 
\lr{ \ge 0 } \cdot \lr{ \ge 0 }^2 + \lr{ \ge 0 } \cdot \lr{ \ge 0 },
\end{aligned}
\end{dmath}

since \( h \) is respectively convex, and non-decreasing.
\item

\begin{dmath}\label{eqn:convexOptimizationLecture8:180}
\begin{aligned}
F'(x) = 
\lr{ \ge 0 } \cdot \lr{ \ge 0 }^2 + \lr{ \le 0 } \cdot \lr{ \le 0 },
\end{aligned}
\end{dmath}

since \( h \) is respectively convex, and non-increasing, and g is concave.
\end{enumerate}

\section{Extending to multiple dimensions}

\begin{dmath}\label{eqn:convexOptimizationLecture8:200}
\begin{aligned}
F(\Bx)
&= h(g(\Bx))
= h( g_1(\Bx), g_2(\Bx), \cdots g_k(\Bx) ) \\
g &: \bbR^n \rightarrow \bbR \\
h &: \bbR^k \rightarrow \bbR.
\end{aligned}
\end{dmath}

is convex if \( g_i \) is convex for each \( i \in [1,k] \) and \( h \) is convex and non-decreasing in each argument.

Proof:

again assume \( n = 1 \), without loss of generality,

\begin{dmath}\label{eqn:convexOptimizationLecture8:220}
\begin{aligned}
g &: \bbR \rightarrow \bbR^k \\
h &: \bbR^k \rightarrow \bbR \\
\end{aligned}
\end{dmath}

\begin{dmath}\label{eqn:convexOptimizationLecture8:240}
F''(\Bx)
=
\begin{bmatrix}
g_1(\Bx) & g_2(\Bx) & \cdots & g_k(\Bx)
\end{bmatrix}
\spacegrad^2 h(g(\Bx))
\begin{bmatrix}
g_1'(\Bx) \\ g_2'(\Bx) \\ \vdots \\ g_k'(\Bx)
\end{bmatrix}
+
\lr{ \spacegrad h(g(x)) }^\T
\begin{bmatrix}
g_1''(\Bx) \\ g_2''(\Bx) \\ \vdots \\ g_k''(\Bx)
\end{bmatrix}
\end{dmath}

The Hessian is PSD.

\paragraph{Example:}

\begin{dmath}\label{eqn:convexOptimizationLecture8:260}
F(x) = \exp( g(x) ) = h( g(x) ),
\end{dmath}

where \( g \) is convex is convex, and \( h(y) = e^y \).  This implies that \( F \) is a convex function.

\paragraph{Example:}

\begin{dmath}\label{eqn:convexOptimizationLecture8:280}
F(x) = \inv{g(x)},
\end{dmath}

is convex if \( g(x) \) is concave and positive.  The most simple such example of such a function is \( h(x) = 1/x, \dom h = \bbR_{++} \), which is plotted in \cref{fig:l8oneOverX:l8oneOverXFig6}.

\imageFigure{../figures/ece1505-convex-optimization/l8oneOverXFig6}{Inverse function is convex over positive domain.}{fig:l8oneOverX:l8oneOverXFig6}{0.2}

\paragraph{Example:}

\begin{equation}\label{eqn:convexOptimizationLecture8:300}
F(x) = - \sum_{i = 1}^n \log( -F_i(x) )
\end{equation}

is convex on \( \setlr{ x | F_i(x) < 0 \forall i } \) if all \( F_i \) are convex.

\begin{itemize}
\item Due to \( \dom F \), \( -F_i(x) > 0 \,\forall x \in \dom F \)
\item \( \log(x) \) concave on \( \bbR_{++} \) so \( -\log \) convex also non-increasing (\cref{fig:l8MinusLogX:l8MinusLogXFig7}).

\imageFigure{../figures/ece1505-convex-optimization/l8MinusLogXFig7}{Negative logarithm convex over positive domain.}{fig:l8MinusLogX:l8MinusLogXFig7}{0.2}

\end{itemize}

\begin{dmath}\label{eqn:convexOptimizationLecture8:320}
F(x) = \sum h_i(x)
\end{dmath}

but
\begin{dmath}\label{eqn:convexOptimizationLecture8:340}
h_i(x) = -\log(-F_i(x)),
\end{dmath}

which is a convex and non-increasing function (\(-\log\)), of a convex function \( -F_i(x) \).  Each 
\( h_i \) is convex, so this is a sum of convex functions, and is therefore convex.

\paragraph{Example:}

Over \( \dom F = S^n_{++} \)

\begin{dmath}\label{eqn:convexOptimizationLecture8:360}
F(X) = \log \det X^{-1}
\end{dmath}

To show that this is convex, check all lines in domain.  A line in \( S^n_{++} \) is a 1D family of matrices

\begin{dmath}\label{eqn:convexOptimizationLecture8:380}
\tilde{F}(t) = \log \det( \lr{X_0 + t H}^{-1} ),
\end{dmath}

where \( X_0 \in S^n_{++}, t \in \bbR, H \in S^n \).

F9

For \( t \) small enough,

\begin{dmath}\label{eqn:convexOptimizationLecture8:400}
X_0 + t H \in S^n_{++}
\end{dmath}

\begin{dmath}\label{eqn:convexOptimizationLecture8:420}
\tilde{F}(t)
= \log \det( \lr{X_0 + t H}^{-1} )
= \log \det\lr{ X_0^{-1/2} \lr{I + t X_0^{-1/2} H X_0^{-1/2} }^{-1} X_0^{-1/2} }
= \log \det\lr{ X_0^{-1} \lr{I + t X_0^{-1/2} H X_0^{-1/2} }^{-1} }
= \log \det X_0^{-1} + \log\det \lr{I + t X_0^{-1/2} H X_0^{-1/2} }^{-1}
= \log \det X_0^{-1} - \log\det \lr{I + t X_0^{-1/2} H X_0^{-1/2} }
= \log \det X_0^{-1} - \log\det \lr{I + t M }.
\end{dmath}

If \( \lambda_i \) are eigenvalues of \( M \), then \( 1 + t \lambda_i \) are eigenvalues of \( I + t M \).  i.e.:

\begin{dmath}\label{eqn:convexOptimizationLecture8:440}
(I + t M) \Bv
 =
I \Bv + t \lambda_i \Bv
=
(1 + t \lambda_i) \Bv.
\end{dmath}

This gives

\begin{dmath}\label{eqn:convexOptimizationLecture8:460}
\tilde{F}(t)
= \log \det X_0^{-1} - \log \prod_{i = 1}^n (1 + t \lambda_i)
= \log \det X_0^{-1} - \sum_{i = 1}^n \log (1 + t \lambda_i)
\end{dmath}

\begin{itemize}
\item \( 1 + t \lambda_i \) is linear in \( t \).
\item \( -\log \) is convex in its argument.
\item sum of convex function is convex.
\end{itemize}

\paragraph{Example:}

\begin{dmath}\label{eqn:convexOptimizationLecture8:480}
F(X) = \lambda_\max(X),
\end{dmath}

is convex on \( \dom F \in S^n \)

(a)
\begin{dmath}\label{eqn:convexOptimizationLecture8:500}
\lambda_{\max} (X) = \sup_{\Norm{\Bv}_2 \le 1} \Bv^\T X \Bv,
\end{dmath}

\begin{dmath}\label{eqn:convexOptimizationLecture8:520}
\begin{bmatrix}
\lambda_1 &
          & \lambda_2 &
          &           & \ddots &
          &           &        & \lambda_n
\end{bmatrix}
\end{dmath}

Recall that a decomposition

\begin{dmath}\label{eqn:convexOptimizationLecture8:540}
\begin{aligned}
X &= Q \Lambda Q^\T \\
Q^\T Q = Q Q^\T = I
\end{aligned}
\end{dmath}

can be used for any \( X \in S^n \).

(b)

Note that \( \Bv^\T X \Bv \) is linear in \( X \).  This is a max of a number of linear (and convex) functions, so it is convex.

Last example:

(non-symmetric matrices)

\begin{dmath}\label{eqn:convexOptimizationLecture8:560}
F(X) = \sigma_\max(X),
\end{dmath}

is convex on \( \dom F = \bbR^{m \times n} \).  Here

\begin{dmath}\label{eqn:convexOptimizationLecture8:580}
\sigma_\max(X) = \sup_{\Norm{\Bv}_2 = 1} \Norm{X \Bv}_2
\end{dmath}

This is called an operator norm of \( X \).  Using the SVD

\begin{dmath}\label{eqn:convexOptimizationLecture8:600}
\begin{aligned}
X &= U \Sigma V^\T \\
U &= \bbR^{m \times r} \\
\Sigma &\in \diag \in \bbR{ r \times r } \\
V^T &\in \bbR^{r \times n}.
\end{aligned}
\end{dmath}

Have

\begin{dmath}\label{eqn:convexOptimizationLecture8:620}
\Norm{X \Bv}_2^2
=
\Norm{ U \Sigma V^\T \Bv }_2^2
=
\Bv^\T V \Sigma U^\T U \Sigma V^\T \Bv
=
\Bv^\T V \Sigma \Sigma V^\T \Bv
=
\Bv^\T V \Sigma^2 V^\T \Bv
=
\tilde{\Bv}^\T \Sigma^2 \tilde{\Bv},
\end{dmath}

where \( \tilde{\Bv} = \Bv^\T V \), so

\begin{dmath}\label{eqn:convexOptimizationLecture8:640}
\Norm{X \Bv}_2^2
=
\sum_{i = 1}^r \sigma_i^2 \Norm{\tilde{\Bv}}
\le \sigma_\max^2 \Norm{\tilde{\Bv}}^2,
\end{dmath}

or
\begin{dmath}\label{eqn:convexOptimizationLecture8:660}
\Norm{X \Bv}_2
\le \sqrt{ \sigma_\max^2 } \Norm{\tilde{\Bv}}
\le
\sigma_\max.
\end{dmath}

Set \( \Bv \) to the right singular value of \( X \) to get equality.

\EndArticle
