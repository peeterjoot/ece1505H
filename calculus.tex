%
% Copyright © 2017 Peeter Joot.  All Rights Reserved.
% Licenced as described in the file LICENSE under the root directory of this GIT repository.
%
\subsubsection{Multivariable Taylor approximation}

The Taylor series expansion for a scalar function \( g : \bbR \rightarrow \bbR \) about the origin is just

\begin{dmath}\label{eqn:jacobianAndHessian:20}
g(t) = g(0) + t g'(0) + \frac{t^2}{2} g''(0) + \cdots
\end{dmath}

In particular

\begin{dmath}\label{eqn:jacobianAndHessian:40}
g(1) = g(0) + g'(0) + \frac{1}{2} g''(0) + \cdots
\end{dmath}

Now consider \( g(t) = f( \Bx + \Ba t ) \), where \( f : \bbR^n \rightarrow \bbR \), \( g(0) = f(\Bx) \), and \( g(1) = f(\Bx + \Ba) \).  
This trick, from
\citep{hestenes1999nfc}
allows for a direct expansion of the multivariable Taylor series of a scalar function

\begin{dmath}\label{eqn:jacobianAndHessian:60}
f( \Bx + \Ba)
= f(\Bx)
+ \evalbar{\frac{df(\Bx + \Ba t)}{dt}}{t = 0} + \frac{1}{2} \evalbar{\frac{d^2f(\Bx + \Ba t)}{dt^2}}{t = 0} + \cdots
\end{dmath}

The first order term is

\begin{dmath}\label{eqn:jacobianAndHessian:80}
\evalbar{\frac{df(\Bx + \Ba t)}{dt}}{t = 0}
=
\sum_{i = 1}^n
\frac{d( x_i + a_i t)}{dt}
\evalbar{\PD{(x_i + a_i t)}{f(\Bx + \Ba t)}}{t = 0}
=
\sum_{i = 1}^n
a_i
\PD{x_i}{f(\Bx)}
= \Ba \cdot \spacegrad f.
\end{dmath}

Similarily, for the second order term

\begin{dmath}\label{eqn:jacobianAndHessian:100}
\evalbar{\frac{d^2 f(\Bx + \Ba t)}{dt^2}}{t = 0}
=
\evalbar{\lr{
   \frac{d}{dt}
   \lr{
   \sum_{i = 1}^n
   a_i
   \PD{(x_i + a_i t)}{f(\Bx + \Ba t)}
   }
   }
}{t = 0}
=
\evalbar{
   \lr{
      \sum_{j = 1}^n
      \frac{d(x_j + a_j t)}{dt}
         \sum_{i = 1}^n
         a_i
      \frac{\partial^2 f(\Bx + \Ba t)}{\partial (x_j + a_j t) \partial (x_i + a_i t) }
   }
}{t = 0}
=
\sum_{i,j = 1}^n a_i a_j \frac{\partial^2 f}{\partial x_i \partial x_j}
=
(\Ba \cdot \spacegrad)^2 f.
\end{dmath}

The complete Taylor expansion of a scalar function \( f : \bbR^n \rightarrow \bbR \) is therefore

\begin{dmath}\label{eqn:jacobianAndHessian:120}
f(\Bx + \Ba)
= f(\Bx) +
\Ba \cdot \spacegrad f +
\inv{2} \lr{ \Ba \cdot \spacegrad}^2 f + \cdots,
\end{dmath}

so the Taylor expansion has an exponential structure

\begin{equation}\label{eqn:jacobianAndHessian:140}
f(\Bx + \Ba) = \sum_{k = 0}^\infty \inv{k!} \lr{ \Ba \cdot \spacegrad}^k f = e^{\Ba \cdot \spacegrad} f.
\end{equation}

Should an approximation of a vector valued function \( \Bf : \bbR^n \rightarrow \bbR^m \) be desired it is only required to form a matrix of the components

\begin{equation}\label{eqn:jacobianAndHessian:160}
\Bf(\Bx + \Ba)
= \Bf(\Bx) +
[\Ba \cdot \spacegrad f_i]_i +
\inv{2} [\lr{ \Ba \cdot \spacegrad}^2 f_i]_i + \cdots,
\end{equation}

where \( [.]_i \) denotes a column vector over the rows \( i \in [1,m] \), and \( f_i \) are the coordinates of \( \Bf \).

\subsubsection{The Jacobian matrix}

In \citep{boyd2004convex} the Jacobian \( D \Bf \) of a function \( \Bf : \bbR^n \rightarrow \bbR^m \) is defined in terms of the limit of the \( l_2 \) norm ratio

\begin{dmath}\label{eqn:jacobianAndHessian:180}
\frac{\Norm{\Bf(\Bz) - \Bf(\Bx) - (D \Bf) (\Bz - \Bx)}_2 }{ \Norm{\Bz - \Bx}_2 },
\end{dmath}

with the statement that the function \( \Bf \) has a derivative if this limit exists.  Here the Jacobian \( D \Bf \in \bbR^{m \times n} \) must be matrix valued.

Let \( \Bz = \Bx + \Ba \), so the first order expansion of \cref{eqn:jacobianAndHessian:160} is

\begin{dmath}\label{eqn:jacobianAndHessian:200}
\Bf(\Bz)
= \Bf(\Bx) + [\lr{ \Bz - \Bx } \cdot \spacegrad f_i]_i
.
%+ \inv{2} [(\lr{ \Bz - \Bx } \cdot \spacegrad)^2 f_i]_i.
\end{dmath}

With the (unproven) assumption that this Taylor expansion satisfies the norm limit criteria of \cref{eqn:jacobianAndHessian:180}, it is possible to extract the structure of the Jacobian by comparison

\begin{dmath}\label{eqn:jacobianAndHessian:220}
(D \Bf)
(\Bz - \Bx)
=
{\begin{bmatrix}
\lr{ \Bz - \Bx } \cdot \spacegrad f_i
\end{bmatrix}}_i
=
{\begin{bmatrix}
\sum_{j = 1}^n (z_j - x_j) \PD{x_j}{f_i}
\end{bmatrix}}_i
=
{\begin{bmatrix}
\PD{x_j}{f_i}
\end{bmatrix}}_{ij}
(\Bz - \Bx),
\end{dmath}

so
%\begin{dmath}\label{eqn:jacobianAndHessian:240}
\boxedEquation{eqn:jacobianAndHessian:240}{
(D \Bf)_{ij} = \PD{x_j}{f_i}
}
%\end{dmath}

Written out explictly as a matrix the Jacobian is

\begin{equation}\label{eqn:jacobianAndHessian:320}
D \Bf
=
\begin{bmatrix}
\PD{x_1}{f_1} & \PD{x_2}{f_1} & \cdots & \PD{x_n}{f_1} \\
\PD{x_1}{f_2} & \PD{x_2}{f_2} & \cdots & \PD{x_n}{f_2} \\
\vdots & \vdots & & \vdots \\
\PD{x_1}{f_m} & \PD{x_2}{f_m} & \cdots & \PD{x_n}{f_m} \\
\end{bmatrix}
=
\begin{bmatrix}
(\spacegrad f_1)^\T \\
(\spacegrad f_2)^\T \\
\vdots \\
(\spacegrad f_m)^\T
\end{bmatrix}.
\end{equation}

In particular, when the function is scalar valued
\begin{dmath}\label{eqn:jacobianAndHessian:360}
D f = (\spacegrad f)^\T.
\end{dmath}

With this notation, the first Taylor expansion, in terms of the Jacobian matrix is

%\begin{dmath}\label{eqn:jacobianAndHessian:260}
\boxedEquation{eqn:jacobianAndHessian:260}{
\Bf(\Bz)
\approx \Bf(\Bx) + (D \Bf) \lr{ \Bz - \Bx }.
}
%\end{dmath}

\subsubsection{The Hessian matrix}

For scalar valued functions, the text expresses the second order expansion of a function in terms of the Jacobian and Hessian matrices

\begin{dmath}\label{eqn:jacobianAndHessian:380}
f(\Bz)
\approx f(\Bx) + (D f) \lr{ \Bz - \Bx }
+ \inv{2} \lr{ \Bz - \Bx }^\T (\spacegrad^2 f) \lr{ \Bz - \Bx }.
\end{dmath}

Because \( \spacegrad^2 \) is the usual notation for a Laplacian operator, this \( \spacegrad^2 f \in \bbR^{n \times n}\) notation for the Hessian matrix is not ideal in my opinion.  Ignoring that notational objection for this class, the structure of the Hessian matrix can be extracted by comparison with the coordinate expansion

\begin{dmath}\label{eqn:jacobianAndHessian:300}
\Ba^\T (\spacegrad^2 f) \Ba
=
\sum_{r,s = 1}^n a_r a_s \frac{\partial^2 f}{\partial x_r \partial x_s}
\end{dmath}

so
%\begin{dmath}\label{eqn:jacobianAndHessian:280}
\boxedEquation{eqn:jacobianAndHessian:280}{
(\spacegrad^2 f)_{ij}
=
\frac{\partial^2 f_i}{\partial x_i \partial x_j}.
}
%\end{dmath}

In explicit matrix form the Hessian is

\begin{dmath}\label{eqn:jacobianAndHessian:340}
\spacegrad^2 f
=
\begin{bmatrix}
\frac{\partial^2 f}{\partial x_1 \partial x_1} & \frac{\partial^2 f}{\partial x_1 \partial x_2} & \cdots &\frac{\partial^2 f}{\partial x_1 \partial x_n} \\
\frac{\partial^2 f}{\partial x_2 \partial x_1} & \frac{\partial^2 f}{\partial x_2 \partial x_2} & \cdots &\frac{\partial^2 f}{\partial x_2 \partial x_n} \\
\vdots & \vdots & & \vdots \\
\frac{\partial^2 f}{\partial x_n \partial x_1} & \frac{\partial^2 f}{\partial x_n \partial x_2} & \cdots &\frac{\partial^2 f}{\partial x_n \partial x_n}
\end{bmatrix}.
\end{dmath}

Is there a similar nice matrix structure for the Hessian of a function \( f : \bbR^n \rightarrow \bbR^m \)?

