%
% Copyright � 2017 Peeter Joot.  All Rights Reserved.
% Licenced as described in the file LICENSE under the root directory of this GIT repository.
%
\input{../latex/blogpost.tex}
\renewcommand{\basename}{convexOptimization6}
\renewcommand{\dirname}{notes/ece1505/}
\newcommand{\keywords}{ECE1505H}
\input{../latex/peeter_prologue_print2.tex}

\usepackage{ece1505}
\usepackage{peeters_braket}
\usepackage{peeters_layout_exercise}
\usepackage{peeters_figures}
\usepackage{macros_qed}
\usepackage{mathtools}
\usepackage{siunitx}
\usepackage{enumerate}

\beginArtNoToc
\generatetitle{ECE1505H Convex Optimization.  Lecture 6: First and second order conditions.  Taught by Prof.\ Stark Draper}
%\chapter{First and second order conditions}
\label{chap:convexOptimization6}

\paragraph{Disclaimer}

Peeter's lecture notes from class.  These may be incoherent and rough.

These are notes for the UofT course ECE1505H, Convex Optimization, taught by Prof. Stark Draper, covering \textchapref{{1}} \citep{boyd2004convex} content.

\paragraph{Today}

\begin{itemize}
\item First and second order conditions for convexity of differentiable functions.
\item Consequences of convexity: local and global optimizality.
\item Properties.
\end{itemize}

\paragraph{Quasi-convex}

\( F_1 \) and \( F_2 \) convex implies \( \max( F_1, F_2) \) convex.

F1

Note that \( \min(F_1, F_2) \) is NOT convex.

If \( F : \bbR^n \rightarrow \bbR \) is convex, then
\( F( \Bx_0 + t \Bv ) \) is convex in \( t\,\forall t \in \bbR, \Bx_0 \in \bbR^n, \Bv \in \bbR^n \), provided \( \Bx_0 + t \Bv \in \dom F \).

Idea: Restrict to a line (line segment) in \( \dom F \).  Take a cross section or slice through \( F \) alone the line.  If the result is a 1D convex function for all slices, then \( F \) is convex.

This is nice since it allows for checking for convexity, and is also nice numerically.  Attempting to test a given data set for non-convexity with some random lines can help disprove convexity.   However, to show that \( F \) is convex it is required to test all possible slices (which isn't possible numerically, but is in some circumstances possible analytically).

\paragraph{Differentiable (convex) functions}

\makedefinition{First order condition}{dfn:convexOptimizationLecture6:20}{
If

\begin{equation*}
F : \bbR^n \rightarrow \bbR
\end{equation*}

is differentiable, then \( F \) is convex iff \( \dom F \) is a convex set and \( \forall \Bx, \Bx_0 \in \dom F \)

\begin{equation*}
F(\Bx) \ge F(\Bx_0) + \lr{\spacegrad F(\Bx_0)}^\T (\Bx - \Bx_0).
\end{equation*}
} % definition

This is the first order Taylor expansion.  If \( n = 1 \), this is \( F(x) \ge F(x_0) + F'(x_0) ( x - x_0) \).

F3

The first order condition says a convex function \underline{always} lies above its first order approximation.  When differentiable, the supporting plane is the tangent plane.
When differentiable, the

\makedefinition{Second order condition}{dfn:convexOptimizationLecture6:40}{
If \( F : \bbR^n \rightarrow \bbR \)
is twice differentiable, then \( F \) is convex iff \( \dom F \) is a convex set and \( \spacegrad^2 F(\Bx) \ge 0 \,\forall \Bx \in \dom F\).
} % definition

The Hessian is always symmetric, but is not necesssarily positive.  Recall that the Hessian is the matrix of the second order partials \( (\spacegrad F)_{ij} = \partial^2 F/(\partial x_i \partial x_j) \).

The scalar case is \( F''(x) \ge 0 \, \forall x \in \dom F \).

An implication is that if \( F \) is convex, then \( F(x) \ge F(x_0) + F'(x_0) (x - x_0) \,\forall x, x_0 \in \dom F\)

Since \( F \) is convex, \( \dom F \) is convex.

Consider any 2 points \( x, y \in \dom F \), and \( \theta \in [0,1] \).  Define

\begin{dmath}\label{eqn:convexOptimizationLecture6:60}
z = (1-\theta) x + \theta y \in \dom F,
\end{dmath}

then since \( \dom F \) is convex

\begin{dmath}\label{eqn:convexOptimizationLecture6:80}
F(z) =
F( (1-\theta) x + \theta y )
\le
(1-\theta) F(x) + \theta F(y )
\end{dmath}

Reordering

\begin{dmath}\label{eqn:convexOptimizationLecture6:220}
\theta F(x) \ge
\theta F(x) + F(z) - F(x),
\end{dmath}

or
\begin{dmath}\label{eqn:convexOptimizationLecture6:100}
F(y) \ge
F(x) + \frac{F(x + \theta(y-x)) - F(x)}{\theta},
\end{dmath}

which is, in the limit,

\begin{dmath}\label{eqn:convexOptimizationLecture6:120}
F(y) \ge
F(x) + F'(x) (y - x)\qedmarker
\end{dmath}

To prove the other direction, showing that

\begin{dmath}\label{eqn:convexOptimizationLecture6:140}
F(x) \ge F(x_0) + F'(x_0) (x - x_0),
\end{dmath}

implies that \( F \) is convex.
Take any \( x, y \in \dom F \) and any \( \theta \in [0,1] \).  Define

\begin{dmath}\label{eqn:convexOptimizationLecture6:160}
z = \theta x + (1 -\theta) y,
\end{dmath}

which is in \( \dom F \) by assumption.  We want to show that

\begin{dmath}\label{eqn:convexOptimizationLecture6:180}
F(z) \le \theta F(x) + (1-\theta) F(y).
\end{dmath}

By assumption

\begin{enumerate}[(i)]
\item \( F(x) \ge F(z) + F'(z) (x - z) \)
\item \( F(y) \ge F(z) + F'(z) (y - z) \)
\end{enumerate}

Compute

\begin{dmath}\label{eqn:convexOptimizationLecture6:200}
\theta F(x) + (1-\theta) F(y)
\ge
\theta \lr{  F(z) + F'(z) (x - z) }
+ (1-\theta) \lr{  F(z) + F'(z) (y - z) }
=
F(z) + F'(z) \lr{ \theta( x - z) + (1-\theta) (y-z) }
=
F(z) + F'(z) \lr{ \theta x  + (1-\theta) y - \theta z - (1 -\theta) z }
=
F(z) + F'(z) \lr{ \theta x  + (1-\theta) y - z}
=
F(z) + F'(z) \lr{ z - z}
= F(z).
\end{dmath}

\paragraph{Proof of the 2nd order case for \( n = 1 \)}

Want to prove that if

\begin{dmath}\label{eqn:convexOptimizationLecture6:240}
F : \bbR \rightarrow \bbR
\end{dmath}

is a convex function, then \( F''(x) \ge 0 \,\forall x \in \dom F \).

By the first order conditions \( \forall x \ne y \in \dom F \)

\begin{dmath}\label{eqn:convexOptimizationLecture6:260}
\begin{aligned}
F(y) &\ge F(x) + F'(x) (y - x)
F(x) &\ge F(y) + F'(y) (x - y)
\end{aligned}
\end{dmath}

Can combine and get

\begin{equation}\label{eqn:convexOptimizationLecture6:280}
F'(x) (y-x) \le F(y) - F(x) \le F'(y)(y-x)
\end{equation}

Subtract the two derivative terms for

\begin{dmath}\label{eqn:convexOptimizationLecture6:340}
\frac{(F'(y) - F'(x))(y - x)}{(y - x)^2} \ge 0,
\end{dmath}

or
\begin{dmath}\label{eqn:convexOptimizationLecture6:300}
\frac{F'(y) - F'(x)}{y - x} \ge 0.
\end{dmath}

In the limit as \( y \rightarrow x \), this is
\boxedEquation{eqn:convexOptimizationLecture6:320}{
F''(x) \ge 0 \,\forall x \in \dom F.
}

Now prove the reverse condition:

If \( F''(x) \ge 0 \,\forall x \in \dom F \subseteq \bbR \), implies that \( F : \bbR \rightarrow \bbR \) is convex.

Note that if \( F''(x) \ge 0 \), then \( F'(x) \) is non-decreasing in \( x \).

i.e. If \( x < y \), where \( x, y \in \dom F\), then

\begin{dmath}\label{eqn:convexOptimizationLecture6:360}
F'(x) \le F'(y).
\end{dmath}

Consider any \( x,y \in \dom F\) such that \( x < y \), where

\begin{dmath}\label{eqn:convexOptimizationLecture6:380}
F(y) - F(x) = \int_x^y F'(t) dt \ge F'(x) \int_x^y 1 dt = F'(x) (y-x).
\end{dmath}

This tells us that

\begin{dmath}\label{eqn:convexOptimizationLecture6:400}
F(y) \ge F(x) + F'(x)(y - x),
\end{dmath}

which is the first order condition.  Similarily consider any \( x,y \in \dom F\) such that \( x < y \), where

\begin{dmath}\label{eqn:convexOptimizationLecture6:420}
F(y) - F(x) = \int_x^y F'(t) dt \le F'(y) \int_x^y 1 dt = F'(y) (y-x).
\end{dmath}

This tells us that

\begin{dmath}\label{eqn:convexOptimizationLecture6:440}
F(x) \ge F(y) + F'(y)(x - y).
\end{dmath}

\paragraph{Vector proof:}

\( F \) is convex iff \( F(\Bx + t \Bv) \) is convex \( \forall \Bx,\Bv \in \bbR^n, t \in \bbR \), keeping \( \Bx + t \Bv \in \dom F\).

Let
\begin{dmath}\label{eqn:convexOptimizationLecture6:460}
h(t ; \Bx, \Bv) = F(\Bx + t \Bv)
\end{dmath}

then \( h(t) \) satisfies scalar first and second order conditions for all \( \Bx, \Bv \).

\begin{dmath}\label{eqn:convexOptimizationLecture6:480}
h(t) = F(\Bx + t \Bv) = F(g(t)),
\end{dmath}

where \( g(t) = \Bx + t \Bv \), where

\begin{dmath}\label{eqn:convexOptimizationLecture6:500}
\begin{aligned}
F &: \bbR^n \rightarrow \bbR \\
g &: \bbR \rightarrow \bbR^n.
\end{aligned}
\end{dmath}

This is expressing \( h(t) \) as a composition of two functions.  By the first order condition for scalar functions we know that

\begin{dmath}\label{eqn:convexOptimizationLecture6:520}
h(t) \ge h(0) + h'(0) t.
\end{dmath}

Note that

\begin{equation}\label{eqn:convexOptimizationLecture6:540}
h(0) = \evalbar{F(\Bx + t \Bv)}{t = 0} = F(\Bx).
\end{equation}

Let's figure out what \( h'(0) \) is.  Recall hat for any \( \tilde{F} : \bbR^n \rightarrow \bbR^m \)

\begin{equation}\label{eqn:convexOptimizationLecture6:560}
D \tilde{F} \in \bbR^{m \times n},
\end{equation}

and
\begin{equation}\label{eqn:convexOptimizationLecture6:580}
{D \tilde{F}(\Bx)}_{ij} = \PD{x_j}{\tilde{F_i}(\Bx)}
\end{equation}

This is one function per row, for \( i \in [1,m], j \in [1,n] \).  This gives

\begin{dmath}\label{eqn:convexOptimizationLecture6:600}
\frac{d}{dt} F(\Bx + \Bv t)
=
\frac{d}{dt} F( g(t) )
=
\frac{d}{dt} h(t)
= D h(t)
= D F(g(t))  \cdot D g(t)
\end{dmath}

The first matrix is in \( \bbR^{1\times n} \) whereas the second is in \( \bbR^{n\times 1} \), since
\( F : \bbR^n \rightarrow \bbR \) and \( g : \bbR \rightarrow \bbR^n \).  This gives

\begin{equation}\label{eqn:convexOptimizationLecture6:620}
\frac{d}{dt} F(\Bx + \Bv t)
= \evalbar{D F(\tilde{\Bx})}{\tilde{\Bx} = g(t)}  \cdot D g(t).
\end{equation}

That first matrix is

\begin{dmath}\label{eqn:convexOptimizationLecture6:640}
\evalbar{D F(\tilde{\Bx})}{\tilde{\Bx} = g(t)}
=
\evalbar{
\lr{\begin{bmatrix}
\PD{\tilde{x}_1}{ F(\tilde{\Bx})} &
\PD{\tilde{x}_2}{ F(\tilde{\Bx})} & \cdots
\PD{\tilde{x}_n}{ F(\tilde{\Bx})}
\end{bmatrix}
}}{ \tilde{\Bx} = g(t) = \Bx + t \Bv }
=
\evalbar{
\lr{ \spacegrad F(\tilde{\Bx}) }^\T
}{
\tilde{\Bx} = g(t)
}
=
\lr{ \spacegrad F(g(t)) }^\T.
\end{dmath}

The second Jacobian is

\begin{dmath}\label{eqn:convexOptimizationLecture6:660}
D g(t)
=
D
\begin{bmatrix}
g_1(t) \\
g_2(t) \\
\vdots \\
g_n(t) \\
\end{bmatrix}
=
D
\begin{bmatrix}
x_1 + t v_1 \\
x_2 + t v_2 \\
\vdots \\
x_n + t v_n \\
\end{bmatrix}
=
\begin{bmatrix}
v_1 \\
v_1 \\
\vdots \\
v_n \\
\end{bmatrix}
=
\Bv.
\end{dmath}

so

\begin{dmath}\label{eqn:convexOptimizationLecture6:680}
h'(t) = D h(t) = \lr{ \spacegrad F(g(t))}^\T \Bv,
\end{dmath}

and
\begin{dmath}\label{eqn:convexOptimizationLecture6:700}
h'(0) = \lr{ \spacegrad F(g(0))}^\T \Bv
=
\lr{ \spacegrad F(\Bx)}^\T \Bv.
\end{dmath}

Finally

\begin{dmath}\label{eqn:convexOptimizationLecture6:720}
F(\Bx + t \Bv)
\ge h(0) + h'(0) t
= F(\Bx) + \lr{ \spacegrad F(\Bx) }^\T (t \Bv)
= F(\Bx) + \innerprod{ \spacegrad F(\Bx) }{ t \Bv}.
\end{dmath}

Which is true for all \( \Bx, \Bx + t \Bv \in \dom F \).  Note that the quantity \( t \Bv \) is a shift.

F4

\paragraph{Epigraph}

Recall that if \( (\Bx, t) \in \epi F \) then \( t \ge F(\Bx) \).

\begin{dmath}\label{eqn:convexOptimizationLecture6:740}
t \ge F(\Bx) \ge F(\Bx_0) + \lr{\spacegrad F(\Bx_0) }^\T (\Bx - \Bx_0),
\end{dmath}

or

\begin{dmath}\label{eqn:convexOptimizationLecture6:760}
0 \ge
-(t - F(\Bx_0)) + \lr{\spacegrad F(\Bx_0) }^\T (\Bx - \Bx_0),
\end{dmath}

In block matrix form

\begin{dmath}\label{eqn:convexOptimizationLecture6:780}
0 \ge
\begin{bmatrix}
\lr{ \spacegrad F(\Bx_0) }^\T & -1
\end{bmatrix}
\begin{bmatrix}
\Bx - \Bx_0 \\
t - F(\Bx_0)
\end{bmatrix}
\end{dmath}

With \( \Bw = 
\begin{bmatrix}
\lr{ \spacegrad F(\Bx_0) }^\T & -1
\end{bmatrix} \), the geometry of the epigraph relation to the half plane is sketched in 

F5

\EndArticle
%\EndNoBibArticle
