%
% Copyright � 2017 Peeter Joot.  All Rights Reserved.
% Licenced as described in the file LICENSE under the root directory of this GIT repository.
%
%----------------------------------------------------------------------------------------
%\part{Lecture notes}
   \mychapter{Introduction.}
      %
% Copyright © 2017 Peeter Joot.  All Rights Reserved.
% Licenced as described in the file LICENSE under the root directory of this GIT repository.
%
\section{What's this course about?}

\begin{itemize}
\item Science of optimization.
\item problem formulation, design, analysis of engineering systems.
\end{itemize}

\section{Basic concepts}

\begin{itemize}
\item Basic concepts.  convex sets, functions, problems.
\item Theory (about 40 \% of the material).  Specifically Lagrangian duality.
\item Algorithms: gradient descent, Newton's, interior point, ...
\end{itemize}

Homework will involve computational work (solving problems, ...)

\section{Goals}

\begin{itemize}
\item Recognize and formulate engineering problems as convex optimization problems.
\item To develop (Matlab) code to solve problems numerically.
\item To characterize the solutions via duality theory
\item NOT a math course, but lots of proofs.
\item NOT a communications course, but lots of ... (?)
\item NOT a CS course, but lots of useful algorithms.
\end{itemize}

\makedefinition{Mathematical program}{dfn:L1introduction:1}{
\begin{equation}\label{eqn:L1introduction:20}
\min_\Bx F_0(\Bx)
\end{equation}
where \( \Bx = (x_1, x_2, \cdots, x_m) \in \Rm{m} \) is subject to constraints \( F_i : \Rm{m} \rightarrow \Rm{1} \)
\begin{equation}\label{eqn:L1introduction:40}
F_i(\Bx) \le 0, \qquad i = 1, \cdots, m.
\end{equation}
The function \( F_0 : \Rm{m} \rightarrow \Rm{1} \) is called the ``objective function''.
} % definition

Solving a problem produces:

An optimal \(\Bx^\conj\) is a value \( \Bx \) that gives the smallest value among all the feasible \( \Bx \) for the objective function \( F_0 \).  Such a function is sketched in \cref{fig:ConvexObjectiveFunction:ConvexObjectiveFunctionFig1}.

\imageFigure{../figures/ece1505-convex-optimization/ConvexObjectiveFunctionFig1}{Convex objective function.}{fig:ConvexObjectiveFunction:ConvexObjectiveFunctionFig1}{0.3}

\begin{itemize}
\item A convex objective looks like a bowl, ``holds water''.
\item If connect two feasible points line segment in the ? above bottom of the bowl.
\end{itemize}

A non-convex function is illustrated in \cref{fig:NonConvexObjectiveFunction:NonConvexObjectiveFunctionFig2}, which has a number of local minimums.

\imageFigure{../figures/ece1505-convex-optimization/NonConvexObjectiveFunctionFig2}{Non-convex (wavy) figure with a number of local minimums.}{fig:NonConvexObjectiveFunction:NonConvexObjectiveFunctionFig2}{0.2}

\section{Example: Line fitting.}

A linear fit of some points distributed around a line \( y = a x + b \) is plotted in \cref{fig:LinearFit:LinearFitFig3}.  Here \( a, b \) are the optimization variables \( \Bx = (a, b) \).

\imageFigure{../figures/ece1505-convex-optimization/LinearFitFig3}{Linear fit of points around a line.}{fig:LinearFit:LinearFitFig3}{0.2}

How is the solution for such a best fit line obtained?

\paragraph{Approach 1:  Calculus minimization of a multivariable error function.}
Describe an error function, describing how far from the line a given point is.
\begin{equation}\label{eqn:L1introduction:100}
y_i - (a x_i + b),
\end{equation}
Because this can be positive or negative, we can define a squared variant of this, and then sum over all data points.
\begin{equation}\label{eqn:L1introduction:120}
F_0 = \sum_{i=1}^n \lr{ y_i - (a x_i + b) }^2.
\end{equation}
One way to solve (for \( a, b \)): Take the derivatives
\begin{equation}\label{eqn:L1introduction:140}
\begin{aligned}
\PD{a}{F_0} &= \sum_{i=1}^n 2 ( y_i - (a x_i + b) )(-x_i) = 0 \\
\PD{b}{F_0} &= \sum_{i=1}^n 2 ( y_i - (a x_i + b) )(-1) = 0.
\end{aligned}
\end{equation}
%
This yields
\begin{equation}\label{eqn:L1introduction:160}
\begin{aligned}
\sum_{i = 1}^n y_i     &= \lr{\sum_{i = 1}^n x_i} a + \lr{\sum_{i = 1}^n 1} b \\
\sum_{i = 1}^n x_i y_i &= \lr{\sum_{i = 1}^n x_i^2} a + \lr{\sum_{i = 1}^n x_i} b.
\end{aligned}
\end{equation}
In matrix form, this is
\begin{equation}\label{eqn:L1introduction:180}
\begin{bmatrix}
\sum x_i y_i \\
\sum y_i
\end{bmatrix}
=
\begin{bmatrix}
\sum x_i^2 & \sum x_i \\
\sum x_i & n
\end{bmatrix}
\begin{bmatrix}
a \\
b
\end{bmatrix}.
\end{equation}
If invertible, have an analytic solution for \( (a^\conj, b^\conj) \).  This is a convex optimization problem because \( F(x) = x^2 \) is a convex ``quadratic program''.  In general a quadratic program has the structure
\begin{equation}\label{eqn:L1introduction:200}
F(a, b) = (\cdots) a^2 + (\cdots) a b + (\cdots) b^2.
\end{equation}
\paragraph{Approach 2:  Linear algebraic formulation.}
\begin{equation}\label{eqn:L1introduction:220}
\begin{bmatrix}
y_1 \\
\vdots \\
y_n
\end{bmatrix}
=
\begin{bmatrix}
x_1 & 1 \\
\vdots & \vdots \\
x_n & 1
\end{bmatrix}
\begin{bmatrix}
a \\
b
\end{bmatrix}
+
\begin{bmatrix}
z_1 \\
\vdots \\
z_n
\end{bmatrix}
,
\end{equation}
or
\begin{equation}\label{eqn:L1introduction:240}
\By = H \Bv + \Bz,
\end{equation}
where \( \Bz \) is the error vector.  The problem is now reduced to : Fit \( \By \) to be as close to \( H \Bv + \Bz \) as possible, or to minimize the norm of the error vector, or
\begin{equation}\label{eqn:L1introduction:260}
\begin{aligned}
\min_\Bv \Norm{ \By - H \Bv }^2_2 &= \min_\Bv \lr{ \By - H \Bv }^\T \lr{ \By - H \Bv } \\
&= \min_\Bv \lr{ \By^\T \By - \By^\T H \Bv - \Bv^\T H \By + \Bv^\T H^\T H \Bv } \\
&= \min_\Bv \lr{ \By^\T \By - 2 \By^\T H \Bv + \Bv^\T H^\T H \Bv }.
\end{aligned}
\end{equation}
%
It is now possible to take the derivative with respect to the \( \Bv \) vector (i.e. the gradient with respect to the coordinates of the constraint vector)
\begin{equation}\label{eqn:L1introduction:280}
\begin{aligned}
\PD{\Bv}{}
\lr{ \By^\T \By - 2 \By^\T H \Bv + \Bv^\T H^\T H \Bv }
&= - 2 \By^\T H + 2 \Bv^\T H^\T H \\
&= 0,
\end{aligned}
\end{equation}
or
\begin{equation}\label{eqn:L1introduction:300}
(H^\T H) \Bv = H^\T \By,
\end{equation}
so, assuming that \( H^\T H \) is invertible, the optimization problem has solution
\begin{equation}\label{eqn:L1introduction:320}
\Bv^\conj = (H^\T H)^{-1} H^\T \By,
\end{equation}
where
\begin{equation}\label{eqn:L1introduction:340}
\begin{aligned}
H^\T H
&=
\begin{bmatrix}
x_1 & \cdots & x_n \\
 1  & \cdots & 1   \\
\end{bmatrix}
\begin{bmatrix}
x_1 & 1 \\
\vdots & \vdots \\
x_n & 1
\end{bmatrix} \\
&=
\begin{bmatrix}
\sum x_i^2 & \sum x_i \\
\sum x_i & n
\end{bmatrix}
,
\end{aligned}
\end{equation}
as seen in the calculus approach.
\section{Maximum Likelyhood Estimation (MLE).}
It is reasonable to ask why the 2-norm was picked for the objective function?
\begin{itemize}
\item One justification is practical: Because we can solve the derivative equation.
\item Another justification: In statistics the error vector \( \Bz = \By - H \Bv \) can be modeled as an IID (Independently and Identically Distributed) Gaussian random variable (i.e. noise).  Under this model, the use of the 2-norm can be viewed as a consequence of such an ML estimation problem (see \citep{boyd2004convex} ch. 7).
\end{itemize}

A Gaussian \cref{fig:Gaussian:GaussianFig4} IID model is given by
\begin{subequations}
\label{eqn:L1introduction:800}
\begin{equation}\label{eqn:L1introduction:360}
y_i = a x_i + b
\end{equation}
\begin{equation}\label{eqn:L1introduction:380}
z_i = y_i - a x_i -b \sim N(O, O^2)
\end{equation}
\begin{equation}\label{eqn:L1introduction:400}
P_Z(z) = \inv{\sqrt{2 \pi \sigma}} \exp\lr{ -\inv{2} z^2/\sigma^2 }.
\end{equation}
\end{subequations}
\imageFigure{../figures/ece1505-convex-optimization/GaussianFig4}{Gaussian probability distribution.}{fig:Gaussian:GaussianFig4}{0.2}
\paragraph{MLE: Maximum Likelyhood Estimator}
Pick \( (a,b) \) to maximize the probability of observed data.
\begin{equation}\label{eqn:L1introduction:420}
\begin{aligned}
(a^\conj, b^\conj)
&= \arg \max P( x, y ; a, b ) \\
&= \arg \max P_Z( y - (a x + b) ) \\
&= \arg \max \prod_{i = 1}^n \\
&= \arg \max \inv{\sqrt{2 \pi \sigma}} \exp\lr{ -\inv{2} (y_i - a x_i - b)^2/\sigma^2 }.
\end{aligned}
\end{equation}
%
Taking logs gives
\begin{equation}\label{eqn:L1introduction:440}
\begin{aligned}
(a^\conj, b^\conj)
&= \arg \max
\lr{
\textrm{constant}
   -\inv{2} \sum_i (y_i - a x_i - b)^2/\sigma^2
} \\
&= \arg \min \inv{2} \sum_i (y_i - a x_i - b)^2/\sigma^2 \\
&= \arg \min \sum_i (y_i - a x_i - b)^2/\sigma^2
\end{aligned}
\end{equation}
%
Here \( \arg \max \) is not the maximum of the function, but the value of the parameter (the argument) that maximizes the function.
\paragraph{Double sides exponential noise}
A double sided exponential distribution is plotted in \cref{fig:DoubleSidedExponentialDist:DoubleSidedExponentialDistFig5}, and has the mathematical form
%
\begin{equation}\label{eqn:L1introduction:460}
P_Z(z) = \inv{2 c} \exp\lr{ -\inv{c} \Abs{z} }.
\end{equation}
\imageFigure{../figures/ece1505-convex-optimization/DoubleSidedExponentialDistFig5}{Double sided exponential probability distribution.}{fig:DoubleSidedExponentialDist:DoubleSidedExponentialDistFig5}{0.2}

The optimization problem is
\begin{equation}\label{eqn:L1introduction:480}
\begin{aligned}
\max_{a,b} \prod_{i = 1}^n P_z(z_i)
&= \max_{a,b} \prod_{i = 1}^n \inv{2 c} \exp\lr{ -\inv{c} \Abs{z_i} } \\
&= \max_{a,b} \prod_{i = 1}^n \inv{2 c} \exp\lr{ -\inv{c} \Abs{y_i - a x_i - b} } \\
&= \max_{a,b} \lr{\inv{2 c}}^n \exp\lr{ -\inv{c} \sum_{i=1}^n \Abs{y_i - a x_i - b} }.
\end{aligned}
\end{equation}
%
This is a L1 norm problem
\begin{equation}\label{eqn:L1introduction:500}
\min_{a,b} \sum_{i = 1}^n \Abs{ y_i - a x_i - b }.
\end{equation}
%
i.e.
%
\begin{equation}\label{eqn:L1introduction:520}
\min_\Bv \Norm{ \By - H \Bv }_1.
\end{equation}
%
This is still convex, but has no analytic solution, and is an example of a linear program.
\subsection{Solution of linear program}
Introduce helper variables \( t_1, \cdots, t_n \), and
minimize \( \sum_i t_i \), such that
\begin{equation}\label{eqn:L1introduction:540}
\Abs{ y_i - a x_i - b } \le t_i.
\end{equation}
This is now an optimization problem for \( a, b, t_1, \cdots t_n \).  A linear program is defined as
\begin{equation}\label{eqn:L1introduction:560}
\min_{a, b, t_1, \cdots t_n} \sum_i t_i
\end{equation}
such that
\begin{equation}\label{eqn:L1introduction:580}
\begin{aligned}
y_i - a x_i - b \le t_i
y_i - a x_i - b \ge -t_i
\end{aligned}
\end{equation}
\paragraph{Single sided exponential}
What if your noise doesn't look double sided, with only noise for values \( x > 0 \).  Can define a single sided probability distribution, as that of \cref{fig:SingleSidedExponentialDist:SingleSidedExponentialDistFig6}.
\imageFigure{../figures/ece1505-convex-optimization/SingleSidedExponentialDistFig6}{Single sided exponential distribution.}{fig:SingleSidedExponentialDist:SingleSidedExponentialDistFig6}{0.2}
%
\begin{equation}\label{eqn:L1introduction:600}
P_Z(z) =
\left\{
\begin{array}{l l}
\inv{c} e^{-z/c} & \quad \mbox{\( z \ge 0 \)} \\
0 & \quad \mbox{\( z < 0 \)}
\end{array}
\right.
\end{equation}
%
i.e. all \( z_i \) error values are always non-negative.
%
\begin{equation}\label{eqn:L1introduction:620}
\log P_z(z) =
\left\{
\begin{array}{l l}
\textrm{const} - z/c & \quad \mbox{\( z > 0\)} \\
-\infty & \quad \mbox{\( z< 0\)}
\end{array}
\right.
\end{equation}
%
Problem becomes
\begin{equation}\label{eqn:L1introduction:640}
\min_{a, b} \sum_i \lr{ y_i - a x_i - b },
\end{equation}
such that
\begin{equation}\label{eqn:L1introduction:660}
y_i - a x_i - b \ge t_i \qquad \forall i
\end{equation}
\paragraph{Uniform noise}
For noise that is uniformly distributed in a range, as that of \cref{fig:stepProbabilityDist:stepProbabilityDistFig7}, which is constant in the range \( [-c,c] \) and zero outside that range.
\imageFigure{../figures/ece1505-convex-optimization/stepProbabilityDistFig7}{Uniform probability distribution.}{fig:stepProbabilityDist:stepProbabilityDistFig7}{0.2}
\begin{equation}\label{eqn:L1introduction:680}
P_Z(z) =
\left\{
\begin{array}{l l}
\inv{2 c} & \quad \mbox{\( \Abs{z} \le c\)} \\
0 & \quad \mbox{\( \Abs{z} > c. \)}
\end{array}
\right.
\end{equation}
or
\begin{equation}\label{eqn:L1introduction:700}
\log P_Z(z) =
\left\{
\begin{array}{l l}
\textrm{const} & \quad \mbox{\( \Abs{z} \le c\)} \\
-\infty & \quad \mbox{\( \Abs{z} > c. \)}
\end{array}
\right.
\end{equation}
%
MLE solution
\begin{equation}\label{eqn:L1introduction:720}
\max_{a,b} \prod_{i = 1}^n P(x, y; a, b)
=
\max_{a,b} \sum_{i = 1}^n \log P_Z( y_i - a x_i - b ).
\end{equation}
%
Here the argument is constant if \( -c \le y_i - a x_i - b \le c \), so an ML solution is \underline{any} \( (a,b) \) such that
\begin{equation}\label{eqn:L1introduction:740}
\Abs{ y_i - a x_i - b } \le c \qquad \forall i \in 1, \cdots, n.
\end{equation}
%
This is a linear program known as a ``feasibility problem''.
\begin{equation}\label{eqn:L1introduction:760}
\min d
\end{equation}
such that
\begin{equation}\label{eqn:L1introduction:780}
\begin{aligned}
y_i - a x_i - b &\le d \\
y_i - a x_i - b &\ge -d
\end{aligned}
\end{equation}
%
If \( d^\conj \le c \), then the problem is feasible, however, if \( d^\conj > c \) it is infeasible.
\subsection{Method comparison}
The double sided exponential, single sided exponential and uniform probability distributions of \cref{fig:threeDistributions} each respectively represent the point plots of the form \cref{fig:threeDistributionsSamples}.  The double sided exponential samples are distributed on both sides of the line, the single sided strictly above or on the line, and the uniform representing error bars distributed around the line of best fit.

\imageThreeFiguresOneLine
{../figures/ece1505-convex-optimization/DoubleSidedExponentialDistFig5}
{../figures/ece1505-convex-optimization/SingleSidedExponentialDistFig6}
{../figures/ece1505-convex-optimization/stepProbabilityDistFig7}
{Distributions}{fig:threeDistributions}{scale=0.3}

\imageThreeFiguresOneLine
{../figures/ece1505-convex-optimization/LinearFitFig3}
{../figures/ece1505-convex-optimization/LinearFitGtFig3}
{../figures/ece1505-convex-optimization/LinearFitErrorBarFig3}
{Samples}{fig:threeDistributionsSamples}{scale=0.3}

%      \section{Problems}
   \mychapter{Mathematical background.}
      %
% Copyright � 2016 Peeter Joot.  All Rights Reserved.
% Licenced as described in the file LICENSE under the root directory of this GIT repository.
%
\input{../latex/blogpost.tex}
\renewcommand{\basename}{convex-optimization2}
\renewcommand{\dirname}{notes/ece1505/}
\newcommand{\keywords}{ECE1505H}
\input{../latex/peeter_prologue_print2.tex}

\usepackage{ece1505}
\usepackage{peeters_braket}
%\usepackage{peeters_layout_exercise}
\usepackage{peeters_figures}
\usepackage{mathtools}
\usepackage{siunitx}
\usepackage{peeters_layout_exercise}

\beginArtNoToc
\generatetitle{ECE1505H Convex Optimization.  Lecture 2: XXX.  Taught by Prof.\ Stark Draper}
%\chapter{XXX}
\label{chap:convex-optimization2}

\paragraph{Disclaimer}

Peeter's lecture notes from class.  These may be incoherent and rough.

These are notes for the UofT course ECE1505H, Convex Optimization, taught by Prof. Stark Draper, covering \textchapref{{1}} \citep{boyd2004convex} content.

\section{Mathematical background}

\begin{itemize}
\item Calculus: Derivatives and Jacobians, Gradients, Hessians, approximation functions.
\item Linear algebra, Matrices, decompositions, ...
\end{itemize}

\subsection{Norms}
\paragraph{Vector space}

\makedefinition{Vector space}{dfn:convex-optimizationLecture2:1}{
A set of elements (vectors) that is closed under vector addition and scaling.
} % definition

\makedefinition{Normed vector spaces}{dfn:convex-optimizationLecture2:2}{
A vector space with a notion of lenght of any signle vector, the ``norm''.
} % definition

\makedefinition{Inner product space.}{dfn:convex-optimizationLecture2:3}{
A normed vector space with a notion of a real angle vetween any pair of vectors.
} % definition

This course has a focus on optimization in \R{n}.  Complex spaces in the context of this course can be considered with a mapping \( \text{\C{n}} \rightarrow \Rm{2 n} \).

\makedefinition{Norm.}{dfn:convex-optimizationLecture2:4}{
A norm is a function operating on a vector

\begin{dmath*}
\Bx = ( x_1, x_2, \cdots, x_n )
\end{dmath*}

that provides a mapping

\begin{equation*}
\Norm{ \cdot } : \Rm{n} \rightarrow \bbR,
\end{equation*}

where

\begin{itemize}
\item \( \Norm{ \Bx } \ge 0 \)
\item \( \Norm{ \Bx } = 0 \qquad \iff \Bx = 0 \)
\item \( \Norm{ t \Bx } = \Abs{t} \Norm{ \Bx } \)
\item \( \Norm{ \Bx + \By } \le \Norm{ \Bx } + \Norm{\By} \).  This is the triangle inequality.
\end{itemize}
} % definition

\paragraph{Example: Euclidean norm}

\begin{dmath}\label{eqn:convex-optimizationLecture2:24}
\Norm{\Bx} = \sqrt{ \sum_{i = 1}^n x_i^2 }
\end{dmath}

\paragraph{Example: \(l_p\)-norms}

\begin{dmath}\label{eqn:convex-optimizationLecture2:44}
\Norm{\Bx}_p = \lr{ \sum_{i = 1}^n \Abs{x_i}^p }^{1/p}.
\end{dmath}

For \( p = 1 \), this is

\begin{dmath}\label{eqn:convex-optimizationLecture2:64}
\Norm{\Bx}_1 = \sum_{i = 1}^n \Abs{x_i},
\end{dmath}

For \( p = 2 \), this is the Euclidean norm \cref{eqn:convex-optimizationLecture2:24}.
For \( p = \infty \), this is

\begin{dmath}\label{eqn:convex-optimizationLecture2:324}
\Norm{\Bx}_\infty = \max_{i = 1}^n \Abs{x_i}.
\end{dmath}

\makedefinition{Unit ball}{dfn:convex-optimizationLecture2:10}{

\begin{dmath*}
\setlr{ \Bx | \Norm{\Bx} \le 1 }
\end{dmath*}

} % definition

The \( l_2 \) norm is not only familiar, but can be ``induced'' by an inner product

\begin{equation}\label{eqn:convex-optimizationLecture2:84}
\innerproduct{\Bx}{\By} = \Bx^\T \By = \sum_{i = 1}^n x_i y_i,
\end{equation}

which is not true for all norms.  The norm induced by this inner product is

\begin{dmath}\label{eqn:convex-optimizationLecture2:104}
\Norm{\Bx}_2 = \sqrt{ \innerproduct{\Bx}{\By} }
\end{dmath}

Inner product spaces have a notion of angle given by

\begin{dmath}\label{eqn:convex-optimizationLecture2:124}
\innerproduct{\Bx}{\By} = \Norm{\Bx} \Norm{\By} \cos \theta,
\end{dmath}

F3

and always satify the Cauchy-Schwartz inequality

\begin{dmath}\label{eqn:convex-optimizationLecture2:144}
\innerproduct{\Bx}{\By} \le \Norm{\Bx}_2 \Norm{\By}_2.
\end{dmath}

In an inner product space we say \( \Bx \) and \( \By \) are orthogonal vectors \( \Bx \perp \By \) if
\( \innerproduct{\Bx}{\By} = 0 \).

F4

\makedefinition{Dual norm}{dfn:convex-optimizationLecture2:20}{
Let \( \Norm{ \cdot } \) be a norm in \R{n}.  The ``dual'' norm \( \Norm{ \cdot }_\conj \) is defined as
\begin{equation*}
\Norm{\Bz}_\conj = \sup_\Bx \setlr{ \Bz^\T \Bx | \Norm{\Bx} \le 1 }.
\end{equation*}

where \( \sup \) is roughly the ``least upper bound''.
\index{sup}

This is a limit over the unit ball of \( \Norm{\cdot} \).
} % definition

\paragraph{\( l_2 \) dual}.

Dual of the \( l_2 \) is the \( l_2 \) norm.

F5

Proof:

\begin{dmath}\label{eqn:convex-optimizationLecture2:164}
\Norm{\Bz}_\conj
= \sup_\Bx \setlr{ \Bz^\T \Bx | \Norm{\Bx}_2 \le 1 }
= \sup_\Bx \setlr{ \Norm{\Bz}_2 \Norm{\Bx}_2 \cos\theta | \Norm{\Bx}_2 \le 1 }
\le \sup_\Bx \setlr{ \Norm{\Bz}_2 \Norm{\Bx}_2 | \Norm{\Bx}_2 \le 1 }
\le
\Norm{\cancel{\Bz}}_2
\Norm{
\frac{\Bz}{ \cancel{\Norm{\Bz}_2} }
}_2
=
\Norm{\Bz}_2.
\end{dmath}

\paragraph{\( l_1 \) dual}.
For \( l_1 \), the dual is the \( l_\infty \) norm.  Proof:

\begin{dmath}\label{eqn:convex-optimizationLecture2:184}
\Norm{\Bz}_\conj
=
\sup_\Bx \setlr{ \Bz^\T \Bx | \Norm{\Bx}_1 \le 1 },
\end{dmath}

but
\begin{dmath}\label{eqn:convex-optimizationLecture2:204}
\Bz^\T \Bx
=
\sum_{i=1}^n z_i x_i \le
\Abs{
\sum_{i=1}^n z_i x_i
}
\le
\sum_{i=1}^n \Abs{z_i x_i },
\end{dmath}

so
\begin{dmath}\label{eqn:convex-optimizationLecture2:224}
\Norm{\Bz}_\conj
=
\sum_{i=1}^n \Abs{z_i}\Abs{ x_i }
\le \lr{ \max_{j=1}^n \Abs{z_j} }
\sum_{i=1}^n \Abs{ x_i }
\le \lr{ \max_{j=1}^n \Abs{z_j} }
=
\Norm{\Bz}_\infty.
\end{dmath}

F6

\paragraph{\( l_\infty \) dual}.

F7
\begin{dmath}\label{eqn:convex-optimizationLecture2:244}
\Norm{\Bz}_\conj
=
\sup_\Bx \setlr{ \Bz^\T \Bx | \Norm{\Bx}_\infty \le 1 }.
\end{dmath}

Here
\begin{dmath}\label{eqn:convex-optimizationLecture2:264}
\Bz^\T \Bx
=
\sum_{i=1}^n z_i x_i
\le
\sum_{i=1}^n \Abs{z_i}\Abs{ x_i }
\le
\lr{ \max_j \Abs{ x_j } }
\sum_{i=1}^n \Abs{z_i}
=
\Norm{\Bx}_\infty
\sum_{i=1}^n \Abs{z_i}.
\end{dmath}

So
\begin{dmath}\label{eqn:convex-optimizationLecture2:284}
\Norm{\Bz}_\conj
\le
\sum_{i=1}^n \Abs{z_i}
=
\Norm{\Bz}_1.
\end{dmath}

\begin{dmath}\label{eqn:convex-optimizationLecture2:304}
x_i^\conj
=
\left\{
\begin{array}{l l}
+1 & \quad \mbox{\( z_i \ge 0 \)} \\
-1 & \quad \mbox{\( z_i \le 0 \)}
\end{array}
\right.
\end{dmath}

\subsection{Calculus}

\subsubsection{Gradient}

Consider a scalar function

\begin{equation}\label{eqn:convex-optimizationLecture2:344}
F: \Rm{n} \rightarrow \bbR
\end{equation}

where for \( \Bx \in \Rm{n} \)

\begin{dmath}\label{eqn:convex-optimizationLecture2:364}
\spacegrad F(\Bx)
=
\begin{bmatrix}
\PD{x_1}{F(\Bx)} \\
\PD{x_2}{F(\Bx)} \\
\vdots \\
\PD{x_n}{F(\Bx)}
\end{bmatrix}
\end{dmath}

Example:

\begin{dmath}\label{eqn:convex-optimizationLecture2:384}
F(\Bx)
= \Bx^\T P \Bx
=
\begin{bmatrix}
x_1 & x_2 & \cdots & x_n
\end{bmatrix}
\begin{bmatrix}
P_{11} & P_{12} & \cdots & P_{1n} \\
P_{21} & P_{22} & \cdots & P_{2n} \\
\cdots
P_{n1} & P_{n2} & \cdots & P_{nn} \\
\end{bmatrix}
\begin{bmatrix}
x_1 \\ x_2 \\ \vdots \\ x_n
\end{bmatrix}
=
\begin{bmatrix}
x_1 & x_2 & \cdots & x_n
\end{bmatrix}
\lr{
x_1
\begin{bmatrix}
P_{11} \\
P_{21} \\
\vdots \\
P_{n1} \\
\end{bmatrix}
+
x_2
\begin{bmatrix}
P_{12} \\
P_{22} \\
\vdots \\
P_{n2} \\
\end{bmatrix}
+
\cdots
}
=
\sum x_j
\begin{bmatrix}
x_1 & x_2 & \cdots & x_n
\end{bmatrix}
\begin{bmatrix}
P_{1j} \\
P_{2j} \\
\vdots \\
P_{nj} \\
\end{bmatrix}
=
\sum_j x_j \sum_i x_i P_{ij}
=
\sum_{i,j = 1}^n x_i x_j P_{ij},
\end{dmath}

We want to show that
\begin{dmath}\label{eqn:convex-optimizationLecture2:404}
\spacegrad F(\Bx) = \lr{P + P^\T } \Bx.
\end{dmath}

Consider the k-th derivative

\begin{dmath}\label{eqn:convex-optimizationLecture2:424}
\PD{x_k}{} F(\Bx)
=
\PD{x_k}{}
\lr{
P_{kk} x_k^2
+
\sum_{i \ne k} x_i x_k \lr{ P_{ik} + P_{ki} }
}
=
2 P_{kk} x_k + 2
\sum_{i \ne k} x_i \frac{\lr{ P_{ik} + P_{ki} }}{2}
=
\sum_{i}^n x_i \frac{\lr{ P_{ik} + P_{ki} }}{2}
=
\sum_{i}^n \lr{ P_{ik} + P_{ki} } x_i
,
\end{dmath}

which proves \cref{eqn:convex-optimizationLecture2:404}.

\paragraph{Symmetric matrices}

Let \( S \) be the set of symmetric matrices

\begin{equation}\label{eqn:convex-optimizationLecture2:444}
S = \setlr{ P \in \Rm{n\times n} | P = P^\T},
\end{equation}

then
%In particular, for a symmetric matrix \( P \), this means

\begin{dmath}\label{eqn:convex-optimizationLecture2:464}
\spacegrad \Bx^\T P \Bx = 2 P \Bx.
\end{dmath}

\paragraph{Gradient}

The gradient provides a linear approximation of a function about a point \( \Bx_0 \in \Rm{n} \).

\begin{dmath}\label{eqn:convex-optimizationLecture2:484}
F(\Bx)
\approx F(\Bx_0) + \spacegrad F(\Bx_0)^\T \lr{ \Bx - \Bx_0 }.
=
F(\Bx_0) + \innerproduct{ \spacegrad F(\Bx_0)}{ \Bx - \Bx_0 },
\end{dmath}

or
\begin{equation}\label{eqn:convex-optimizationLecture2:504}
F(\Bx + \Delta \Bx)
=
F(\Bx) + \innerproduct{ \spacegrad F(\Bx)}{ \Delta \Bx }.
\end{equation}

This can be thought of as the definition of the gradient in an inner product space.  It will be possible to find the structure of the gradient by considering a pertubation of a function about a point.

\subsubsection{Chain rule}

Gradients for compositions of functions.

\paragraph{Example 1:}

\begin{dmath}\label{eqn:convex-optimizationLecture2:524}
\begin{aligned}
F &: \Rm{n} \rightarrow \bbR \\
g &: \bbR \rightarrow \bbR,
\end{aligned}
\end{dmath}

and let

\begin{dmath}\label{eqn:convex-optimizationLecture2:544}
h(\Bx) = g(F(\Bx)),
\end{dmath}

for \( \Bx \in \Rm{n} \), then

\begin{dmath}\label{eqn:convex-optimizationLecture2:564}
\spacegrad h(\Bx)
=
g'(F(\Bx)) \spacegrad F(\Bx).
\end{dmath}

\paragraph{Example 2:}

\begin{dmath}\label{eqn:convex-optimizationLecture2:584}
\begin{aligned}
F &: \Rm{n} \rightarrow \Rm{n} \\
g &: \Rm{n} \rightarrow \bbR,
\end{aligned}
\end{dmath}

and let

\begin{dmath}\label{eqn:convex-optimizationLecture2:604}
h(\Bx)
= g(F(\Bx))
= g\lr{
\begin{bmatrix}
F_1(\Bx) \\
F_2(\Bx) \\
\vdots \\
F_n(\Bx) \\
\end{bmatrix}
}
\end{dmath}

for \( \Bx \in \Rm{n} \), then

\begin{dmath}\label{eqn:convex-optimizationLecture2:624}
\PD{x_k}{h(\Bx)}
=
\PD{F_1}{g}
\PD{x_k}{F_1}
+
\PD{F_2}{g}
\PD{x_k}{F_2}
+
\cdots
\end{dmath}

Let

\begin{dmath}\label{eqn:convex-optimizationLecture2:644}
D F(\Bx)
=
\begin{bmatrix}
\PD{x_1}{F_1} & \PD{x_2}{F_1} & \PD{x_n}{F_1} \\
\PD{x_1}{F_2} & \PD{x_2}{F_2} & \PD{x_n}{F_2} \\
\vdots \\
\PD{x_1}{F_n} & \PD{x_2}{F_n} & \PD{x_n}{F_n} \\
\end{bmatrix}
\end{dmath}

so

\begin{dmath}\label{eqn:convex-optimizationLecture2:664}
\spacegrad h(\Bx)
=
\lr{ D F(\Bx) }^\T
\spacegrad g(F(\Bx)).
\end{dmath}

\paragraph{In general}

\begin{dmath}\label{eqn:convex-optimizationLecture2:684}
\begin{aligned}
F &: \Rm{n} \rightarrow \Rm{m} \\
g &: \Rm{m} \rightarrow \Rm{p},
\end{aligned}
\end{dmath}

With \( h(\Bx) = g(F(\Bx) \),

\begin{dmath}\label{eqn:convex-optimizationLecture2:704}
D h(\Bx) = D g(F(\Bx)) D F(\Bx),
\end{dmath}

and
\begin{dmath}\label{eqn:convex-optimizationLecture2:724}
\spacegrad h(\Bx) =
\lr{
D F(\Bx)
}^\T
\spacegrad g(F(\Bx))
\end{dmath}

Note: \( \spacegrad h(\Bx) = D h(\Bx)^\T \) if \( p = 1 \).

EXERSIZE: work this out for a general non-square case such as \( n = 4, m = 3, p = 2 \).

\paragraph{Affine functions}

An important example are affine functions of \( \Bx \)

\begin{dmath}\label{eqn:convex-optimizationLecture2:744}
\begin{aligned}
F &: \Rm{n} \rightarrow \Rm{n} \\
g &: \Rm{n} \rightarrow \bbR,
\end{aligned}
\end{dmath}

\begin{dmath}\label{eqn:convex-optimizationLecture2:764}
F(\Bx) = A \Bx + \Bb,
\end{dmath}

where \( A \) is an \( n \times n \) matrix and \( \Bb \) is an \( n \times 1 \) column vector.

Given a function

\begin{equation}\label{eqn:convex-optimizationLecture2:784}
h(\Bx) = g(F(\Bx)) = g(A \Bx + \Bb).
\end{equation}

\begin{dmath}\label{eqn:convex-optimizationLecture2:804}
F(\Bx)
= A \Bx + \Bb
=
\begin{bmatrix}
a_1^\T \\
a_2^\T \\
\vdots \\
a_n^\T \\
\end{bmatrix}
\Bx
+
\Bb
=
\begin{bmatrix}
\innerproduct{a_1}{\Bx} \\
\innerproduct{a_2}{\Bx} \\
\vdots \\
\innerproduct{a_n}{\Bx}
\end{bmatrix}
+
\Bb
=
\begin{bmatrix}
\sum_{i = 1}^n a_{1i} x_i \\
\sum_{i = 1}^n a_{2i} x_i \\
\vdots \\
\sum_{i = 1}^n a_{ni} x_i
\vdots \\
\end{bmatrix}
+
\Bb
=
\begin{bmatrix}
F_1(\Bx) \\
F_2(\Bx) \\
\vdots \\
F_n(\Bx) \\
\end{bmatrix}
+
\Bb,
\end{dmath}

so

\begin{dmath}\label{eqn:convex-optimizationLecture2:824}
D F(\Bx) = A.
\end{dmath}

\subsubsection{Second derivative.}

The second derivative for \( \Rm{n} \rightarrow \bbR \), the ``Hessian'' is

\begin{dmath}\label{eqn:convex-optimizationLecture2:844}
\spacegrad^2 F =
\begin{bmatrix}
\frac{\partial^2 F}{\partial x_i \partial x_j }
\end{bmatrix}
=
\begin{bmatrix}
\frac{\partial^2 F}{\partial x_1^2 } & \frac{\partial^2 F}{\partial x_1 \partial x_2} & \cdots & \frac{\partial^2 F}{\partial x_1 \partial x_n} \\
\frac{\partial^2 F}{\partial x_2 \partial x_1 } & \frac{\partial^2 F}{\partial x_2^2 } & \cdots & \frac{\partial^2 F}{\partial x_2 \partial x_n} \\
\vdots \\
\end{bmatrix}
\end{dmath}

\paragraph{Example}

Given

\begin{dmath}\label{eqn:convex-optimizationLecture2:864}
F(\Bx)
= \inv{2} \Bx^\T P \Bx
+ \Bq^\T \Bx
+ \Bc,
\end{dmath}

where \( P = P^\T \), then

\begin{dmath}\label{eqn:convex-optimizationLecture2:884}
\spacegrad F
=
\inv{2} \lr{ P + P^\T } \Bx + \Bq
=
P \Bx + \Bq,
\end{dmath}

and

\begin{dmath}\label{eqn:convex-optimizationLecture2:904}
\spacegrad^2 F  P.
\end{dmath}

\EndArticle
%\EndNoBibArticle

      \section{Problems.}
         %
% Copyright � 2017 Peeter Joot.  All Rights Reserved.
% Licenced as described in the file LICENSE under the root directory of this GIT repository.
%
\makeproblem{Taylor series expansion}{convex-optimization:problemSet1:1}{
Consider the function
%
\begin{equation}\label{eqn:ProblemSet1Problem1:20}
f(\Bx) = -\sum_{l=1}^m \log( b_l - \Ba_l^\T \Bx ),
\end{equation}
%
where \( \Bx \in \Rm{n} \), \( b_l \in \bbR \) and \( \Ba_l \in \Rm{n}\).
Compute \( \spacegrad f(\Bx) \) and \( \spacegrad^2 f(\Bx)\).
Write down the first three terms of the Taylor series expansion of \(f(\Bx)\) around some \(\Bx_0\).
} % makeproblem

\makeanswer{convex-optimization:problemSet1:1}{
\withproblemsetsParagraph{

Application of the chain rule, with
%
\begin{equation}\label{eqn:ProblemSet1Problem1:40}
g_l(\Bx) = b_l - \Ba_l^\T \Bx,
\end{equation}
%
gives
%
\begin{equation}\label{eqn:ProblemSet1Problem1:60}
\begin{aligned}
\spacegrad f(\Bx)
&= - \sum_{l=1}^m \spacegrad \log(g_l(\Bx)) \\
&= -\sum_{l=1}^m \inv{g_l(\Bx)} \spacegrad g_l(\Bx) \\
&= -\sum_{l=1}^m \inv{b_l - \Ba_l^\T \Bx} \spacegrad (b_l - \Ba_l^\T \Bx) \\
&= -\sum_{l=1}^m \inv{b_l - \Ba_l^\T \Bx} (- \Ba_l),
\end{aligned}
\end{equation}
%
or
%\begin{equation}\label{eqn:ProblemSet1Problem1:80}
\boxedEquation{eqn:ProblemSet1Problem1:80}{
\spacegrad f
=
\sum_{l=1}^m
\frac{\Ba_l}{b_l - \Ba_l^\T \Bx}.
}
%\end{equation}
%
The Hessian components are
%
\begin{equation}\label{eqn:ProblemSet1Problem1:100}
\begin{aligned}
(\spacegrad^2 f)_{ij}
&= \frac{\partial^2 f}{\partial x_i \partial x_j} \\
&= \frac{\partial}{\partial x_i} \Be_j^\T \spacegrad f \\
&= \frac{\partial}{\partial x_i} \sum_{l=1}^m \frac{a_{lj}}{b_l - \Ba_l^\T \Bx} \\
&= \sum_{l=1}^m -\frac{a_{lj}(-a_{li})}{(b_l - \Ba_l^\T \Bx)^2},
\end{aligned}
\end{equation}
%
or

\boxedEquation{eqn:ProblemSet1Problem1:120}{
\spacegrad^2 f
=
\sum_{l=1}^m
\inv{(b_l - \Ba_l^\T \Bx)^2}
{\begin{bmatrix}
a_{li}
a_{lj}
\end{bmatrix}}_{ij}.
}

For the Taylor expansion, note that
%
\begin{equation}\label{eqn:ProblemSet1Problem1:140}
\begin{aligned}
(\Delta \Bx)^\T
\spacegrad^2 f
\Delta \Bx
&=
\sum_{l=1}^m
\inv{(b_l - \Ba_l^\T \Bx)^2}
(\Delta x_j) a_{lj} a_{li} (\Delta x_i) \\
&=
\sum_{l=1}^m
\inv{(b_l - \Ba_l^\T \Bx)^2}
(\Ba_l^\T \Delta x)^2,
\end{aligned}
\end{equation}
%
so
\begin{equation}\label{eqn:ProblemSet1Problem1:160}
f(\Bx_0 + \Delta \Bx)
=
f(\Bx_0)
+ (\spacegrad f)^\T (\Delta \Bx)
+ \inv{2}
(\Delta \Bx)^\T
(\spacegrad^2 f)
(\Delta \Bx),
\end{equation}
%
which is
\boxedEquation{eqn:ProblemSet1Problem1:180}{
f(\Bx_0 + \Delta \Bx)
=
\sum_{l=1}^m
\lr{
-\log \lr{ b_l - \Ba_l^\T \Bx_0 }
+
\frac{\Ba_l^\T \Delta \Bx}{
b_l - \Ba_l^\T \Bx_0
}
+
\inv{2}
\frac{(\Ba_l^\T \Delta \Bx)^2}{
(b_l - \Ba_l^\T \Bx_0)^2
}.
}
}

} % redaction
} % answer
 % Taylor series expansion
         \shipoutAnswer
         %
% Copyright � 2017 Peeter Joot.  All Rights Reserved.
% Licenced as described in the file LICENSE under the root directory of this GIT repository.
%
\makeproblem{Inversion formula for ``small'' matrices}{convex-optimization:problemSet1:2}{
Prove the relation

\begin{dmath}\label{eqn:ProblemSet1Problem2:20}
(I + A)^{-1} = I - A,
\end{dmath}

for \(A\) ``small''.  We used this in class to derive the second order expansion of

\begin{dmath}\label{eqn:ProblemSet1Problem2:40}
\log \det(I + A).
\end{dmath}

Prove this result in two ways:
\makesubproblem{}{convex-optimization:problemSet1:2a}
First, prove this for the special case of \( A \in S^n_{++} \) where the eigenvalues are small.
This is what we needed in class. Use a decomposition of \( A \) and Taylor approximation of the eigenvalues.

\makesubproblem{}{convex-optimization:problemSet1:2b}
Next prove the general relation: If \( A \in \Rm{n \times n} \) and \( \Norm{A}_p < 1 \) then \( I - A \) is non-singular, and

\begin{dmath}\label{eqn:ProblemSet1Problem2:60}
\lr{ I - A }^{-1} = \sum_{k=0}^\infty A^k
\end{dmath}

where

\begin{dmath}\label{eqn:ProblemSet1Problem2:80}
\Norm{(I - A)^{-1}}_p \le \inv{1 - \Norm{A}_p}.
\end{dmath}

The p-th matrix norm \( \Norm{A}_p \) is defined in terms of the vector p-norm as

\begin{dmath}\label{eqn:ProblemSet1Problem2:100}
\Norm{A}_p = \sup_{\Bx \ne 0} \frac{\Norm{A \Bx }_p}{\Norm{\Bx}_p}
\end{dmath}

which, using the scaling property of a norm, can be seen to be equivalent to

\begin{dmath}\label{eqn:ProblemSet1Problem2:120}
\Norm{A}_p = \max_{\Norm{\Bx}_p = 1} \Norm{A \Bx}_p.
\end{dmath}

In our derivation in class we used only the zeroth and first-order terms of the expansion.

Some hints that outline one approach to the above result:

\begin{enumerate}[(i)]
\item
One approach to proving the first statement (about non-singularity) is by contradiction: note
that if \( I - A \) is singular then there exists a vector \( \Bv \) such that \((I - A) \Bv = 0 \) and work from
there.
\item Next, consider the telescoping sum

\begin{dmath}\label{eqn:ProblemSet1Problem2:140}
\sum_{k=0}^N A^k (I - A) = I - A^{N+1},
\end{dmath}

and show
\begin{dmath}\label{eqn:ProblemSet1Problem2:160}
\lim_{k \rightarrow \infty } A^k = 0.
\end{dmath}

\item
To show that \( \lim_{k \rightarrow \infty } A^k = 0\), it is helpful first to prove that

\begin{dmath}\label{eqn:ProblemSet1Problem2:180}
\Norm{ A^{k+1} }_p \le \Norm{ A }_p \norm{ A^k }_p.
\end{dmath}

\item
Finally, combine your above results and the properties of a norm to show the desired result.
\end{enumerate}
} % makeproblem

\makeanswer{convex-optimization:problemSet1:2}{
\withproblemsetsParagraph{
\makeSubAnswer{}{convex-optimization:problemSet1:2a}

A matrix \( A \in S^n_{++} \) admits a representation

\begin{dmath}\label{eqn:ProblemSet1Problem2:200}
A = Q \Lambda Q^\T,
\end{dmath}

so

\begin{dmath}\label{eqn:ProblemSet1Problem2:220}
\lr{I + A}^{-1}
=
\lr{ I + Q \Lambda Q^\T }^{-1}
=
\lr{ Q Q^\T + Q \Lambda Q^\T }^{-1}
=
\lr{ Q ( I + \Lambda ) Q^\T }^{-1}
=
Q ( I + \Lambda)^{-1} Q^\T.
\end{dmath}

This reduces the problem to the computation of \( ( I + \Lambda)^{-1} \), where \( \Lambda \) is a diagonal matrix.  Such a matrix inverts trivially, provided all the eigenvalues are sufficiently small \( \lambda_i \in [0, 1) \)

\begin{dmath}\label{eqn:ProblemSet1Problem2:240}
\inv{ I + \Lambda }
=
{\begin{bmatrix}
\frac{\delta_{ij}}{ 1 + \lambda_i }
\end{bmatrix}}_{ij}
\approx
[ \delta_{ij}( 1 - \lambda_i ) ]_{ij}
=
1 - \Lambda.
\end{dmath}

This gives

\begin{dmath}\label{eqn:ProblemSet1Problem2:260}
\lr{I + A}^{-1}
=
Q ( I + \Lambda)^{-1} Q^\T
\approx
Q ( I - \Lambda) Q^\T
=
I - A. \qedmarker
\end{dmath}

\makeSubAnswer{}{convex-optimization:problemSet1:2b}

Removing the requirement that the matrix be symmetric, let

\begin{dmath}\label{eqn:ProblemSet1Problem2:280}
S_N
= \sum_{k = 0}^{N-1} A^k.
\end{dmath}

As with a scalar geometric series the sum can be found by subtracting the difference \( S_N - S_{N+1} \)

\begin{equation}\label{eqn:ProblemSetIProblem2:300}
S_N - S_{N+I}
=
S_N(I - A)
=
I - A^N.
\end{equation}

If it can be shown that \( A^N \rightarrow 0 \), then this shows that \( I - A \) must be non-singular.  Additionally, rearranging for \( S^N \), this is

\begin{dmath}\label{eqn:ProblemSetIProblem2:320}
S_N = (I - A^N)(I-A)^{-1},
\end{dmath}

so if \( I - A^N \rightarrow I \), then \cref{eqn:ProblemSet1Problem2:60} has been proven.  Intuitively, if \( A^N \rightarrow 0 \), then \( I - A^N \rightarrow I \), but to make this more meaningful, we want to express this limiting statement in terms of a norm.  The matrix p-norm of this difference is

\begin{dmath}\label{eqn:ProblemSet1Problem2:340}
\Norm{I - A^N}_p
= \sup_{\Bx \ne 0} \frac{ \Norm{ (I - A^N) \Bx }_p }{\Norm{\Bx}_p}
= \max_{\Norm{\Bx}_p = 1} \Norm{ (I - A^N) \Bx }_p
= \max_{\Norm{\Bx}_p = 1} \Norm{ \Bx - A^N \Bx }_p
\le \max_{\Norm{\Bx}_p = 1} \Norm{ \Bx }_p + \Norm{ A^N \Bx }_p
= 1 + \max_{\Norm{\Bx}_p = 1} \Norm{ A^N \Bx }_p.
\end{dmath}

The second last step uses the triangle law property of the norm.  Note that we did not prove that the p-norm has this property in class (called the Minkowski inequality for p-norms), but a nice proof can be found in \citep{tisdellMinkowski}.

\Cref{eqn:ProblemSet1Problem2:340} shows that if \( \Norm{A^N}_p \rightarrow 0 \), that \( I - A \) is non-singular, and the inverse is given by \cref{eqn:ProblemSet1Problem2:60}.

To examine this limit,
first consider the p-norm of a matrix-vector product

\begin{dmath}\label{eqn:ProblemSet1Problem2:360}
\Norm{A \Bx}_p
=
\lr{ \sum_{i = 1}^n \Abs{a_{ij} x_j}^p }^{1/p}
\le
\lr{ \sum_{i = 1}^n \Abs{a_{ij}}^p \Abs{x_j}^p }^{1/p}
\le
\lr{ \sum_{i = 1}^n \Abs{a_{ij}}^p }^{1/p} \lr{ \sum_{i = 1}^n \Abs{x_j}^p }^{1/p}
=
\Norm{A}_p \Norm{\Bx}_p.
\end{dmath}

For the norm of a product of matrices acting on a vector, this means that

\begin{dmath}\label{eqn:ProblemSet1Problem2:380}
\Norm{ A B \Bx }_p \le
\Norm{ A }_p \Norm{B \Bx }_p
\le
\Norm{ A }_p \Norm{B}_p \Norm{ \Bx }_p,
\end{dmath}

in particular

\begin{dmath}\label{eqn:ProblemSet1Problem2:400}
\Norm{A^N \Bx}_p \le \lr{ \Norm{A}_p }^N \Norm{\Bx}_p,
\end{dmath}

and finally
\begin{dmath}\label{eqn:ProblemSet1Problem2:420}
\Norm{I - A^N}_p
\le
1 + \max_{\Norm{\Bx}_p = 1} \lr{ \Norm{ A }_p }^N \Norm{\Bx}_p
= 1 + \lr{ \Norm{ A }_p }^N.
\end{dmath}

This tends to \( 1 \) as \( N \rightarrow \infty \) provided \( \lr{ \Norm{ A }_p } < 1\), proving the inverse power series identity given this constraint.  The only remaining thing to proof is to find the bound on the p-norm of \( (1-A)^{-1} \).  That is

\begin{dmath}\label{eqn:ProblemSet1Problem2:440}
\Norm{ (I - A)^{-1} }_p
=
\Norm{
\sum_{k=0}^\infty A^k
}_p
\le
\sum_{k=0}^\infty
\Norm{
A^k
}_p
\le
\sum_{k=0}^\infty
\lr{ \Norm{ A }_p }^k
= \inv{ 1 - \Norm{A}_p }. \qedmarker
\end{dmath}
} % redaction
} % answer
 % Inversion formula for ``small'' matrices
         \shipoutAnswer
   \mychapter{Matrix inner product, SVD, and set types.}
      %
% Copyright � 2017 Peeter Joot.  All Rights Reserved.
% Licenced as described in the file LICENSE under the root directory of this GIT repository.
%
\section{Matrix inner product}

Given real matrices \( X, Y \in \bbR^{m\times n} \), one possible matrix inner product definition is

\begin{dmath}\label{eqn:convexOptimizationLecture3:20}
\innerprod{X}{Y}
= \Tr( X^\T Y)
= \Tr \lr{ \sum_{k = 1}^m X_{ki} Y_{kj} }
= \sum_{k = 1}^m \sum_{j = 1}^n X_{kj} Y_{kj}
= \sum_{i = 1}^m \sum_{j = 1}^n X_{ij} Y_{ij}.
\end{dmath}

This inner product induces a norm on the (matrix) vector space, called the Frobenius norm

\begin{dmath}\label{eqn:convexOptimizationLecture3:40}
\Norm{X }_F
= \Tr( X^\T X)
= \sqrt{
\innerprod{X}{X} }
=
\sum_{i = 1}^m \sum_{j = 1}^n X_{ij}^2.
\end{dmath}

\section{Range, nullspace.}

\makedefinition{Range.}{dfn:convexOptimizationLecture3:10}{
Given \( A \in \bbR^{m \times n} \), the range of A is the set:

\begin{equation*}
\calR(A) = \setlr{ A \Bx | \Bx \in \bbR^n }.
\end{equation*}
} % definition

\makedefinition{Nullspace.}{dfn:convexOptimizationLecture3:20}{
Given \( A \in \bbR^{m \times n} \),
the nullspace of A is the set:

\begin{equation*}
\calN(A) = \setlr{ \Bx | A \Bx = 0 }.
\end{equation*}
} % definition

\section{SVD.}

To understand operation of \( A \in \bbR^{m \times n} \), a representation of a linear transformation from \R{n} to \R{m},
decompose \( A \) using the singular value decomposition (SVD).

\makedefinition{SVD.}{dfn:convexOptimizationLecture3:40}{
Given \( A \in \bbR^{m \times n} \),
an operator on \( \Bx \in \bbR^n \),
a decomposition of the following form is always possible

\begin{equation*}
\begin{aligned}
A &= U \Sigma V^\T \\
U &\in \bbR^{m \times r} \\
V &\in \bbR^{n \times r},
\end{aligned}
\end{equation*}

where
\( r \) is the rank of \(A\), and
both \( U \) and \( V \) are orthogonal

\begin{equation*}
\begin{aligned}
U^\T U &= I \in \bbR^{r \times r} \\
V^\T V &= I \in \bbR^{r \times r}.
\end{aligned}
\end{equation*}

Here \( \Sigma = \diag( \sigma_1, \sigma_2, \cdots, \sigma_r ) \), is a
diagonal matrix of ``singular'' values, where

\begin{equation*}
\sigma_1 \ge \sigma_2 \ge \cdots \ge \sigma_r.
\end{equation*}
} % definition

For simplicity consider square case \( m = n \)

\begin{dmath}\label{eqn:convexOptimizationLecture3:100}
A \Bx = \lr{ U \Sigma V^\T } \Bx.
\end{dmath}

The first product \( V^\T \Bx \) is a rotation, which can be checked by looking at the length

\begin{dmath}\label{eqn:convexOptimizationLecture3:120}
\Norm{ V^\T \Bx}_2
= \sqrt{ \Bx^\T V V^\T \Bx }
= \sqrt{ \Bx^\T \Bx }
= \Norm{ \Bx }_2,
\end{dmath}

which shows that the length of the vector is unchanged after application of the linear transformation represented by \( V^\T \) so that operation must be a rotation.

Similarly the operation of \( U \) on \( \Sigma V^\T \Bx \) also must be a rotation.  The operation \( \Sigma = [\sigma_i]_i \) applies a scaling operation to each component of the vector \( V^\T \Bx \).

All linear (square) transformations can therefore be thought of as a rotate-scale-rotate operation.  Often the \( A \) of interest will be symmetric \( A = A^\T \).

\section{Set of symmetric matrices}

Let \( S^n \) be the set of real, symmetric \( n \times n \) matrices.

\maketheorem{Spectral theorem.}{thm:convexOptimizationLecture3:50}{
When \( A \in S^n \) then it is possible to factor \( A \) as

\begin{equation*}
A = Q \Lambda Q^\T,
\end{equation*}

where \( Q \) is an orthogonal matrix, and \( \Lambda = \diag( \lambda_1, \lambda_2, \cdots \lambda_n)\).  Here \( \lambda_i \in \bbR \, \forall i \) are the (real) eigenvalues of \( A \).

A real symmetric matrix \( A \in S^n\)
is ``positive semi-definite'' if

\begin{equation*}
\Bv^\T A \Bv \ge 0 \qquad\forall \Bv \in \bbR^n, \Bv \ne 0,
\end{equation*}
and is ``positive definite'' if

\begin{equation*}
\Bv^\T A \Bv > 0 \qquad\forall \Bv \in \bbR^n, \Bv \ne 0.
\end{equation*}

The set of such matrices is denoted \( S^n_{+} \), and \( S^n_{++} \) respectively.
} % theorem

Consider \( A \in S^n_{+} \) (or \( S^n_{++} \) )

\begin{dmath}\label{eqn:convexOptimizationLecture3:200}
A = Q \Lambda Q^\T,
\end{dmath}

possible since the matrix is symmetric.  For such a matrix

\begin{dmath}\label{eqn:convexOptimizationLecture3:220}
\Bv^\T A \Bv
=
\Bv^\T Q \Lambda A^\T \Bv
=
\Bw^\T \Lambda \Bw,
\end{dmath}

where \( \Bw = A^\T \Bv \).  Such a product is

\begin{dmath}\label{eqn:convexOptimizationLecture3:240}
\Bv^\T A \Bv
=
\sum_{i = 1}^n \lambda_i w_i^2.
\end{dmath}

So, if \( \lambda_i \ge 0 \) (\(\lambda_i > 0 \) ) then \( \sum_{i = 1}^n \lambda_i w_i^2 \) is non-negative (positive) \( \forall \Bw \in \bbR^n, \Bw \ne 0 \).  Since \( \Bw \) is just a rotated version of \( \Bv \) this also holds for all \( \Bv \).  A necessary and sufficient condition for \( A \in S^n_{+} \) (\( S^n_{++} \) ) is \( \lambda_i \ge 0 \) (\(\lambda_i > 0\)).

\section{Square root of positive semi-definite matrix}

Real symmetric matrix power relationships such as

\begin{dmath}\label{eqn:convexOptimizationLecture3:260}
A^2
=
Q \Lambda Q^\T
Q \Lambda Q^\T
=
Q \Lambda^2
Q^\T
,
\end{dmath}

or more generally \( A^k = Q \Lambda^k Q^\T,\, k \in \bbZ \), can be further generalized to non-integral powers.  In particular, the square root (non-unique) of a square matrix can be written

\begin{dmath}\label{eqn:convexOptimizationLecture3:280}
A^{1/2} = Q
\begin{bmatrix}
\sqrt{\lambda_1} &                  &        &  \\
                 & \sqrt{\lambda_2} &        & \\
                 &                  & \ddots & \\
                 &                  &        & \sqrt{\lambda_n} \\
\end{bmatrix}
Q^\T,
\end{dmath}

since \( A^{1/2} A^{1/2} = A \), regardless of the sign picked for the square roots in question.

\section{Functions of matrices}

Consider \( F : S^n \rightarrow \bbR \), and define

\begin{dmath}\label{eqn:convexOptimizationLecture3:300}
F(X) = \log \det X,
\end{dmath}

Here \( \dom F = S^n_{++} \).
The task is to find \( \spacegrad F \), which can be done by looking at the perturbation
\( \log \det ( X + \Delta X ) \)

\begin{dmath}\label{eqn:convexOptimizationLecture3:320}
\log \det ( X + \Delta X )
=
\log \det ( X^{1/2} (I + X^{-1/2} \Delta X X^{-1/2}) X^{1/2} )
=
\log \det ( X (I + X^{-1/2} \Delta X X^{-1/2}) )
=
\log \det X  + \log \det (I + X^{-1/2} \Delta X X^{-1/2}).
\end{dmath}

Let \( X^{-1/2} \Delta X X^{-1/2} = M \) where \( \lambda_i \) are the eigenvalues of \( M : M \Bv = \lambda_i \Bv \) when \( \Bv \) is an eigenvector of \( M \).  In particular

\begin{dmath}\label{eqn:convexOptimizationLecture3:340}
(I + M) \Bv =
(1 + \lambda_i) \Bv,
\end{dmath}

where \( 1 + \lambda_i \) are the eigenvalues of the \( I + M \) matrix.  Since the determinant is the product of the eigenvalues, this gives

\begin{dmath}\label{eqn:convexOptimizationLecture3:360}
\log \det ( X + \Delta X )
=
\log \det X +
\log \prod_{i = 1}^n (1 + \lambda_i)
=
\log \det X +
\sum_{i = 1}^n \log (1 + \lambda_i).
\end{dmath}

If \( \lambda_i \) are sufficiently ``small'', then \( \log ( 1 + \lambda_i ) \approx \lambda_i \), giving

\begin{dmath}\label{eqn:convexOptimizationLecture3:380}
\log \det ( X + \Delta X )
=
\log \det X +
\sum_{i = 1}^n \lambda_i
\approx
\log \det X +
\Tr( X^{-1/2} \Delta X X^{-1/2} ).
\end{dmath}

Since
\begin{dmath}\label{eqn:convexOptimizationLecture3:400}
\Tr( A B ) = \Tr( B A ),
\end{dmath}

% also used above:
%%%\det (A B) = \det (BA) = \det A \det B

this trace operation can be written as

\begin{dmath}\label{eqn:convexOptimizationLecture3:420}
\log \det ( X + \Delta X )
\approx
\log \det X +
\Tr( X^{-1} \Delta X )
=
\log \det X +
\innerprod{ X^{-1}}{\Delta X},
\end{dmath}

so
\begin{dmath}\label{eqn:convexOptimizationLecture3:440}
\spacegrad F(X) = X^{-1}.
\end{dmath}

%%This trace is the ``matrix inner product''.

To check this, consider the simplest example with \( X \in \bbR^{1 \times 1} \), where we have

\begin{dmath}\label{eqn:convexOptimizationLecture3:460}
\frac{d}{dX} \lr{ \log \det X } = \frac{d}{dX} \lr{ \log X } = \inv{X} = X^{-1}.
\end{dmath}

This is a nice example demonstrating how the gradient can be obtained by performing a
first order perturbation of the function.  The gradient can then be read off from the result.

\section{Second order perturbations}

\begin{itemize}
\item To get first order approximation found the part that varied linearly in \( \Delta X \).
\item To get the second order part, perturb \( X^{-1} \) by \( \Delta X \) and see how that perturbation varies in \( \Delta X \).
\end{itemize}

For \( G(X) = X^{-1} \), this is

\begin{dmath}\label{eqn:convexOptimizationLecture3:480}
(X + \Delta X)^{-1}
=
\lr{ X^{1/2} (I + X^{-1/2} \Delta X X^{-1/2} ) X^{1/2} }^{-1}
=
X^{-1/2} (I + X^{-1/2} \Delta X X^{-1/2} )^{-1} X^{-1/2}
\end{dmath}

To be proven in the homework (for ``small'' A)

\begin{dmath}\label{eqn:convexOptimizationLecture3:500}
(I + A)^{-1} \approx I - A.
\end{dmath}

This gives

\begin{dmath}\label{eqn:convexOptimizationLecture3:520}
(X + \Delta X)^{-1}
=
X^{-1/2} (I - X^{-1/2} \Delta X X^{-1/2} ) X^{-1/2}
=
X^{-1} - X^{-1} \Delta X X^{-1},
\end{dmath}

or

\begin{dmath}\label{eqn:convexOptimizationLecture3:800}
G(X + \Delta X)
= G(X) + (D G) \Delta X
= G(X) + (\spacegrad G)^\T \Delta X,
\end{dmath}

so
\begin{dmath}\label{eqn:convexOptimizationLecture3:820}
(\spacegrad G)^\T \Delta X
=
- X^{-1} \Delta X X^{-1}.
\end{dmath}

The Taylor expansion of \( F \) to second order is

\begin{dmath}\label{eqn:convexOptimizationLecture3:840}
F(X + \Delta X)
=
F(X)
+
\Tr \lr{ (\spacegrad F)^\T \Delta X}
+
\inv{2}
\lr{ (\Delta X)^\T (\spacegrad^2 F) \Delta X}.
%=
%F(X)
%+
%\Tr \lr{ (\spacegrad F)^\T \Delta X}
%+
%\inv{2}
%\Tr \lr{ (\Delta X)^\T (\spacegrad^2 F) \Delta X},
\end{dmath}

The first trace can be expressed as an inner product

\begin{dmath}\label{eqn:convexOptimizationLecture3:860}
\Tr \lr{ (\spacegrad F)^\T \Delta X}
=
\innerprod{ \spacegrad F }{\Delta X}
=
\innerprod{ X^{-1} }{\Delta X}.
\end{dmath}

The second trace also has the structure of an inner product

\begin{dmath}\label{eqn:convexOptimizationLecture3:880}
(\Delta X)^\T (\spacegrad^2 F) \Delta X
=
\Tr \lr{ (\Delta X)^\T (\spacegrad^2 F) \Delta X}
=
\innerprod{ (\spacegrad^2 F)^\T \Delta X }{\Delta X},
\end{dmath}

where a no-op trace could be inserted in the second order term since that quadratic form is already a scalar.
This \( (\spacegrad^2 F)^\T \Delta X \) term has essentially been found implicitly by performing the linear variation of \( \spacegrad F \) in \( \Delta X \), showing that we must have

\begin{dmath}\label{eqn:convexOptimizationLecture3:900}
\Tr \lr{ (\Delta X)^\T (\spacegrad^2 F) \Delta X}
=
\innerprod{ - X^{-1} \Delta X X^{-1} }{\Delta X},
\end{dmath}

%Since the term \( X^{-1} \Delta X X^{-1} \) varies linearly in \( \Delta X \) the first and second order changes are respectively
so
\begin{dmath}\label{eqn:convexOptimizationLecture3:560}
F( X + \Delta X) = F(X) +
\innerprod{X^{-1}}{\Delta X}
+\inv{2} \innerprod{-X^{-1} \Delta X X^{-1}}{\Delta X},
\end{dmath}

or
\begin{dmath}\label{eqn:convexOptimizationLecture3:580}
\log \det ( X + \Delta X) = \log \det X +
\Tr( X^{-1} \Delta X )
- \inv{2} \Tr( X^{-1} \Delta X X^{-1} \Delta X ).
\end{dmath}

%FIXME: The trace operations in
%\cref{eqn:convexOptimizationLecture3:840} weren't in the
%Jacobian and Hessian expansion in lecture 2, nor in the text?

\section{Convex Sets}

\begin{itemize}
\item Types of sets: Affine, convex, cones
\item Examples: Hyperplanes, polyhedra, balls, ellipses, norm balls, cone of PSD matrices.
\end{itemize}

\makedefinition{Affine set}{dfn:convexOptimizationLecture3:1}{

A set \( C \subseteq \bbR^n \) is affine if \( \forall \Bx_1, \Bx_2 \in C \) then

\begin{equation*}
\theta \Bx_1 + (1 -\theta) \Bx_2 \in C, \qquad \forall \theta \in \bbR.
\end{equation*}
} % definition

The affine sum above can
be rewritten as

\begin{dmath}\label{eqn:convexOptimizationLecture3:600}
\Bx_2 + \theta (\Bx_1 - \Bx_2).
\end{dmath}

Since \( \theta \) is a scaling, this is the line containing \( \Bx_2 \) in the direction between \( \Bx_1 \) and \( \Bx_2 \).

Observe that the solution to a set of linear equations

\begin{equation}\label{eqn:convexOptimizationLecture3:620}
C = \setlr{ \Bx | A \Bx = \Bb },
\end{equation}

is an affine set.  To check, note that

\begin{dmath}\label{eqn:convexOptimizationLecture3:640}
A (\theta \Bx_1 + (1 - \theta) \Bx_2)
=
\theta A \Bx_1 + (1 - \theta) A \Bx_2
=
\theta \Bb + (1 - \theta) \Bb
= \Bb.
\end{dmath}

\makedefinition{Affine combination.}{dfn:convexOptimizationLecture3:71}{
An affine combination of points \( \Bx_1, \Bx_2, \cdots \Bx_n \) is

\begin{equation*}
\sum_{i = 1}^n \theta_i \Bx_i,
\end{equation*}

such that for \( \theta_i \in \bbR \)

\begin{equation*}
\sum_{i = 1}^n \theta_i = 1.
\end{equation*}
} % definition

An affine set contains all affine combinations of points in the set.  Examples of a couple
affine sets are sketched in
\cref{fig:l3AffineFig1and2}.

\imageTwoFigures
{../figures/ece1505-convex-optimization/l3AffineFig1}
{../figures/ece1505-convex-optimization/l3AffineFig2}
{Affine.}
{fig:l3AffineFig1and2}
{scale=0.1}

For comparison, a couple of non-affine sets are sketched in \cref
{fig:l3NotAffineFig3and4}.

\imageTwoFigures
{../figures/ece1505-convex-optimization/l3NotAffineFig3}
{../figures/ece1505-convex-optimization/l3NotAffineFig4}
{Not affine.}
{fig:l3NotAffineFig3and4}
{scale=0.1}

\makedefinition{Convex set}{dfn:convexOptimizationLecture3:2}{

A set \( C \subseteq \bbR^n \) is convex if \( \forall \Bx_1, \Bx_2 \in C \) and \( \forall \theta \in \bbR, \theta \in [0,1] \), the combination

\begin{dmath}\label{eqn:convexOptimizationLecture3:700}
\theta \Bx_1 + (1 - \theta) \Bx_2 \in C.
\end{dmath}
} % definition

\makedefinition{Convex combination}{dfn:convexOptimizationLecture3:3}{

A convex combination of \( \Bx_1, \Bx_2, \cdots \Bx_n \) is

\begin{equation*}
\sum_{i = 1}^n \theta_i \Bx_i,
\end{equation*}

such that \( \forall \theta_i \ge 0 \)

\begin{equation*}
\sum_{i = 1}^n \theta_i = 1
\end{equation*}

} % definition

\makedefinition{Convex hull.}{dfn:convexOptimizationLecture3:3a}{

Convex hull of a set \( C \) is a set of all convex combinations of points in \(C\), denoted

\begin{equation}\label{eqn:convexOptimizationLecture3:720}
\conv(C) = \setlr{ \sum_{i=1}^n \theta_i \Bx_i | \Bx_i \in C, \theta_i \ge 0, \sum_{i=1}^n \theta_i = 1 }.
\end{equation}
} % definition

A non-convex set can be converted into a convex hull by filling in all the combinations of points connecting points in the set, as sketched in \cref
{fig:l3NotAffineHull:l3NotAffineHullFig6}.

\imageTwoFigures{../figures/ece1505-convex-optimization/l3NotAffineHullFig5}
{../figures/ece1505-convex-optimization/l3NotAffineHullFig6}
{Convex hulls.}
{fig:l3NotAffineHull:l3NotAffineHullFig6}
{scale=0.1}

\makedefinition{Cones.}{dfn:convexOptimizationLecture3:4}{

A set \(C\) is a cone if \( \forall \Bx \in C \) and \( \forall \theta \ge 0 \) we have \( \theta \Bx \in C\).
} % definition

This scales out if \(\theta > 1\) and scales in if \(\theta < 1\).

A convex cone is a cone that is also a convex set.  A conic combination is

\begin{equation*}
\sum_{i=1}^n \theta_i \Bx_i, \theta_i \ge 0.
\end{equation*}

A convex and non-convex 2D cone is sketched in \cref{fig:ConesFig8}

\imageTwoFigures{../figures/ece1505-convex-optimization/l3ConvexConeFig7}
{../figures/ece1505-convex-optimization/l3NotConvexConeFig8}
{Convex and non-convex cone.}
{fig:ConesFig8}
{scale=0.1}

A comparison of properties for different set types is tabulated in \cref{tab:affineConvexConic:1}.

\captionedTable{Affine, Convex, and Conic properties.}{tab:affineConvexConic:1}{
\centering
\begin{tabular}{l|l|l|}
\cline{2-3}
                             & \( \theta_i \ge 0 \) & \( \sum \theta_i = 1 \) \\ \hline
\multicolumn{1}{|l|}{Affine} & No                   & Yes                     \\ \hline
\multicolumn{1}{|l|}{Convex} & Yes                  & Yes                     \\ \hline
\multicolumn{1}{|l|}{Conic}  & Yes                  & No                      \\ \hline
\end{tabular}
}

\section{Hyperplanes and half spaces}

\makedefinition{Hyperplane.}{dfn:convexOptimizationLecture3:99}{
A hyperplane is defined by

\begin{equation*}
\setlr{ \Bx | \Ba^\T \Bx = \Bb, \Ba \ne 0 }.
\end{equation*}
} % definition

A line and plane are examples of this general construct as sketched in
\cref{fig:l3hyperPlaneLine:l3hyperPlanesFig9}.

\imageTwoFigures{../figures/ece1505-convex-optimization/l3hyperPlaneLineFig9}
{../figures/ece1505-convex-optimization/l3hyperplanePlaneFig10}
{Hyperplanes.}
{fig:l3hyperPlaneLine:l3hyperPlanesFig9}
{scale=0.1}

An alternate view is possible should one
find any specific \( \Bx_0 \) such that \( \Ba^\T \Bx_0 = \Bb \)

\begin{dmath}\label{eqn:convexOptimizationLecture3:740}
\setlr{\Bx | \Ba^\T \Bx = b }
=
\setlr{\Bx | \Ba^\T (\Bx -\Bx_0) = 0 }
\end{dmath}

This shows that \( \Bx - \Bx_0 = \Ba^\perp \) is perpendicular to \( \Ba \), or

\begin{dmath}\label{eqn:convexOptimizationLecture3:780}
\Bx
=
\Bx_0 + \Ba^\perp.
\end{dmath}

This is the subspace perpendicular to \( \Ba \) shifted by \(\Bx_0\), subject to \( \Ba^\T \Bx_0 = \Bb \).  As a set

\begin{equation}\label{eqn:convexOptimizationLecture3:760}
\Ba^\perp = \setlr{ \Bv | \Ba^\T \Bv = 0 }.
\end{equation}

\section{Half space}

\makedefinition{Half space.}{dfn:convexOptimizationLecture3:101}{

The half space is defined as
\begin{equation*}
\setlr{ \Bx | \Ba^\T \Bx = \Bb }
= \setlr{ \Bx | \Ba^\T (\Bx - \Bx_0) \le 0 }.
\end{equation*}
} % definition

This can also be expressed as \( \setlr{ \Bx | \innerprod{ \Ba }{\Bx - \Bx_0 } \le 0 } \).

      \section{Problems.}
         %
% Copyright � 2017 Peeter Joot.  All Rights Reserved.
% Licenced as described in the file LICENSE under the root directory of this GIT repository.
%
\makeproblem{Matrix inner product}{convex-optimization:problemSet1:3}{
\makesubproblem{}{convex-optimization:problemSet1:3a}
Verify that \( \calS^n \subseteq \Rm{n \times n} \) is a vector space under the regular matrix addition and scaling (multiplication by scalars in \(\bbR\)) operations.
Accomplish this by verifying that all properties of a vector space are satisfied.

\makesubproblem{}{convex-optimization:problemSet1:3b}
Verify that \( \innerprod{A}{B} = \Tr(A^\T B) \) where \( A, B \in \calS^n \) satisfies all the properties of an inner product.
} % makeproblem

\makeanswer{convex-optimization:problemSet1:3}{
\withproblemsetsParagraph{
\makeSubAnswer{}{convex-optimization:problemSet1:3a}

From \citep{damiano1988course},
\makedefinition{Vector space.}{dfn:ProblemSet1Problem3:1}{
A vector space is a set of elements (vectors),
which is

\begin{enumerate}[(i)]
\item
closed under scaling \( a \Bx \in V \, \forall \Bx \in V \),
\item closed with respect to an additive operation \( \Bx + \By \in V\, \forall \Bx, \By \in V \), and
\end{enumerate}
for \( \Bx, \By, \Bz \in V \) and scalars \( c, d \in \bbR \) satisfies the following properties

\begin{enumerate}
\item \( (\Bx + \By) + \Bz = \Bx + (\By + \Bz ) \).
\item \( \Bx + \By = \By + \Bx \).
\item \( \Bx + \Bzero = \Bx\) for some vector \( \Bzero \in V \).
\item There exists a vector \( -\Bx \in V \), such that \( \Bx + (-\Bx) = \Bzero\).
\item \( c( \Bx + \By ) = c \Bx + c \By \).
\item \( (c + d)\Bx = c \Bx + d \Bx \).
\item \( (c d) \Bx = c (d \Bx) \).
\item \( 1 \Bx = \Bx \).
\end{enumerate}
} % definition

Properties 1-8 are all clearly satisfied by \( S^n \), leaving just closure to demonstrate.

%For the particular case of the set of real symmetric matrices \( S^n \), the properties that have to be checked are really just closure under scaling and closure under addition.  To verify this,
For \( A, B \in S^n \), with elements \( a_{ij} = a_{ji} \) and \( b_{ij} = b_{ji} \) respectively,
the elements of the sum \( \alpha A + \beta B \) are

\begin{dmath}\label{eqn:ProblemSet1Problem3:21}
\alpha a_{ij} + \beta b_{ij}
= \alpha a_{ji} + \beta b_{ji},
\end{dmath}

or
\begin{dmath}\label{eqn:ProblemSet1Problem3:41}
\alpha A + \beta B = (\alpha A + \beta B)^\T,
\end{dmath}

which demonstrates the closure properties.

\makeSubAnswer{}{convex-optimization:problemSet1:3b}

From \citep{nicholson1990elementary}

\makedefinition{Real inner product.}{dfn:ProblemSet1Problem3:13}{
A real inner product over the vector space \( V \) has the properties

\begin{enumerate}
\item \( \innerprod{A}{B} \in \bbR, \,\forall A,B \in V \).
\item \( \innerprod{A}{B} = \innerprod{B}{A} \,\forall A,B \in V \).
\item \( \innerprod{A + B}{C} = \innerprod{A}{C} + \innerprod{B}{C} \,\forall A,B,C \in V \).
\item \( \innerprod{r A}{B} = r \innerprod{A,B}\,\forall A,B \in V, r \in \bbR \).
\item \( \innerprod{A}{A} > 0 \,\forall A \ne 0 \in V \).
\end{enumerate}
} % definition

For matrices \( A, B, C \in S^n \) with elements \( a_{ij}, b_{ij}, c_{ij} \) respectively, the proofs of these properties are

\begin{enumerate}
\item
\begin{dmath}\label{eqn:ProblemSet1Problem3:61}
\innerprod{A}{B} = \Tr A^\T B = \sum_k \Tr a_{ki} b_{kj} = \sum_{ki} a_{ki} b_{ki} \in \bbR. \qedmarker
\end{dmath}
\item
Changing indexes in \cref{eqn:ProblemSet1Problem3:61}, and making use of \( a_{ij} = a_{ji} \) for matrices in \( S^n \), we have
\begin{dmath}\label{eqn:ProblemSet1Problem3:81}
\innerprod{A}{B} = \sum_{ij} a_{ij} b_{ij},
\end{dmath}

Similarly, the inner product in the reverse order is

\begin{dmath}\label{eqn:ProblemSet1Problem3:101}
\innerprod{B}{A} = \Tr B^\T A
= \sum_k \Tr b_{ki} a_{kj}
= \sum_{ki} b_{ki} a_{ki}
= \sum_{ji} b_{ji} a_{ji}
= \sum_{ji} a_{ij} b_{ij}
= \innerprod{A}{B}. \qedmarker
\end{dmath}
\item
\begin{dmath}\label{eqn:ProblemSet1Problem3:121}
\innerprod{A + B}{C}
= \Tr ((A + B)^\T C)
= \Tr ((A^\T + B^\T) C)
= \Tr (A^\T C + B^\T C)
= \Tr (A^\T C) + \Tr (B^\T C)
=
\innerprod{A}{C}
+
\innerprod{B}{C} .\qedmarker
\end{dmath}
\item
\begin{dmath}\label{eqn:ProblemSet1Problem3:141}
\innerprod{r A}{B}
=
\Tr (r A)^\T B
=
\Tr r A^\T B
=
r \Tr A^\T B
= r \innerprod{A}{B}.\qedmarker
\end{dmath}
\item
\begin{dmath}\label{eqn:ProblemSet1Problem3:161}
\innerprod{A}{A}
=
\sum_{ij} a_{ij}^2,
\end{dmath}

which is greater than zero for \( A \ne 0 \).
\end{enumerate}
} % redaction
} % answer
 % Matrix inner product
         \shipoutAnswer
   \mychapter{Sets and convexity.}
      %
% Copyright � 2017 Peeter Joot.  All Rights Reserved.
% Licenced as described in the file LICENSE under the root directory of this GIT repository.
%
\input{../latex/blogpost.tex}
\renewcommand{\basename}{convexOptimization4}
\renewcommand{\dirname}{notes/ece1505/}
\newcommand{\keywords}{ECE1505H}
\input{../latex/peeter_prologue_print2.tex}

\usepackage{ece1505}
\usepackage{peeters_braket}
\usepackage{peeters_layout_exercise}
\usepackage{peeters_figures}
\usepackage{mathtools}
\usepackage{siunitx}
\usepackage{macros_cal}

\beginArtNoToc
\generatetitle{ECE1505H Convex Optimization.  Lecture 4: Sets and convexity.  Taught by Prof.\ Stark Draper}
%\chapter{Sets and convexity}
\label{chap:convexOptimization4}

\paragraph{Disclaimer}

Peeter's lecture notes from class.  These may be incoherent and rough.

These are notes for the UofT course ECE1505H, Convex Optimization, taught by Prof. Stark Draper, covering \citep{boyd2004convex} content.

%\paragraph{Last time}
\paragraph{Today}

\begin{itemize}
\item more on various sets: hyperplanes, half-spaces, polyhedra, balls, ellipses, norm balls, cone of PSD
\item generalize inequalities
\item operations that preserve convexity
\item separating and supporting hyperplanes.
\end{itemize}

\section{Hyperplanes}

Find some \( \Bx_0 \in \bbR^n \) such that \( \Ba^\T \Bx_0 = \Bb \), so

\begin{dmath}\label{eqn:convexOptimizationLecture4:20}
\begin{aligned}
\setlr{ \Bx | \Ba^\T \Bx = \Bb }
&=
\setlr{ \Bx | \Ba^\T \Bx = \Ba^\T \Bx_0 } \\
&=
\setlr{ \Bx | \Ba^\T (\Bx - \Bx_0) } \\
&=
\Bx_0 + \Ba^\perp,
\end{aligned}
\end{dmath}

where

\begin{equation}\label{eqn:convexOptimizationLecture4:40}
\Ba^\perp = \setlr{ \Bv | \Ba^\T \Bv = 0 }.
\end{equation}

%F1
%\cref{fig:l4:l4Fig1}.
\imageFigure{../figures/ece1505-convex-optimization/l4Fig1}{CAPTION: l4Fig1}{fig:l4:l4Fig1}{0.2}

Recall

\begin{equation}\label{eqn:convexOptimizationLecture4:60}
\Norm{\Bz}_\conj = \sup_\Bx \setlr{ \Bz^\T \Bx | \Norm{\Bx} \le 1 }
\end{equation}

Denote the optimizer of above as \( \Bx^\conj \).  By definition

\begin{equation}\label{eqn:convexOptimizationLecture4:80}
\Bz^\T \Bx^\conj \ge \Bz^\T \Bx \quad \forall \Bx, \Norm{\Bx} \le 1
\end{equation}

This defines a half space in which the unit ball

\begin{equation}\label{eqn:convexOptimizationLecture4:100}
\setlr{ \Bx | \Bz^\T (\Bx - \Bx^\conj \le 0 }
\end{equation}

Start with the \( l_1 \) norm, duals of \( l_1 \) is \( l_\infty \)

%F2
%\cref{fig:l4HalfspaceContiningUnitBall:l4HalfspaceContiningUnitBallFig2}.
\imageFigure{../figures/ece1505-convex-optimization/l4HalfspaceContiningUnitBallFig2}{CAPTION: l4HalfspaceContiningUnitBallFig2}{fig:l4HalfspaceContiningUnitBall:l4HalfspaceContiningUnitBallFig2}{0.2}

Similar pic for \( l_\infty \), for which the dual is the \( l_1 \) norm, as sketched in
\cref{fig:l4LinfinityUnitBallHalfSpace:l4LinfinityUnitBallHalfSpaceFig3}.
Here the optimizer point is at \( (1,1) \)

\imageFigure{../figures/ece1505-convex-optimization/l4LinfinityUnitBallHalfSpaceFig3}{CAPTION: l4LinfinityUnitBallHalfSpaceFig3}{fig:l4LinfinityUnitBallHalfSpace:l4LinfinityUnitBallHalfSpaceFig3}{0.2}

and a similar pic for \( l_2 \), which is sketched in \cref{fig:l4L2unitballhalfplane:l4L2unitballhalfplaneFig4}.

\imageFigure{../figures/ece1505-convex-optimization/l4L2unitballhalfplaneFig4}{CAPTION: l4L2unitballhalfplaneFig4}{fig:l4L2unitballhalfplane:l4L2unitballhalfplaneFig4}{0.2}

Q: What was this optimizer point?

\section{Polyhedra}

\begin{equation}\label{eqn:convexOptimizationLecture4:120}
\calP = \setlr{ \Bx |
\Ba_j^\T \Bx \le \Bb_j, j \in [1,m],
\Bc_i^\T \Bx = \Bd_i, i \in [1,p]
}
=
\setlr{ \Bx | A \Bx \le \Bb, C \Bx = d },
\end{equation}

where the final inequality and equality are component wise.

Proving \( \calP \) is convex:

\begin{itemize}
\item Pick \(\Bx_1 \in \calP\), \(\Bx_2 \in \calP \)
\item Pick any \(\theta \in [0,1]\)
\item Test \( \theta \Bx_1 + (1-\theta) \Bx_2 \).  Is it in \(\calP\)?
\end{itemize}

\begin{dmath}\label{eqn:convexOptimizationLecture4:140}
A \lr{ \theta \Bx_1 + (1-\theta) \Bx_2 }
=
\theta A \Bx_1 + (1-\theta) A \Bx_2
\le
\theta \Bb + (1-\theta) \Bb
=
\Bb.
\end{dmath}

\section{Balls}

Euclidean ball for \( \Bx_c \in \bbR^n, r \in \bbR \)

\begin{equation}\label{eqn:convexOptimizationLecture4:160}
\calB(\Bx_c, r)
= \setlr{ \Bx | \Norm{\Bx - \Bx_c}_2 \le r },
\end{equation}

or
\begin{equation}\label{eqn:convexOptimizationLecture4:180}
\calB(\Bx_c, r)
= \setlr{ \Bx | \lr{\Bx - \Bx_c}^\T \lr{\Bx - \Bx_c} \le r^2 }.
\end{equation}

Let \( \Bx_1, \Bx_2 \), \(\theta \in [0,1]\)

\begin{dmath}\label{eqn:convexOptimizationLecture4:200}
\Norm{ \theta \Bx_1 + (1-\theta) \Bx_2 - \Bx_c }_2
=
\Norm{ \theta (\Bx_1 - \Bx_c) + (1-\theta) (\Bx_2 - \Bx_c) }_2
\le
\Norm{ \theta (\Bx_1 - \Bx_c)}_2 + \Norm{(1-\theta) (\Bx_2 - \Bx_c) }_2
=
\Abs{\theta} \Norm{ \Bx_1 - \Bx_c}_2 + \Abs{1 -\theta} \Norm{ \Bx_2 - \Bx_c }_2
=
\theta \Norm{ \Bx_1 - \Bx_c}_2 + \lr{1 -\theta} \Norm{ \Bx_2 - \Bx_c }_2
\le
\theta r + (1 - \theta) r
= r
\end{dmath}

\section{Ellipse}

\begin{equation}\label{eqn:convexOptimizationLecture4:220}
\calE(\Bx_c, P)
=
\setlr{ \Bx | (\Bx - \Bx_c)^\T P^{-1} (\Bx - \Bx_c) \le 1 },
\end{equation}

where \( P \in S^n_{++} \).

\begin{itemize}
\item Euclidean ball is an ellipse with \( P = I r^2 \)
\item Ellipse is image of Euclidean ball \( \calB(0,1) \) under affine mapping.
\end{itemize}

%\cref{fig:l4CircleAndEllipse:l4CircleAndEllipseFig5}.
\imageFigure{../figures/ece1505-convex-optimization/l4CircleAndEllipseFig5}{CAPTION: l4CircleAndEllipseFig5}{fig:l4CircleAndEllipse:l4CircleAndEllipseFig5}{0.2}

Given

\begin{dmath}\label{eqn:convexOptimizationLecture4:240}
F(\Bu) = P^{1/2} \Bu + \Bx_c
\end{dmath}

\begin{dmath}\label{eqn:convexOptimizationLecture4:260}
\begin{aligned}
\setlr{ F(\Bu) | \Norm{\Bu}_2 \le r }
&=
\setlr{ P^{1/2} \Bu + \Bx_c | \Bu^\T \Bu \le r^2 } \\
&=
\setlr{ \Bx | \Bx = P^{1/2} \Bu + \Bx_c, \Bu^\T \Bu \le r^2 } \\
&=
\setlr{ \Bx | \Bu = P^{-1/2} (\Bx - \Bx_c), \Bu^\T \Bu \le r^2 } \\
&=
\setlr{ \Bx | (\Bx - \Bx_c)^\T P^{-1} (\Bx - \Bx_c) \le r^2 }
\end{aligned}
\end{dmath}

\section{Geometry of an ellipse}

%\succeq
Decomposition of positive definite matrix \( P \in S^n_{++} \subset S^n \) is:

\begin{dmath}\label{eqn:convexOptimizationLecture4:280}
\begin{aligned}
P &= Q \diag(\lambda_i) Q^\T \\
Q^\T Q &= 1
\end{aligned},
\end{dmath}

where \( \lambda_i \in \bbR\), and \(\lambda_i > 0 \).

The ellipse is defined by

\begin{equation}\label{eqn:convexOptimizationLecture4:300}
(\Bx - \Bx_c)^\T Q \diag(1/\lambda_i) (\Bx - \Bx_c) Q \le r^2
\end{equation}

The term \( (\Bx - \Bx_c)^\T Q \) projects \( \Bx - \Bx_c \) onto the columns of \( Q \).  Those columns are perpendicular since \( Q \) is an orthogonal matrix.

Let

\begin{equation}\label{eqn:convexOptimizationLecture4:320}
\tilde{\Bx} = Q^\T (\Bx - \Bx_c),
\end{equation}

this shifts the origin around \( \Bx_c \) and \( Q \) rotates into a new coordinate system.

The ellipse is therefore

\begin{equation}\label{eqn:convexOptimizationLecture4:340}
\tilde{\Bx}^\T
\begin{bmatrix}
\inv{\lambda_1} &                &        & \\
                &\inv{\lambda_2} &        & \\
                                 & \ddots & \\
                &                &        & \inv{\lambda_n}
\end{bmatrix}
\tilde{\Bx}
=
\sum_{i = 1}^n \frac{\tilde{x}_i^2}{\lambda_i} \le 1.
\end{equation}

An example is sketched for \( \lambda_1 > \lambda_2 \) in \cref{fig:l4EllipseGeometry:l4EllipseGeometryFig6Lambda1gtLambda2}.
\imageFigure{../figures/ece1505-convex-optimization/l4EllipseGeometryFig6Lambda1gtLambda2}{Ellipse with \( \lambda_1 > \lambda_2 \).}{fig:l4EllipseGeometry:l4EllipseGeometryFig6Lambda1gtLambda2}{0.2}

\begin{itemize}
\item \( \lambda_i \) tells us length of the semi-major axis.
\item Larger \( \lambda_i \) means \( \tilde{x}_i^2 \) can be bigger and still satisfy constraint \( \le 1 \).
\item Volume of ellipse if proportional to \( \sqrt{ \det P } = \sqrt{ \prod_{i = 1}^n \lambda_i } \).
\item When any \( \lambda_i \rightarrow 0 \) a dimension is lost and the volume goes to zero.  That removes the invertibility required.
\end{itemize}

Ellipses will be seen a lot in this course, since we are interested in ``bowl'' like geometries (and the ellipse is the image of a Euclidean ball).

\section{Norm ball.}

The norm ball

\begin{equation}\label{eqn:convexOptimizationLecture4:360}
\calB = \setlr{ \Bx | \Norm{\Bx} \le 1 },
\end{equation}

is a convex set for all norms.  Proof:

Take any \( \Bx, \By \in \calB \)

\begin{dmath}\label{eqn:convexOptimizationLecture4:380}
\Norm{ \theta \Bx + (1 - \theta) \By }
\le
\Abs{\theta} \Norm{ \Bx } + \Abs{1 - \theta} \Norm{ \By }
=
\theta \Norm{ \Bx } + \lr{1 - \theta} \Norm{ \By }
\lr
\theta + \lr{1 - \theta}
=
1.
\end{dmath}

This is true for any p-norm \( 1 \le p \), \( \Norm{\Bx}_p = \lr{ \sum_{i = 1}^n \Abs{x_i}^p }^{1/p} \).

%F7
%\cref{fig:l4:l4Fig7Pge1}.
\imageFigure{../figures/ece1505-convex-optimization/l4Fig7Pge1}{CAPTION: l4Fig7Pge1}{fig:l4:l4Fig7Pge1}{0.2}

The shape of a \( p < 1 \) norm unit ball is sketched in \cref{fig:l4PlessThan1:l4PlessThan1Fig7b} (lines connecting points in such a region can exit the region).

%\imageFigure{../figures/ece1505-convex-optimization/l4PlessThan1Fig7b}{CAPTION: l4PlessThan1Fig7b}{fig:l4PlessThan1:l4PlessThan1Fig7b}{0.2}
\imageFigure{../figures/ece1505-convex-optimization/l4PNormPLessThan1Fig7b}{Unit ball for \( l_{0.6} \) ``p-norm''.}{fig:l4PlessThan1:l4PlessThan1Fig7b}{0.2}

\section{Cones}

Recall that \( C \) is a cone if \( \forall \Bx \in C, \theta \ge 0, \theta \Bx \in C \).

Impt cone of PSD matrices

\begin{dmath}\label{eqn:convexOptimizationLecture4:400}
\begin{aligned}
S^n &= \setlr{ X \in \bbR^{n \times n} | X = X^\T } \\
S^n_{+} &= \setlr{ X \in S^n | \Bv^\T X \Bv \ge 0, \quad \forall v \in \bbR^n } \\
S^n_{++} &= \setlr{ X \in S^n_{+} | \Bv^\T X \Bv > 0, \quad \forall v \in \bbR^n } \\
\end{aligned}
\end{dmath}

These have respectively

\begin{itemize}
\item \( \lambda_i \in \bbR \)
\item \( \lambda_i \in \bbR_{+} \)
\item \( \lambda_i \in \bbR_{++} \)
\end{itemize}

\( S^n_{+} \) is a cone if:

\( X \in S^n_{+}\), then \( \theta X \in S^n_{+}, \quad \forall \theta \ge 0 \)

\begin{dmath}\label{eqn:convexOptimizationLecture4:420}
\Bv^\T (\theta X) \Bv
= \theta \Bv^\T \Bv
\ge 0,
\end{dmath}

since \( \theta \ge 0 \) and because \( X \in S^n_{+} \).

Shorthand:

% succeq: curly \ge
\begin{dmath}\label{eqn:convexOptimizationLecture4:440}
\begin{aligned}
X &\in S^n_{+} \implies X \succeq 0
X &\in S^n_{++} \implies X \succ 0.
\end{aligned}
\end{dmath}

Further \( S^n_{+} \) is a convex cone.

Let \( A \in S^n_{+} \), \( B \in S^n_{+} \), \( \theta_1, \theta_2 \ge 0, \theta_1 + \theta_2 = 1 \), or \( \theta_2 = 1 - \theta_1 \).

Show that \( \theta_1 A + \theta_2 B \in S^n_{+} \) :

\begin{dmath}\label{eqn:convexOptimizationLecture4:460}
\Bv^\T \lr{  \theta_1 A + \theta_2 B } \Bv
=
\theta_1 \Bv^\T A \Bv
+\theta_2 \Bv^\T B \Bv
\ge 0,
\end{dmath}

since \( \theta_1 \ge 0, \theta_2 \ge 0, \Bv^\T A \Bv \ge 0, \Bv^\T B \Bv \ge 0 \).

%F8
%\cref{fig:l4cone:l4coneFig8}.
\imageFigure{../figures/ece1505-convex-optimization/l4coneFig8}{CAPTION: l4coneFig8}{fig:l4cone:l4coneFig8}{0.2}

Inequalities:

Start with a proper cone \( K \subseteq \bbR^n \)

\begin{itemize}
\item closed, convex
\item non-empty interior (``solid'')
\item ``pointed'' (contains no lines)
\end{itemize}

%F9
%\cref{fig:l4cone:l4coneFig9}.
\imageFigure{../figures/ece1505-convex-optimization/l4coneFig9}{CAPTION: l4coneFig9}{fig:l4cone:l4coneFig9}{0.2}

The \( K \) defines a generalized inequality in \R{n} defined as ``\(\le_K\)''

Interpreting

\begin{dmath}\label{eqn:convexOptimizationLecture4:480}
\begin{aligned}
\Bx \le_K \By &\leftrightarrow \By - \Bx \in K
\Bx <_K \By   &\leftrightarrow \By - \Bx \in \interior K
\end{aligned}
\end{dmath}

%F10
%\cref{fig:l4Cone:l4ConeFig10a}.
\imageFigure{../figures/ece1505-convex-optimization/l4ConeFig10a}{CAPTION: l4ConeFig10a}{fig:l4Cone:l4ConeFig10a}{0.2}
%\cref{fig:l4inequalityK:l4inequalityKFig10b}.
\imageFigure{../figures/ece1505-convex-optimization/l4inequalityKFig10b}{CAPTION: l4inequalityKFig10b}{fig:l4inequalityK:l4inequalityKFig10b}{0.2}

Why pointed?  Want if \( \Bx \le_K \By \) and \( \By \le_K \Bx \) with this \( K \) is a half space.

%F11
%\cref{fig:l4region:l4regionFig11}.
\imageFigure{../figures/ece1505-convex-optimization/l4regionFig11}{CAPTION: l4regionFig11}{fig:l4region:l4regionFig11}{0.2}

Example:1: \( K = \bbR^n_{+}, \Bx \in \bbR^n, \By \in \bbR^n \)

%F12: K is non-negative ``orthant''.
%\cref{fig:l4:l4Fig12}.
\imageFigure{../figures/ece1505-convex-optimization/l4Fig12}{\(K\) is non-negative ``orthant''}{fig:l4:l4Fig12}{0.2}

\begin{equation}\label{eqn:convexOptimizationLecture4:500}
\Bx \le_K \By \implies \By - \Bx \in K
\end{equation}

say:

\begin{dmath}\label{eqn:convexOptimizationLecture4:520}
\begin{bmatrix}
y_1 - x_1
y_2 - x_2
\end{bmatrix}
\in R^2_{+}
\end{dmath}

Also:

\begin{dmath}\label{eqn:convexOptimizationLecture4:540}
K = R^1_{+}
\end{dmath}

%F13
%\cref{fig:l4:l4Fig13}.
\imageFigure{../figures/ece1505-convex-optimization/l4Fig13}{CAPTION: l4Fig13}{fig:l4:l4Fig13}{0.2}

(pointed, since it contains no rays)

\begin{equation}\label{eqn:convexOptimizationLecture4:560}
\Bx \le_K \By ,
\end{equation}

with respect to \( K = \bbR^n_{+} \) means that \( x_i \le y_i \) for all \( i \in [1,n]\).

Example:2: For \( K = PSD \subseteq S^n \),

\begin{equation}\label{eqn:convexOptimizationLecture4:580}
\Bx \le_K \By ,
\end{equation}

means that

\begin{equation}\label{eqn:convexOptimizationLecture4:600}
\By - \Bx \in K = S^n_{+}.
\end{equation}

\begin{itemize}
\item Difference \( \By - \Bx \) is always in \( S \)
\item check if in \( K \) by checking if all eigenvalues \( \ge 0 \).
\item \( S^n_{++} \) is the interior of \( S^n_{+} \).
\end{itemize}

Interpretation:

\begin{equation}\label{eqn:convexOptimizationLecture4:620}
\begin{aligned}
\Bx \le_K \By &\leftrightarrow \By - \Bx \in K \\
\Bx <_K \By   &\leftrightarrow \By - \Bx \in \interior K.
\end{aligned}
\end{equation}

We'll use these with vectors and matrices so often the \( K \) subscript will often be dropped, writing instead (for vectors)

\begin{equation}\label{eqn:convexOptimizationLecture4:640}
\begin{aligned}
\Bx \le \By &\leftrightarrow \By - \Bx \in \bbR^n_{+} \\
\Bx < \By   &\leftrightarrow \By - \Bx \in \interior \bbR^n_{++}
\end{aligned}
\end{equation}

and for matrices

\begin{equation}\label{eqn:convexOptimizationLecture4:660}
\begin{aligned}
\Bx \le \By &\leftrightarrow \By - \Bx \in S^n_{+} \\
\Bx < \By   &\leftrightarrow \By - \Bx \in \interior S^n_{++}.
\end{aligned}
\end{equation}

\section{Intersection}

Take the intersection of (perhaps infinitely many) sets \( S_\alpha \):

If \( S_\alpha \) is (affine,convex, conic) for all \( \alpha \in A \) then

%(upside down U)
\begin{dmath}\label{eqn:convexOptimizationLecture4:680}
\cap_\alpha S_\alpha
\end{dmath}

is

(affine,convex, conic).

To prove in homework:

\begin{equation}\label{eqn:convexOptimizationLecture4:700}
\calP = \setlr{ \Bx | \Ba_i^\T \Bx \le \Bb_i, \Bc_j^\T \Bx = \Bd_j, \quad \forall i \cdots j }
\end{equation}

This is convex since the intersection of a bunch of hyperplane and half space constraints.

\begin{enumerate}
\item If \( S \subseteq \bbR^n \) is convex then

\begin{equation}\label{eqn:convexOptimizationLecture4:720}
F(S) = \setlr{ F(\Bx) | \Bx \in S }
\end{equation}

is convex.
\item If \( S \subseteq \bbR^m \) then

\begin{equation}\label{eqn:convexOptimizationLecture4:740}
F^{-1}(S) = \setlr{ \Bx | F(\Bx) \in S }
\end{equation}

is convex.
\end{enumerate}

%F14
%\cref{fig:l4:l4Fig14}.
\imageFigure{../figures/ece1505-convex-optimization/l4Fig14}{CAPTION: l4Fig14}{fig:l4:l4Fig14}{0.2}

\EndArticle
%\EndNoBibArticle

      \section{Problems.}
         %
% Copyright � 2017 Peeter Joot.  All Rights Reserved.
% Licenced as described in the file LICENSE under the root directory of this GIT repository.
%
\makeproblem{Convex, affine, and conic hulls}{convex-optimization:problemSet1:4}{
\makesubproblem{}{convex-optimization:problemSet1:4a}
Consider the set

\begin{equation}\label{eqn:ProblemSet1Problem4:20}
\calS = \setlr{
\begin{bmatrix}
1 \\
1
\end{bmatrix},
\begin{bmatrix}
1 \\
2
\end{bmatrix}
}
\subseteq \Rm{2}.
\end{equation}

Sketch \(\conv(\calS)\), \(\affine(\calS)\) and \(\conic(\calS)\), respectively the convex, affine, and conic hulls of the set \(\calS\). Each is the union of all combinations of the respective type (convex, affine or conic).

\makesubproblem{}{convex-optimization:problemSet1:4b}

Repeat \partref{convex-optimization:problemSet1:4a} for the set

\begin{equation}\label{eqn:ProblemSet1Problem4:40}
\calS = \setlr{
\begin{bmatrix}
1 \\
1
\end{bmatrix},
\begin{bmatrix}
1 \\
2
\end{bmatrix},
\begin{bmatrix}
0.5 \\
0.25
\end{bmatrix}
}
.
\end{equation}

\makesubproblem{}{convex-optimization:problemSet1:4c}
Consider a set \(\calS\). What are the respective inclusion relations between the convex hull, the affine hull, and the conic hull of \(\calS\). I.e., which of these three sets are always subsets of the other, regardless of the original \(\calS\)?
} % makeproblem

\makeanswer{convex-optimization:problemSet1:4}{
\withproblemsetsParagraph{
\makeSubAnswer{}{convex-optimization:problemSet1:4a}

\imageThreeFiguresOneLine
{../figures/ece1505-convex-optimization/ps1p4aFig1}
{../figures/ece1505-convex-optimization/ps1p4aFig2}
{../figures/ece1505-convex-optimization/ps1p4aFig3}
{\( \conv(S), \affine(S), \conic(S) \)}{fig:ps1p4a}{scale=0.3}

\makeSubAnswer{}{convex-optimization:problemSet1:4b}
\imageThreeFiguresOneLine
{../figures/ece1505-convex-optimization/ps1p4bFig1}
{../figures/ece1505-convex-optimization/ps1p4bFig2}
{../figures/ece1505-convex-optimization/ps1p4bFig3}
{\( \conv(S), \affine(S), \conic(S) \)}{fig:ps1p4b}{scale=0.4}

\makeSubAnswer{}{convex-optimization:problemSet1:4c}

Recall that the respective sets in question are:

\begin{equation}\label{eqn:ProblemSet1Problem4:60}
\begin{aligned}
\conv(S) &= \setlr{ \sum_{i=1}^n \theta_i \Bx_i | \Bx_i \in S, \theta_i \ge 0, \sum_{i=1}^n \theta_i = 1 } \\
\affine(S) &= \setlr{ \sum_{i=1}^n \theta_i \Bx_i | \Bx_i \in S, \theta_i \in \bbR, \sum_{i=1}^n \theta_i = 1 } \\
\conic(S) &= \setlr{ \sum_{i=1}^n \theta_i \Bx_i | \Bx_i \in S, \theta_i \ge 0 }.
\end{aligned}
\end{equation}

The convex hull can always is always a subset of either of the affine or conic hulls, since it takes that same set and imposes additional restrictions on it

\begin{equation}\label{eqn:ProblemSet1Problem4:80}
\begin{aligned}
\conv(S) &\subseteq \conic(S) \\
\conv(S) &\subseteq \affine(S).
\end{aligned}
\end{equation}

} % redaction
} % answer
 % Convex, affine, and conic hulls
         \shipoutAnswer
         %
% Copyright � 2017 Peeter Joot.  All Rights Reserved.
% Licenced as described in the file LICENSE under the root directory of this GIT repository.
%
\makeoproblem{Distance between two parallel hyperplanes}{convex-optimization:problemSet1:5}{
\citep{boyd2004convex} pr. 2.5
}{
What is the distance between two parallel hyperplanes
\( \setlr{ \Bx \in \bbR^n | \Ba^\T \Bx = b_1 } \)
and
\( \setlr{ \Bx \in \bbR^n | \Ba^\T \Bx = b_2 } \).
} % makeproblem

\makeanswer{convex-optimization:problemSet1:5}{
\withproblemsetsParagraph{

The geometry of the situation is illustrated in \cref{fig:ps1HyperplaneSeparation:ps1HyperplaneSeparationFig1}.

\imageFigure{../figures/ece1505-convex-optimization/ps1HyperplaneSeparationFig1}{Hyperplane separation.}{fig:ps1HyperplaneSeparation:ps1HyperplaneSeparationFig1}{0.35}

As illustrated the two hyperplanes are
\( \setlr{ \Bx \in \bbR^n | \Ba^\T (\Bx - \Bx_0) = 0 } \)
and
\( \setlr{ \Bx \in \bbR^n | \Ba^\T (\By - \By_0) = 0 } \), where \( \Bx_0, \By_0 \) are a specific points on the respective planes.

The vector separating the planes, for some point \( \Bx_0 \) int the top plane, must be

\begin{dmath}\label{eqn:ProblemSet1Problem5:20}
\By_0 + d \frac{\Ba}{\Norm{\Ba}} = \Bx.
\end{dmath}

Dotting with \( \Ba \) gives

\begin{dmath}\label{eqn:ProblemSet1Problem5:40}
\Ba^\T \Bx
=
\Ba^\T \By_0 + d \Ba^\T \frac{\Ba}{\Norm{\Ba}}
=
\Ba^\T \By_0 + d \Norm{\Ba},
\end{dmath}

or
\begin{dmath}\label{eqn:ProblemSet1Problem5:60}
d = \frac{ \Ba^\T \Bx - \Ba^\T \By_0 }{\Norm{\Ba} }.
\end{dmath}

In terms of \( b_1 = \Ba^\T \Bx_0, b_2 = \Ba^\T \By_0 \), the absolute distance between the hyperplanes is therefore

\boxedEquation{eqn:ProblemSet1Problem5:80}{
\Abs{d}
= \frac{ \Abs{ b_1 - b_2 } }{\Norm{\Ba} }.
}
} % redaction
} % answer
 % Distance between two parallel hyperplanes
         \shipoutAnswer
   \mychapter{Sets, epigraphs, quasi-convexity, and sublevel sets.}
      %
% Copyright � 2017 Peeter Joot.  All Rights Reserved.
% Licenced as described in the file LICENSE under the root directory of this GIT repository.
%
%\chapter{Sets, epigraphs, quasi-convexity, and sublevel sets}
\label{chap:convexOptimization5}

%%% from last class:
%%\section{Clarification}
%%
%%\begin{equation}\label{eqn:convexOptimizationLecture5:20}
%%\Bx \le_K \By \mathif \By - \Bx \in K
%%\end{equation}
%%
%%%\cref{fig:l5:l5Fig1c}.
%%\imageFigure{../figures/ece1505-convex-optimization/l5Fig1c}{}{fig:l5:l5Fig1c}{0.2}
%%
\section{Operations that preserve convexity}

If \( S_\alpha \) is convex \( \forall \alpha \in A \), then
%
\begin{equation}\label{eqn:convexOptimizationLecture5:40}
\cup_{\alpha \in A} S_\alpha,
\end{equation}
%
is convex.

Example:
%
\begin{equation}\label{eqn:convexOptimizationLecture5:60}
F(\Bx) = A \Bx + \Bb
\end{equation}
%
\begin{equation}\label{eqn:convexOptimizationLecture5:80}
\begin{aligned}
\Bx &\in \bbR^n \\
A &\in \bbR^{m \times n} \\
F &: \bbR^{n} \rightarrow \bbR^m \\
\Bb &\in \bbR^m
\end{aligned}
\end{equation}
%

\begin{enumerate}[(i)]
\item
If \( S \in \bbR^n \) is convex, then
%
\begin{equation}\label{eqn:convexOptimizationLecture5:100}
F(S) = \setlr{ F(\Bx) | \Bx \in S }
\end{equation}
%
is convex if \( F \) is affine.
\item

If \( S \in \bbR^m \) is convex, then
%
\begin{equation}\label{eqn:convexOptimizationLecture5:120}
F^{-1}(S) = \setlr{ \Bx | F(\Bx) \in S }
\end{equation}
%
is convex.
\end{enumerate}

Example:
%
\begin{equation}\label{eqn:convexOptimizationLecture5:140}
\setlr{ \By | \By = A \Bx + \Bb, \Norm{\Bx} \le 1}
\end{equation}
%
is convex.  Here \( A \Bx + \Bb \) is an affine function (\(F(\Bx)\).  This is the image of a (convex) unit ball, through an affine map.

Earlier saw when defining ellipses
%
\begin{equation}\label{eqn:convexOptimizationLecture5:160}
\By = P^{1/2} \Bx + \Bx_c
\end{equation}
%
Example :
%
\begin{equation}\label{eqn:convexOptimizationLecture5:180}
\setlr{ \Bx | \Norm{ A \Bx + \Bb } \le 1 },
\end{equation}
%
is convex.  This can be seen by writing
%
\begin{equation}\label{eqn:convexOptimizationLecture5:200}
\setlr{ \Bx | \Norm{ A \Bx + \Bb } \le 1 }
=
\setlr{ \Bx | \Norm{ F(\Bx) } \le 1 }
=
\setlr{ \Bx | F(\Bx) \in \calB  },
\end{equation}
%
where \( \calB = \setlr{ \By | \Norm{\By} \le 1 } \).  This is the pre-image (under \(F()\)) of a unit norm ball.

Example:
%
\begin{equation}\label{eqn:convexOptimizationLecture5:220}
\setlr{ \Bx \in \bbR^n | x_1 A_1 + x_2 A_2 + \cdots x_n A_n \le \calB }
\end{equation}
%
where \( A_i \in S^m \) and \( \calB \in S^m \), and the inequality is a matrix inequality.  This is a convex set.  The constraint is a ``linear matrix inequality'' (LMI).

This has to do with an affine map:
%
\begin{equation}\label{eqn:convexOptimizationLecture5:240}
F(\Bx) = B - 1 x_1 A_1 - x_2 A_2 - \cdots x_n A_n \ge 0
\end{equation}
%
(positive semi-definite inequality).  This is a mapping
%
\begin{equation}\label{eqn:convexOptimizationLecture5:480}
F : \bbR^n \rightarrow S^m,
\end{equation}
%
since all \( A_i \) and \( B \) are in \( S^m \).

This \( F(\Bx) = B - A(\Bx) \) is a constant and a factor linear in x, so is affine.  Can be written
%
\begin{equation}\label{eqn:convexOptimizationLecture5:260}
\setlr{ \Bx | B - A(\Bx) \ge 0 }
=
\setlr{ \Bx | B - A(\Bx) \in S^m_{+} }
\end{equation}
%
This is a pre-image of a cone of PSD matrices, which is convex.  Therefore, this is a convex set.

\section{Separating hyperplanes}

\maketheorem{Separating hyperplanes}{thm:convexOptimizationLecture5:280}{

If \( S, T \subseteq \bbR^n \) are convex and disjoint
i.e. \( S \cup T = 0\), then
there exists on \( \Ba \in \bbR^n \) \( \Ba \ne 0 \) and a \( \Bb \in \bbR^n \) such that
%
\begin{equation*}
\Ba^\T \Bx \ge \Bb \, \forall \Bx \in S
\end{equation*}
%
and
\begin{equation*}
\Ba^\T \Bx < \Bb \,\forall \Bx \in T.
\end{equation*}
} % theorem

An example of a hyperplanes that separates two sets and two sets that are not separable is sketched in
\cref{fig:l5:l5Fig2a}.

\imageTwoFigures
{../figures/ece1505-convex-optimization/l5Fig2a}
{../figures/ece1505-convex-optimization/l5Fig2b}
{separable and non-separable sets}
{fig:l5:l5Fig2a}
{scale=0.1}

Proof in the book.

\maketheorem{Supporting hyperplane}{thm:convexOptimizationLecture5:300}{
If \( S \) is convex then \( \forall x_0 \in \partial S = \closure(S) \ \interior(S) \), where
\( \partial S \) is the boundary of \( S \), then \( \exists \) an \( \Ba \ne 0 \in \bbR^n \) such that \( \Ba^\T \Bx \le \Ba^\T x_0 \, \forall \Bx \in S \).

} % theorem

Here \( \ \) denotes ``without''.

An example is sketched in \cref{fig:l5:l5Fig3}, for which

\imageFigure{../figures/ece1505-convex-optimization/l5Fig3}{Supporting hyperplane.}{fig:l5:l5Fig3}{0.2}

\begin{itemize}
\item The vector \( \Ba \) perpendicular to tangent plane.
\item inner product \( \Ba^\T (\Bx - \Bx_0) \le 0 \).
\end{itemize}

A set with a supporting hyperplane is sketched in
\cref{fig:l5:l5Fig4a}, whereas
\cref{fig:l5:l5Fig4b}
shows that there is not necessarily a unique supporting hyperplane at any given point, even if \( S \) is convex.

\imageFigure{../figures/ece1505-convex-optimization/l5Fig4a}{Set with supporting hyperplane.}{fig:l5:l5Fig4a}{0.2}

\imageFigure{../figures/ece1505-convex-optimization/l5Fig4b}{No unique supporting hyperplane possible.}{fig:l5:l5Fig4b}{0.2}

\section{basic definitions of convex functions}

\maketheorem{Convex functions}{thm:convexOptimizationLecture5:320}{
If \( F : \bbR^n \rightarrow \bbR \) is defined on a convex domain (i.e. \( \dom F \subseteq \bbR^n \) is a convex set), then \( F \) is convex if \( \forall \Bx, \By \in \dom F \), \( \forall \theta \in [0,1] \in \bbR \)
%
\begin{equation}\label{eqn:convexOptimizationLecture5:340}
F( \theta \Bx + (1-\theta) \By \le \theta F(\Bx) + (1-\theta) F(\By)
\end{equation}
} % theorem

An example is sketched in \cref{fig:l5:l5Fig5}.

\imageFigure{../figures/ece1505-convex-optimization/l5Fig5}{Example of convex function.}{fig:l5:l5Fig5}{0.4}

Remarks

\begin{itemize}
\item Require \( \dom F \) to be a convex set.  This is required so that the function at the point \( \theta u + (1-\theta) v \) can be evaluated.  i.e. so that \( F(\theta u + (1-\theta) v) \) is well defined.  Example: \( \dom F = (-\infty, 0] \cup [1, \infty) \) is not okay, because a linear combination in \( (0,1) \) would be undesirable.
\item Parameter \( \theta \) is ``how much up'' the line segment connecting \( (u, F(u) \) and \( (v, F(v) \).  This line segment never below the bottom of the bowl.
The function is \underlineAndIndex{concave}, if \( -F \) is convex.
i.e. If the convex function is flipped upside down.  That is
%
\begin{equation}\label{eqn:convexOptimizationLecture5:360}
F(\theta \Bx + (1-\theta) \By ) \ge \theta F(\Bx) + (1-\theta) F(\By) \,\forall \Bx,\By \in \dom F, \theta \in [0,1].
\end{equation}
\item a ``strictly'' convex function means \(\forall \theta \in [0,1] \)
%
\begin{equation}\label{eqn:convexOptimizationLecture5:380}
F(\theta \Bx + (1-\theta) \By ) < \theta F(\Bx) + (1-theta) F(\By).
\end{equation}
\item Strictly concave function \( F \) means \( -F \) is strictly convex.
\item Examples:

%\cref{fig:l5:l5Fig6a}.
\imageFigure{../figures/ece1505-convex-optimization/l5Fig6a}{Not convex or concave.}{fig:l5:l5Fig6a}{0.2}
%\cref{fig:l5:l5Fig6b}.
\imageFigure{../figures/ece1505-convex-optimization/l5Fig6b}{Not strictly convex}{fig:l5:l5Fig6b}{0.2}
%F6a: not convex or concave.
%F6b: not strictly convex (understand).
\end{itemize}

\makedefinition{Epigraph of a function}{dfn:convexOptimizationLecture5:400}{

The epigraph \( \epi F \) of a function \( F : \bbR^n \rightarrow \bbR \) is
%
\begin{equation*}
\epi F = \setlr{ (\Bx,t) \in \bbR^{n +1} | \Bx \in \dom F, t \ge F(\Bx) },
\end{equation*}
%
where \( \Bx \in \bbR^n, t \in \bbR \).
} % definition

%F7
%\cref{fig:l5:l5Fig7}.
\imageFigure{../figures/ece1505-convex-optimization/l5Fig7}{Epigraph.}{fig:l5:l5Fig7}{0.3}

\maketheorem{Convexity and epigraph.}{thm:convexOptimizationLecture5:420}{
If \( F \) is convex implies \( \epi F \) is a convex set.
} % theorem

Proof:

For convex function, a line segment connecting any 2 points on function is above the function.  i.e. it is \( \epi F \).

Many authors will go the other way around, showing \cref{dfn:convexOptimizationLecture5:400} from \cref{thm:convexOptimizationLecture5:420}.  That is:

Pick any 2 points in \( \epi F \), \( (\Bx,\mu) \in \epi F\) and \( (\By, \nu) \in \epi F \).  Consider convex combination
%
\begin{equation}\label{eqn:convexOptimizationLecture5:420}
\theta( \Bx, \mu ) + (1-\theta) (\By, \nu) =
(\theta \Bx  (1-\theta) \By, \theta \mu  (1-\theta) \nu )
\in \epi F,
\end{equation}
%
since \( \epi F \) is a convex set.

By definition of \( \epi F \)
%
\begin{equation}\label{eqn:convexOptimizationLecture5:440}
F( \theta \Bx  (1-\theta) \By ) \le \theta \mu  (1-\theta) \nu.
\end{equation}
%
Picking \( \mu = F(\Bx), \nu = F(\By) \) gives
\begin{equation}\label{eqn:convexOptimizationLecture5:460}
F( \theta \Bx  (1-\theta) \By ) \le \theta F(\Bx)  (1-\theta) F(\By).
\end{equation}
%
\section{Extended value function}

Sometimes convenient to work with ``extended value function''
%
\begin{equation}\label{eqn:convexOptimizationLecture5:500}
\tilde{F}(\Bx) =
\left\{
\begin{array}{l l}
F(\Bx) & \quad \mbox{If \( \Bx \in \dom F\)} \\
\infty & \quad \mbox{otherwise.}
\end{array}
\right.
\end{equation}
%
Examples:

\begin{itemize}
\item Linear (affine) functions (\cref{fig:l5:l5Fig8}) are both convex and concave.
\imageFigure{../figures/ece1505-convex-optimization/l5Fig8}{Linear functions.}{fig:l5:l5Fig8}{0.2}
\item \( x^2 \) is convex, sketched in \cref{fig:l5:l5Fig9}.
\imageFigure{../figures/ece1505-convex-optimization/l5Fig9}{Convex (quadratic.)}{fig:l5:l5Fig9}{0.2}
\item \( \log x, \dom F = \bbR_{+} \) concave, sketched in \cref{fig:l5:l5Fig10}.
\imageFigure{../figures/ece1505-convex-optimization/l5Fig10}{Concave (logarithm.)}{fig:l5:l5Fig10}{0.2}
\item \( \Norm{\Bx} \) is convex.  \( \Norm{ \theta \Bx + (1-\theta) \By } \le \theta \Norm{ \Bx } + (1-\theta) \Norm{\By } \).
\item \( 1/x \) is convex on \( \setlr{ x | x > 0 } = \dom F \), and concave on \( \setlr{ x | x < 0 } = \dom F \).
%
\begin{equation}\label{eqn:convexOptimizationLecture5:520}
\tilde{F}(x) =
\left\{
\begin{array}{l l}
\inv{x} & \quad \mbox{If \( x > 0 \)} \\
\infty & \quad \mbox{else.}
\end{array}
\right.
\end{equation}
\end{itemize}

\makedefinition{Sublevel}{dfn:convexOptimizationLecture5:540}{

The sublevel set of a function \( F : \bbR^n \rightarrow \bbR \) is
%
\begin{equation*}
C(\alpha) = \setlr{ \Bx \in \dom F | F(\Bx) \le \alpha }
\end{equation*}
} % definition

%F11
%\cref{fig:l5:l5Fig11a}.
\imageFigure{../figures/ece1505-convex-optimization/l5Fig11a}{Convex sublevel.}{fig:l5:l5Fig11a}{0.2}
%\cref{fig:l5:l5Fig11b}.
\imageFigure{../figures/ece1505-convex-optimization/l5Fig11b}{Non-convex sublevel.}{fig:l5:l5Fig11b}{0.2}

\maketheorem{}{thm:convexOptimizationLecture5:560}{
If \( F \) is convex then \( C(\alpha) \) is a convex set \( \forall \alpha \).
} % theorem

This is not an if and only if condition, as illustrated in \cref{fig:l5:l5Fig12}.

\imageFigure{../figures/ece1505-convex-optimization/l5Fig12}{Convex sublevel does not imply convexity.}{fig:l5:l5Fig12}{0.2}

There \( C(\alpha) \) is convex, but the function itself is not.

Proof:

Since \( F \) is convex, then \( \epi F \) is a convex set.

\begin{itemize}
\item
Let
%
\begin{equation}\label{eqn:convexOptimizationLecture5:580}
\calA = \setlr{ (\Bx,t) | t = \alpha }
\end{equation}
%
is a convex set.

\item
\( \calA \cap \epi F \)

is a convex set since it is the intersection of convex sets.

\item Project \( \calA \cap \epi F \) onto \R{n} (i.e. domain of \( F \) ).  The projection is an affine mapping.  Image of a convex set through affine mapping is a convex set.
\end{itemize}

\makedefinition{Quasi-convex.}{dfn:convexOptimizationLecture5:600}{

A function is quasi-convex if \underline{all} of its sublevel sets are convex.
} % definition

\section{Composing convex functions}

Properties of convex functions:

\begin{itemize}
\item If \( F \) is convex, then \( \alpha F \) is convex \( \forall \alpha > 0 \).
\item If \( F_1, F_2 \) are convex, then the sum \( F_1 + F_2 \) is convex.
\item If \( F \) is convex, then \( g(\Bx) = F(A \Bx + \Bb) \) is convex \( \forall \Bx \in \setlr{ \Bx | A \Bx + \Bb \in \dom F } \).
\end{itemize}

Note: for the last
%
\begin{equation}\label{eqn:convexOptimizationLecture5:620}
\begin{aligned}
g &: \bbR^m \rightarrow \bbR  \\
F &: \bbR^n \rightarrow \bbR  \\
\Bx &\in \bbR^m \\
A &\in \bbR^{n \times m} \\
\Bb &\in \bbR^n
\end{aligned}
\end{equation}
%
Proof (of last):
%
\begin{equation}\label{eqn:convexOptimizationLecture5:640}
\begin{aligned}
g( \theta \Bx + (1-\theta) \By )
&= F( \theta (A \Bx + \Bb) + (1-\theta) (A \By + \Bb) ) \\
&\le \theta F( A \Bx + \Bb) + (1-\theta) F (A \By + \Bb) \\
&= \theta g(\Bx) + (1-\theta) g(\By).
\end{aligned}
\end{equation}

      \section{Problems.}
         %
% Copyright � 2017 Peeter Joot.  All Rights Reserved.
% Licenced as described in the file LICENSE under the root directory of this GIT repository.
%
\makeproblem{Ellipses, eigenvalues, eigenvectors, and volume}{convex-optimization:problemSet1:6}{
Make neat and clearly-labelled sketches of the ellipsoid \( \calE = \setlr{\Bx | (\Bx - \Bx_c)^\T P^{-1} (\Bx - \Bx_c) = 1} \) for the following sets of parameters:
\makesubproblem{}{convex-optimization:problemSet1:6a}

Center \( \Bx_c =
\begin{bmatrix}
0 \\ 0
\end{bmatrix} \)
and \( P =
\begin{bmatrix}
1.5 & -0.5 \\
-0.5 & 1.5
\end{bmatrix} \).

\makesubproblem{}{convex-optimization:problemSet1:6b}
Center \( \Bx_c =
\begin{bmatrix}
1 \\ -2
\end{bmatrix} \)
and \( P =
\begin{bmatrix}
3 & 1 \\
1 & 3
\end{bmatrix} \).

\makesubproblem{}{convex-optimization:problemSet1:6c}
Center \( \Bx_c =
\begin{bmatrix}
-2 \\ 1
\end{bmatrix} \)
and \( P =
\begin{bmatrix}
9 & -2 \\
-2 & 6
\end{bmatrix} \).

For each part (a)-(c) also compute each pair of eigenvalues and corresponding eigenvectors.

\makesubproblem{}{convex-optimization:problemSet1:6d}
Recall that the most geometrically meaningful property of the determinant of a square real matrix \(A\) is that its magnitude \( \Abs{\det A} \) is equal to the volume of the parallelepiped \( \calP \) formed by applying \( A \) to the unit cube \( \calC = \setlr{x|0 \le x \le 1} \).
(Recall that since \( \Bx \in \Rm{n} \) we interpret the inequalities coordinate-wise, i.e., \( 0 \le x_i \le 1 \) for all \( i = 1, \cdots, n \).)
In other words, if \( \calP = \setlr{ A \Bx| \Bx \in \calC} \) then \( \Abs{\det(A)}\) is equal to the volume of \( \calP\).
Furthermore, recall that the determinant of a matrix is zero if any of its eigenvalues are zero.
Explain how to interpret this latter fact in terms of the interpretation of \( \Abs{\det(A)} \) as the volume of \( \calP\).
} % makeproblem

\makeanswer{convex-optimization:problemSet1:6}{
\makeSubAnswer{}{convex-optimization:problemSet1:6a}

The eigenvalues were found to be \( \setlr{2,1} \) with respective eigenvectors

\begin{dmath}\label{eqn:ProblemSet1Problem6:20}
\begin{aligned}
\Be_1 &= \inv{\sqrt{2}}
\begin{bmatrix}
1 \\
-1
\end{bmatrix} \\
\Be_2 &= \inv{\sqrt{2}}
\begin{bmatrix}
1 \\
1
\end{bmatrix}.
\end{aligned}
\end{dmath}

The ellipsoid is plotted in \cref{fig:ps1p6a:ps1p6aFig2}.

\imageFigure{../figures/ece1505-convex-optimization/ps1p6aFig2}{}{fig:ps1p6a:ps1p6aFig2}{0.5}

\makeSubAnswer{}{convex-optimization:problemSet1:6b}

The eigenvalues were found to be \( \setlr{4,2} \) with respective eigenvectors

\begin{dmath}\label{eqn:ProblemSet1Problem6:40}
\begin{aligned}
\Be_1 &= \inv{\sqrt{2}}
\begin{bmatrix}
1 \\
1
\end{bmatrix} \\
\Be_2 &= \inv{\sqrt{2}}
\begin{bmatrix}
1 \\
-1
\end{bmatrix}.
\end{aligned}
\end{dmath}

The ellipsoid is plotted in \cref{fig:ps1p6b:ps1p6bFig3}.

\imageFigure{../figures/ece1505-convex-optimization/ps1p6bFig3}{}{fig:ps1p6b:ps1p6bFig3}{0.5}

\makeSubAnswer{}{convex-optimization:problemSet1:6c}

The eigenvalues were found to be \( \setlr{10,5} \) with respective eigenvectors

\begin{dmath}\label{eqn:ProblemSet1Problem6:60}
\begin{aligned}
\Be_1 &= \inv{\sqrt{5}}
\begin{bmatrix}
2 \\
-1
\end{bmatrix} \\
\Be_2 &= \inv{\sqrt{5}}
\begin{bmatrix}
1 \\
2
\end{bmatrix}.
\end{aligned}
\end{dmath}

The ellipsoid is plotted in \cref{fig:ps1p6c:ps1p6cFig4}.

\imageFigure{../figures/ece1505-convex-optimization/ps1p6cFig4}{}{fig:ps1p6c:ps1p6cFig4}{0.5}

\makeSubAnswer{}{convex-optimization:problemSet1:6d}

Some of the abstraction of the statement

\begin{equation}\label{eqn:ProblemSet1Problem6:80}
\calP = \setlr{ A \Bx| \Bx \in \calC }
\end{equation}

can be removed by expressing the matrix in terms of its columns \( A = [ \Ba_1 \Ba_2 \cdots \Ba_n ] \), and the vector in terms of coordinates \( \Bx = \sum x_i \Be_i\, x_i \in [0,1] \), so

\begin{equation}\label{eqn:ProblemSet1Problem6:100}
\calP = \setlr{ \sum \Ba_i x_i | x_i \in [0,1] },
\end{equation}

which shows that this set \( \calP \) is the span of the columns of the matrix \( A \) over the unit cube.  The span of the columns takes a particularly simple form if the linear transformation represented by the matrix is represented in a basis for which that transformation takes its
Jordan canonical form \citep{damiano1988course}.
This is a basis for which the matrix representation of the linear transformation is either completely diagonal, or in specific upper triangular form with zeros everywhere but the superdiagonal (and only ones or zeros on the superdiagonal).  Such a similarity transformation

\begin{dmath}\label{eqn:ProblemSet1Problem6:140}
A = E J E^{-1},
\end{dmath}

is determinant preserving, since

\begin{dmath}\label{eqn:ProblemSet1Problem6:160}
\Abs{ A }
= \Abs{ E } \Abs{ J } \Abs{ E^{-1} }
= \Abs{ E } \Abs{ J } \inv{ \Abs{E} }
= \Abs{ J }.
\end{dmath}

A couple examples are

\begin{dmath}\label{eqn:ProblemSet1Problem6:120}
\begin{bmatrix}
3 & 1 \\
0 & 3
\end{bmatrix}, \quad
\begin{bmatrix}
1 & 0 \\
0 & 2
\end{bmatrix}, \quad
\begin{bmatrix}
\epsilon & 0 & 0 \\
0        & 2 & 0 \\
0        & 0 & 3
\end{bmatrix}, \quad
\begin{bmatrix}
\epsilon & 0 & 0 \\
0        & 2 & 1 \\
0        & 0 & 2
\end{bmatrix},
\end{dmath}

for which the respective sets \( \calP \) are sketched in \cref{fig:ps1p6d:ps1p6dFig1}.

\imageFourFiguresTwoLines
{../figures/ece1505-convex-optimization/ps1p6dFig1}
{../figures/ece1505-convex-optimization/ps1p6dFig2}
{../figures/ece1505-convex-optimization/ps1p6dFig3}
{../figures/ece1505-convex-optimization/ps1p6dFig4}
{Parallelepiped volumes}{fig:ps1p6d:ps1p6dFig1}{scale=0.15}

Note that the parallelograms are altered by such a similarity transformation, but the (oriented area) given by the determinant is invariant.  This is illustrated in \cref{fig:ps1p6d:ps1p6dParallelogramsFig1} for

\begin{dmath}\label{eqn:ProblemSet1Problem6:180}
\begin{bmatrix}
3 & 1 \\
-1 & 1
\end{bmatrix}
=
\begin{bmatrix}
1 & 1 \\
-1 & 1
\end{bmatrix}
\begin{bmatrix}
1 & 1 \\
0 & 1
\end{bmatrix}
\begin{bmatrix}
\inv{2} & -\inv{2} \\
\inv{2} & \inv{2}
\end{bmatrix},
\end{dmath}

for which the oriented area given by the determinant is unity in both cases.

\imageFigure{../figures/ece1505-convex-optimization/ps1p6dParallelogramsFig1}{Parallelograms after similarity transformation}{fig:ps1p6d:ps1p6dParallelogramsFig1}{0.3}

In a basis for which the matrix has its Jordan form, if any one of the eigenvalues is zero, then the volume of the parallelepiped in that basis will be zero, since the parallelepiped cannot have any volume along that dimension.  In the two three dimensional examples above, one of the eigenvalues was picked to be \( \epsilon \), a quantity that can be made arbitrarily small.  When any eigenvector is decreased to zero in that matrix's Jordan basis representation, it is clear that the corresponding parallelepiped volume also goes to zero, as does the determinant (which is just the product of the diagonal elements in this representation).

In summary, if \( n - r \ne 0 \) is the number of zero eigenvalues of a matrix (where \( n \) is the dimension, and \( r \) is the rank), then
parallelepiped represented by the span of the columns of a matrix over the unit cube has no volume (and zero determinant), because
such a parallelepiped will have no height along \( n - r \) dimensions.
}
 % Ellipses, eigenvalues, eigenvectors, and volume
         \shipoutAnswer
         %
% Copyright � 2017 Peeter Joot.  All Rights Reserved.
% Licenced as described in the file LICENSE under the root directory of this GIT repository.
%
\makeproblem{Proving convexity-preserving operations}{convex-optimization:problemSet1:7}{
\makesubproblem{}{convex-optimization:problemSet1:7a}
Prove that the set \( \calS \) resulting from taking the intersection of a set of convex sets \( \calS_\alpha \) is itself a convex set.
I.e., \( \calS = \cap_\alpha \calS_\alpha \) is a convex set when all the \( \calS_\alpha \) are convex sets.

\makesubproblem{}{convex-optimization:problemSet1:7b}
Consider any affine function \( f : \Rm{n} \rightarrow \Rm{m} \) and convex set \( \calS \subseteq \Rm{n}\).
Prove that the image of \( \calS \) under \( f \), i.e., \( f(\calS) = \setlr{f(\Bx)| \Bx \in \calS}\), is a convex set.

\makesubproblem{}{convex-optimization:problemSet1:7c}
Consider any affine function \( f : \Rm{n} \rightarrow \Rm{m} \) and convex set \( \calS \subseteq \Rm{m} \).
Prove that the inverse (or pre-) image of \( \calS \) under \(f\), i.e., \( f^{-1}(\calS) = \setlr{\Bx|f(\Bx) \in \calS}\), is a convex set.
} % makeproblem

\makeanswer{convex-optimization:problemSet1:7}{
\makeSubAnswer{}{convex-optimization:problemSet1:7a}

TODO.
\makeSubAnswer{}{convex-optimization:problemSet1:7b}

TODO.
\makeSubAnswer{}{convex-optimization:problemSet1:7c}

TODO.
}
 % Proving convexity-preserving operations
         \shipoutAnswer
         %
% Copyright � 2017 Peeter Joot.  All Rights Reserved.
% Licenced as described in the file LICENSE under the root directory of this GIT repository.
%
\makeoproblem{Expanded and restricted sets}{convex-optimization:problemSet1:8}{
\citep{boyd2004convex} pr. 2.14(a)}{
Let \( S \subseteq \bbR^n \), and let \( \Norm{ \cdot } \) be a norm on \R{n}.

For \( a \ge 0 \) we define \( S_a \) as \( \setlr{ \Bx | \dist(\Bx, S) \le a } \), where \( \dist(\Bx, S) = \inf_{\By \in S} \Norm{ \Bx - \By }\).  We refer to \( S_a \) as \( S \) expanded or extended by \( a \).  Show that if \( S \) is convex, then \( S_a \) is convex.
} % makeproblem

\makeanswer{convex-optimization:problemSet1:8}{

The geometry of this set extension is sketched in \cref{fig:ps1p8:ps1p8Fig1}.

\imageFigure{../figures/ece1505-convex-optimization/ps1p8Fig1}{Expanded convex set.}{fig:ps1p8:ps1p8Fig1}{0.3}

The sphere of radius \( a \) centered on a point \( \By \in S \), as sketched in \cref{fig:ps1p8:ps1p8Fig2}, is

\imageFigure{../figures/ece1505-convex-optimization/ps1p8Fig2}{Sphere of radius \( a \).}{fig:ps1p8:ps1p8Fig2}{0.2}

\begin{dmath}\label{eqn:ProblemSet1Problem8:20}
\setlr{ \By + \theta a \rcap},
\end{dmath}

where \( \theta \in [0,1] \) and \( \rcap \) is a unit vector directed outwards from \( \By \).  If \( \Bx \) is a point on the exterior of that sphere, the set of points in the sphere can be written

\begin{dmath}\label{eqn:ProblemSet1Problem8:40}
\setlr{ \By + \theta a \rcap}
=
\setlr{ \By + \theta a \frac{\Bx - \By}{\Norm{\Bx - \By}} }
=
\setlr{ \By + \theta (\Bx - \By) }
=
\setlr{ \theta \Bx + (1-\theta) \By }.
\end{dmath}

The union of such spheres, over all \( \By \in S \), is precisely the set \( S_a \), seen to be the convex combinations of points \( \Bx \in S_a, \By \in S \), or \( \Bx, \By \in S \).
}
 % Expanded and restricted sets
         \shipoutAnswer
   \mychapter{First and second order conditions.}
      %
% Copyright � 2017 Peeter Joot.  All Rights Reserved.
% Licenced as described in the file LICENSE under the root directory of this GIT repository.
%
\input{../latex/blogpost.tex}
\renewcommand{\basename}{convexOptimization6}
\renewcommand{\dirname}{notes/ece1505/}
\newcommand{\keywords}{ECE1505H}
\input{../latex/peeter_prologue_print2.tex}

\usepackage{ece1505}
\usepackage{peeters_braket}
\usepackage{peeters_layout_exercise}
\usepackage{peeters_figures}
\usepackage{macros_qed}
\usepackage{mathtools}
\usepackage{siunitx}
\usepackage{enumerate}

\beginArtNoToc
\generatetitle{ECE1505H Convex Optimization.  Lecture 6: First and second order conditions.  Taught by Prof.\ Stark Draper}
%\chapter{First and second order conditions}
\label{chap:convexOptimization6}

\paragraph{Disclaimer}

Peeter's lecture notes from class.  These may be incoherent and rough.

These are notes for the UofT course ECE1505H, Convex Optimization, taught by Prof. Stark Draper, covering \textchapref{{1}} \citep{boyd2004convex} content.

\paragraph{Today}

\begin{itemize}
\item First and second order conditions for convexity of differentiable functions.
\item Consequences of convexity: local and global optimizality.
\item Properties.
\end{itemize}

\paragraph{Quasi-convex}

\( F_1 \) and \( F_2 \) convex implies \( \max( F_1, F_2) \) convex.

F1

Note that \( \min(F_1, F_2) \) is NOT convex.

If \( F : \bbR^n \rightarrow \bbR \) is convex, then
\( F( \Bx_0 + t \Bv ) \) is convex in \( t\,\forall t \in \bbR, \Bx_0 \in \bbR^n, \Bv \in \bbR^n \), provided \( \Bx_0 + t \Bv \in \dom F \).

Idea: Restrict to a line (line segment) in \( \dom F \).  Take a cross section or slice through \( F \) alone the line.  If the result is a 1D convex function for all slices, then \( F \) is convex.

This is nice since it allows for checking for convexity, and is also nice numerically.  Attempting to test a given data set for non-convexity with some random lines can help disprove convexity.   However, to show that \( F \) is convex it is required to test all possible slices (which isn't possible numerically, but is in some circumstances possible analytically).

\paragraph{Differentiable (convex) functions}

\makedefinition{First order condition}{dfn:convexOptimizationLecture6:20}{
If

\begin{equation*}
F : \bbR^n \rightarrow \bbR
\end{equation*}

is differentiable, then \( F \) is convex iff \( \dom F \) is a convex set and \( \forall \Bx, \Bx_0 \in \dom F \)

\begin{equation*}
F(\Bx) \ge F(\Bx_0) + \lr{\spacegrad F(\Bx_0)}^\T (\Bx - \Bx_0).
\end{equation*}
} % definition

This is the first order Taylor expansion.  If \( n = 1 \), this is \( F(x) \ge F(x_0) + F'(x_0) ( x - x_0) \).

F3

The first order condition says a convex function \underline{always} lies above its first order approximation.  When differentiable, the supporting plane is the tangent plane.
When differentiable, the

\makedefinition{Second order condition}{dfn:convexOptimizationLecture6:40}{
If \( F : \bbR^n \rightarrow \bbR \)
is twice differentiable, then \( F \) is convex iff \( \dom F \) is a convex set and \( \spacegrad^2 F(\Bx) \ge 0 \,\forall \Bx \in \dom F\).
} % definition

The Hessian is always symmetric, but is not necesssarily positive.  Recall that the Hessian is the matrix of the second order partials \( (\spacegrad F)_{ij} = \partial^2 F/(\partial x_i \partial x_j) \).

The scalar case is \( F''(x) \ge 0 \, \forall x \in \dom F \).

An implication is that if \( F \) is convex, then \( F(x) \ge F(x_0) + F'(x_0) (x - x_0) \,\forall x, x_0 \in \dom F\)

Since \( F \) is convex, \( \dom F \) is convex.

Consider any 2 points \( x, y \in \dom F \), and \( \theta \in [0,1] \).  Define

\begin{dmath}\label{eqn:convexOptimizationLecture6:60}
z = (1-\theta) x + \theta y \in \dom F,
\end{dmath}

then since \( \dom F \) is convex

\begin{dmath}\label{eqn:convexOptimizationLecture6:80}
F(z) =
F( (1-\theta) x + \theta y )
\le
(1-\theta) F(x) + \theta F(y )
\end{dmath}

Reordering

\begin{dmath}\label{eqn:convexOptimizationLecture6:220}
\theta F(x) \ge
\theta F(x) + F(z) - F(x),
\end{dmath}

or
\begin{dmath}\label{eqn:convexOptimizationLecture6:100}
F(y) \ge
F(x) + \frac{F(x + \theta(y-x)) - F(x)}{\theta},
\end{dmath}

which is, in the limit,

\begin{dmath}\label{eqn:convexOptimizationLecture6:120}
F(y) \ge
F(x) + F'(x) (y - x)\qedmarker
\end{dmath}

To prove the other direction, showing that

\begin{dmath}\label{eqn:convexOptimizationLecture6:140}
F(x) \ge F(x_0) + F'(x_0) (x - x_0),
\end{dmath}

implies that \( F \) is convex.
Take any \( x, y \in \dom F \) and any \( \theta \in [0,1] \).  Define

\begin{dmath}\label{eqn:convexOptimizationLecture6:160}
z = \theta x + (1 -\theta) y,
\end{dmath}

which is in \( \dom F \) by assumption.  We want to show that

\begin{dmath}\label{eqn:convexOptimizationLecture6:180}
F(z) \le \theta F(x) + (1-\theta) F(y).
\end{dmath}

By assumption

\begin{enumerate}[(i)]
\item \( F(x) \ge F(z) + F'(z) (x - z) \)
\item \( F(y) \ge F(z) + F'(z) (y - z) \)
\end{enumerate}

Compute

\begin{dmath}\label{eqn:convexOptimizationLecture6:200}
\theta F(x) + (1-\theta) F(y)
\ge
\theta \lr{  F(z) + F'(z) (x - z) }
+ (1-\theta) \lr{  F(z) + F'(z) (y - z) }
=
F(z) + F'(z) \lr{ \theta( x - z) + (1-\theta) (y-z) }
=
F(z) + F'(z) \lr{ \theta x  + (1-\theta) y - \theta z - (1 -\theta) z }
=
F(z) + F'(z) \lr{ \theta x  + (1-\theta) y - z}
=
F(z) + F'(z) \lr{ z - z}
= F(z).
\end{dmath}

\paragraph{Proof of the 2nd order case for \( n = 1 \)}

Want to prove that if

\begin{dmath}\label{eqn:convexOptimizationLecture6:240}
F : \bbR \rightarrow \bbR
\end{dmath}

is a convex function, then \( F''(x) \ge 0 \,\forall x \in \dom F \).

By the first order conditions \( \forall x \ne y \in \dom F \)

\begin{dmath}\label{eqn:convexOptimizationLecture6:260}
\begin{aligned}
F(y) &\ge F(x) + F'(x) (y - x)
F(x) &\ge F(y) + F'(y) (x - y)
\end{aligned}
\end{dmath}

Can combine and get

\begin{equation}\label{eqn:convexOptimizationLecture6:280}
F'(x) (y-x) \le F(y) - F(x) \le F'(y)(y-x)
\end{equation}

Subtract the two derivative terms for

\begin{dmath}\label{eqn:convexOptimizationLecture6:340}
\frac{(F'(y) - F'(x))(y - x)}{(y - x)^2} \ge 0,
\end{dmath}

or
\begin{dmath}\label{eqn:convexOptimizationLecture6:300}
\frac{F'(y) - F'(x)}{y - x} \ge 0.
\end{dmath}

In the limit as \( y \rightarrow x \), this is
\boxedEquation{eqn:convexOptimizationLecture6:320}{
F''(x) \ge 0 \,\forall x \in \dom F.
}

Now prove the reverse condition:

If \( F''(x) \ge 0 \,\forall x \in \dom F \subseteq \bbR \), implies that \( F : \bbR \rightarrow \bbR \) is convex.

Note that if \( F''(x) \ge 0 \), then \( F'(x) \) is non-decreasing in \( x \).

i.e. If \( x < y \), where \( x, y \in \dom F\), then

\begin{dmath}\label{eqn:convexOptimizationLecture6:360}
F'(x) \le F'(y).
\end{dmath}

Consider any \( x,y \in \dom F\) such that \( x < y \), where

\begin{dmath}\label{eqn:convexOptimizationLecture6:380}
F(y) - F(x) = \int_x^y F'(t) dt \ge F'(x) \int_x^y 1 dt = F'(x) (y-x).
\end{dmath}

This tells us that

\begin{dmath}\label{eqn:convexOptimizationLecture6:400}
F(y) \ge F(x) + F'(x)(y - x),
\end{dmath}

which is the first order condition.  Similarily consider any \( x,y \in \dom F\) such that \( x < y \), where

\begin{dmath}\label{eqn:convexOptimizationLecture6:420}
F(y) - F(x) = \int_x^y F'(t) dt \le F'(y) \int_x^y 1 dt = F'(y) (y-x).
\end{dmath}

This tells us that

\begin{dmath}\label{eqn:convexOptimizationLecture6:440}
F(x) \ge F(y) + F'(y)(x - y).
\end{dmath}

\paragraph{Vector proof:}

\( F \) is convex iff \( F(\Bx + t \Bv) \) is convex \( \forall \Bx,\Bv \in \bbR^n, t \in \bbR \), keeping \( \Bx + t \Bv \in \dom F\).

Let
\begin{dmath}\label{eqn:convexOptimizationLecture6:460}
h(t ; \Bx, \Bv) = F(\Bx + t \Bv)
\end{dmath}

then \( h(t) \) satisfies scalar first and second order conditions for all \( \Bx, \Bv \).

\begin{dmath}\label{eqn:convexOptimizationLecture6:480}
h(t) = F(\Bx + t \Bv) = F(g(t)),
\end{dmath}

where \( g(t) = \Bx + t \Bv \), where

\begin{dmath}\label{eqn:convexOptimizationLecture6:500}
\begin{aligned}
F &: \bbR^n \rightarrow \bbR \\
g &: \bbR \rightarrow \bbR^n.
\end{aligned}
\end{dmath}

This is expressing \( h(t) \) as a composition of two functions.  By the first order condition for scalar functions we know that

\begin{dmath}\label{eqn:convexOptimizationLecture6:520}
h(t) \ge h(0) + h'(0) t.
\end{dmath}

Note that

\begin{equation}\label{eqn:convexOptimizationLecture6:540}
h(0) = \evalbar{F(\Bx + t \Bv)}{t = 0} = F(\Bx).
\end{equation}

Let's figure out what \( h'(0) \) is.  Recall hat for any \( \tilde{F} : \bbR^n \rightarrow \bbR^m \)

\begin{equation}\label{eqn:convexOptimizationLecture6:560}
D \tilde{F} \in \bbR^{m \times n},
\end{equation}

and
\begin{equation}\label{eqn:convexOptimizationLecture6:580}
{D \tilde{F}(\Bx)}_{ij} = \PD{x_j}{\tilde{F_i}(\Bx)}
\end{equation}

This is one function per row, for \( i \in [1,m], j \in [1,n] \).  This gives

\begin{dmath}\label{eqn:convexOptimizationLecture6:600}
\frac{d}{dt} F(\Bx + \Bv t)
=
\frac{d}{dt} F( g(t) )
=
\frac{d}{dt} h(t)
= D h(t)
= D F(g(t))  \cdot D g(t)
\end{dmath}

The first matrix is in \( \bbR^{1\times n} \) whereas the second is in \( \bbR^{n\times 1} \), since
\( F : \bbR^n \rightarrow \bbR \) and \( g : \bbR \rightarrow \bbR^n \).  This gives

\begin{equation}\label{eqn:convexOptimizationLecture6:620}
\frac{d}{dt} F(\Bx + \Bv t)
= \evalbar{D F(\tilde{\Bx})}{\tilde{\Bx} = g(t)}  \cdot D g(t).
\end{equation}

That first matrix is

\begin{dmath}\label{eqn:convexOptimizationLecture6:640}
\evalbar{D F(\tilde{\Bx})}{\tilde{\Bx} = g(t)}
=
\evalbar{
\lr{\begin{bmatrix}
\PD{\tilde{x}_1}{ F(\tilde{\Bx})} &
\PD{\tilde{x}_2}{ F(\tilde{\Bx})} & \cdots
\PD{\tilde{x}_n}{ F(\tilde{\Bx})}
\end{bmatrix}
}}{ \tilde{\Bx} = g(t) = \Bx + t \Bv }
=
\evalbar{
\lr{ \spacegrad F(\tilde{\Bx}) }^\T
}{
\tilde{\Bx} = g(t)
}
=
\lr{ \spacegrad F(g(t)) }^\T.
\end{dmath}

The second Jacobian is

\begin{dmath}\label{eqn:convexOptimizationLecture6:660}
D g(t)
=
D
\begin{bmatrix}
g_1(t) \\
g_2(t) \\
\vdots \\
g_n(t) \\
\end{bmatrix}
=
D
\begin{bmatrix}
x_1 + t v_1 \\
x_2 + t v_2 \\
\vdots \\
x_n + t v_n \\
\end{bmatrix}
=
\begin{bmatrix}
v_1 \\
v_1 \\
\vdots \\
v_n \\
\end{bmatrix}
=
\Bv.
\end{dmath}

so

\begin{dmath}\label{eqn:convexOptimizationLecture6:680}
h'(t) = D h(t) = \lr{ \spacegrad F(g(t))}^\T \Bv,
\end{dmath}

and
\begin{dmath}\label{eqn:convexOptimizationLecture6:700}
h'(0) = \lr{ \spacegrad F(g(0))}^\T \Bv
=
\lr{ \spacegrad F(\Bx)}^\T \Bv.
\end{dmath}

Finally

\begin{dmath}\label{eqn:convexOptimizationLecture6:720}
F(\Bx + t \Bv)
\ge h(0) + h'(0) t
= F(\Bx) + \lr{ \spacegrad F(\Bx) }^\T (t \Bv)
= F(\Bx) + \innerprod{ \spacegrad F(\Bx) }{ t \Bv}.
\end{dmath}

Which is true for all \( \Bx, \Bx + t \Bv \in \dom F \).  Note that the quantity \( t \Bv \) is a shift.

F4

\paragraph{Epigraph}

Recall that if \( (\Bx, t) \in \epi F \) then \( t \ge F(\Bx) \).

\begin{dmath}\label{eqn:convexOptimizationLecture6:740}
t \ge F(\Bx) \ge F(\Bx_0) + \lr{\spacegrad F(\Bx_0) }^\T (\Bx - \Bx_0),
\end{dmath}

or

\begin{dmath}\label{eqn:convexOptimizationLecture6:760}
0 \ge
-(t - F(\Bx_0)) + \lr{\spacegrad F(\Bx_0) }^\T (\Bx - \Bx_0),
\end{dmath}

In block matrix form

\begin{dmath}\label{eqn:convexOptimizationLecture6:780}
0 \ge
\begin{bmatrix}
\lr{ \spacegrad F(\Bx_0) }^\T & -1
\end{bmatrix}
\begin{bmatrix}
\Bx - \Bx_0 \\
t - F(\Bx_0)
\end{bmatrix}
\end{dmath}

With \( \Bw =
\begin{bmatrix}
\lr{ \spacegrad F(\Bx_0) }^\T & -1
\end{bmatrix} \), the geometry of the epigraph relation to the half plane is sketched in

F5

\EndArticle
%\EndNoBibArticle

      %\section{Problems}
   \mychapter{Examples of convex and concave functions, local and global minimums.}
      %
% Copyright � 2017 Peeter Joot.  All Rights Reserved.
% Licenced as described in the file LICENSE under the root directory of this GIT repository.
%
%\chapter{Examples of convex and concave functions, local and global minimums}
%\paragraph{Today}
%
%\begin{itemize}
%\item Local and global optimality
%\item Compositions of functions
%\item Examples
%\end{itemize}

\paragraph{Example:}
%
\begin{equation}\label{eqn:convexOptimizationLecture7:20}
\begin{aligned}
F(x) &= x^2  \\
F''(x) &= 2 > 0
\end{aligned}
\end{equation}
%
strictly convex.

\paragraph{Example:}
%
\begin{equation}\label{eqn:convexOptimizationLecture7:40}
\begin{aligned}
F(x) &= x^3  \\
F''(x) &= 6 x.
\end{aligned}
\end{equation}
%
Not always non-negative, so not convex.  However \( x^3 \) is convex on \( \dom F = \bbR_{+} \).

\paragraph{Example:}
%
\begin{equation}\label{eqn:convexOptimizationLecture7:60}
\begin{aligned}
F(x) &= x^\alpha \\
F'(x) &= \alpha x^{\alpha-1} \\
F''(x) &= \alpha(\alpha-1) x^{\alpha-2}.
\end{aligned}
\end{equation}
%
%\cref{fig:l7xToTheN:l7xToTheNFig1}.
\imageFigure{../figures/ece1505-convex-optimization/l7xToTheNFig1}{Powers of x.}{fig:l7xToTheN:l7xToTheNFig1}{0.3}

This is convex on \( \bbR_{+} \), if \( \alpha \ge 1 \), or \( \alpha \le 0 \).

\paragraph{Example:}
%
\begin{equation}\label{eqn:convexOptimizationLecture7:80}
\begin{aligned}
F(x) &= \log x \\
F'(x) &= \inv{x} \\
F''(x) &= -\inv{x^2} \le 0
\end{aligned}
\end{equation}
%
This is concave.

\paragraph{Example:}
%
\begin{equation}\label{eqn:convexOptimizationLecture7:100}
\begin{aligned}
F(x) &= x\log x \\
F'(x) &= \log x + x \inv{x} = 1 + \log x \\
F''(x) &= \inv{x}
\end{aligned}
\end{equation}
%
This is strictly convex on
\( \bbR_{++} \), where
\( F''(x) \ge 0 \).

\paragraph{Example:}
%
\begin{equation}\label{eqn:convexOptimizationLecture7:120}
\begin{aligned}
F(x) &= e^{\alpha x} \\
F'(x) &= \alpha e^{\alpha x} \\
F''(x) &= \alpha^2 e^{\alpha x} \ge 0
\end{aligned}
\end{equation}
%
%F2
\imageFigure{../figures/ece1505-convex-optimization/l7eToAlphaXFig2}{Exponential.}{fig:l7eToAlphaX:l7eToAlphaXFig2}{0.3}

Such functions are plotted in
\cref{fig:l7eToAlphaX:l7eToAlphaXFig2}, and are
convex function for all \( \alpha \).

\paragraph{Example:}

For symmetric \( P \in S^n \)
%
\begin{equation}\label{eqn:convexOptimizationLecture7:140}
\begin{aligned}
F(\Bx) &= \Bx^\T P \Bx + 2 \Bq^\T \Bx + r \\
\spacegrad F &= (P + P^\T) \Bx + 2 \Bq = 2 P \Bx + 2 \Bq \\
\spacegrad^2 F &= 2 P.
\end{aligned}
\end{equation}
%
This is convex(concave) if \( P \ge 0 \) (\( P \le 0\)).

\paragraph{Example:}

A quadratic function
%
\begin{equation}\label{eqn:convexOptimizationLecture7:780}
F(x, y) = x^2 + y^2 + 3 x y,
\end{equation}
%
that is neither convex nor concave is plotted in \cref{fig:l7quadratic3d:l7quadratic3dFig7a}

\imageTwoFigures
{../figures/ece1505-convex-optimization/l7quadratic3dFig7a}
{../figures/ece1505-convex-optimization/l7quadraticContourFig7b}
{Function with saddle point (3d and contours).}{fig:l7quadratic3d:l7quadratic3dFig7a}{scale=0.3}

This function can be put in matrix form
%
\begin{equation}\label{eqn:convexOptimizationLecture7:160}
F(x, y) = x^2 + y^2 + 3 x y
=
\begin{bmatrix}
x & y
\end{bmatrix}
\begin{bmatrix}
1 & 1.5 \\
1.5 & 1
\end{bmatrix}
\begin{bmatrix}
x \\
 y
\end{bmatrix},
\end{equation}
%
and has the Hessian
%
\begin{equation}\label{eqn:convexOptimizationLecture7:180}
\begin{aligned}
\spacegrad^2 F
&=
\begin{bmatrix}
\partial_{xx} F & \partial_{xy} F \\
\partial_{yx} F & \partial_{yy} F \\
\end{bmatrix} \\
&=
\begin{bmatrix}
2 & 3 \\
3 & 2
\end{bmatrix} \\
&= 2 P.
\end{aligned}
\end{equation}
%
From the plot we know that this is not PSD, but this can be confirmed by checking the eigenvalues
%
\begin{equation}\label{eqn:convexOptimizationLecture7:200}
\begin{aligned}
0 
&= \det ( P - \lambda I ) \\
&= (1 - \lambda)^2 - 1.5^2,
\end{aligned}
\end{equation}
%
which has solutions
%
\begin{equation}\label{eqn:convexOptimizationLecture7:220}
\lambda = 1 \pm \frac{3}{2} = \frac{3}{2}, -\frac{1}{2}.
\end{equation}
%
This is not PSD nor negative semi-definite, because it has one positive and one negative eigenvalues.  This is neither convex nor concave.

Along \( y = -x \),
%
\begin{equation}\label{eqn:convexOptimizationLecture7:240}
\begin{aligned}
F(x,y)
&= F(x,-x) \\
&= 2 x^2 - 3 x^2 \\
&= - x^2,
\end{aligned}
\end{equation}
%
so it is concave along this line.  Along \( y = x \)
%
\begin{equation}\label{eqn:convexOptimizationLecture7:260}
\begin{aligned}
F(x,y)
&= F(x,x) \\
&= 2 x^2 + 3 x^2 \\
&= 5 x^2,
\end{aligned}
\end{equation}
%
so it is convex along this line.

\paragraph{Example:}
%
\begin{equation}\label{eqn:convexOptimizationLecture7:280}
F(\Bx) = \sqrt{ x_1 x_2 },
\end{equation}
%
on \( \dom F = \setlr{ x_1 \ge 0, x_2 \ge 0 } \)

For the Hessian
\begin{equation}\label{eqn:convexOptimizationLecture7:300}
\begin{aligned}
\PD{x_1}{F} &= \frac{1}{2} x_1^{-1/2} x_2^{1/2} \\
\PD{x_2}{F} &= \frac{1}{2} x_2^{-1/2} x_1^{1/2}
\end{aligned}
\end{equation}
%
The Hessian components are
%
\begin{equation}\label{eqn:convexOptimizationLecture7:320}
\begin{aligned}
\PD{x_1}{} \PD{x_1}{F} &= -\frac{1}{4} x_1^{-3/2} x_2^{1/2} \\
\PD{x_1}{} \PD{x_2}{F} &= \frac{1}{4} x_2^{-1/2} x_1^{-1/2} \\
\PD{x_2}{} \PD{x_1}{F} &= \frac{1}{4} x_1^{-1/2} x_2^{-1/2} \\
\PD{x_2}{} \PD{x_2}{F} &= -\frac{1}{4} x_2^{-3/2} x_1^{1/2}
\end{aligned}
\end{equation}
%
or
\begin{equation}\label{eqn:convexOptimizationLecture7:340}
\spacegrad^2 F
=
-\frac{\sqrt{x_1 x_2}}{4}
\begin{bmatrix}
\inv{x_1^2} & -\inv{x_1 x_2} \\
-\inv{x_1 x_2} & \inv{x_2^2}
\end{bmatrix}.
\end{equation}
%
Checking this for PSD against \( \Bv = (v_1, v_2) \), we have
\begin{equation}\label{eqn:convexOptimizationLecture7:360}
\begin{aligned}
\begin{bmatrix}
v_1 & v_2
\end{bmatrix}
\begin{bmatrix}
\inv{x_1^2} & -\inv{x_1 x_2} \\
-\inv{x_1 x_2} & \inv{x_2^2}
\end{bmatrix}
\begin{bmatrix}
v_1 \\ v_2
\end{bmatrix}
&=
\begin{bmatrix}
v_1 & v_2
\end{bmatrix}
\begin{bmatrix}
\inv{x_1^2} v_1 -\inv{x_1 x_2} v_2  \\
-\inv{x_1 x_2} v_1 + \inv{x_2^2} v_2
\end{bmatrix} \\
&=
\lr{ \inv{x_1^2} v_1 -\inv{x_1 x_2} v_2 } v_1 +
\lr{ -\inv{x_1 x_2} v_1 + \inv{x_2^2} v_2 } v_2 \\
&=
\inv{x_1^2} v_1^2
+ \inv{x_2^2} v_2^2
-2 \inv{x_1 x_2} v_1 v_2 \\
&=
\lr{
\frac{v_1}{x_1}
-\frac{v_2}{x_2}
}^2 \\
&\ge 0,
\end{aligned}
\end{equation}
%
so \( \spacegrad^2 F \le 0 \).  This is a negative semi-definite function (concave).  Observe that this check required checking PSD for all values of \( \Bx \).

This is an example of a more general result
%
\begin{equation}\label{eqn:convexOptimizationLecture7:380}
F(x) = \lr{ \prod_{i = 1}^n x_i }^{1/n},
\end{equation}
%
which is concave (prove on homework).

\paragraph{Summary.}

If \( F \) is differentiable in \R{n}, then check the curvature of the function along all lines.  i.e.  At all locations and in all directions.

If the Hessian is PSD at all \( \Bx \in \dom F \), that is
%
\begin{equation}\label{eqn:convexOptimizationLecture7:400}
\spacegrad^2 F \ge 0 \, \forall \Bx \in \dom F,
\end{equation}
%
then the function is convex.

\paragraph{more examples of convex, but not necessarily differentiable functions}

\paragraph{Example:}
Over \( \dom F = \bbR^n \)
%
\begin{equation}\label{eqn:convexOptimizationLecture7:420}
F(\Bx) = \max_{i = 1}^n x_i
\end{equation}
%
i.e.
\begin{equation}\label{eqn:convexOptimizationLecture7:440}
\begin{aligned}
F((1,2) &= 2 \\
F((3,-1) &= 3
\end{aligned}
\end{equation}
%
\paragraph{Example:}
%
\begin{equation}\label{eqn:convexOptimizationLecture7:460}
F(\Bx) = \max_{i = 1}^n F_i(\Bx),
\end{equation}
%
where
%
\begin{equation}\label{eqn:convexOptimizationLecture7:480}
F_i(\Bx)
=
... ?
\end{equation}
%
max of a set of convex functions is a convex function.

\paragraph{Example:}
%
\begin{equation}\label{eqn:convexOptimizationLecture7:500}
F(x) =
x_{[1]} +
x_{[2]} +
x_{[3]}
\end{equation}
%
where

\( x_{[k]} \) is the k-th largest number in the list

Write
%
\begin{equation}\label{eqn:convexOptimizationLecture7:520}
F(x) = \max x_i + x_j + x_k
\end{equation}
%
\begin{equation}\label{eqn:convexOptimizationLecture7:540}
(i,j,k) \in \binom{n}{3}
\end{equation}
%
\paragraph{Example:}

For \( \Ba \in \bbR^n \) and \( b_i \in \bbR \)
%
\begin{equation}\label{eqn:convexOptimizationLecture7:560}
\begin{aligned}
F(\Bx)
&= \sum_{i = 1}^n \log( b_i - \Ba^\T \Bx )^{-1} \\
&= -\sum_{i = 1}^n \log( b_i - \Ba^\T \Bx ).
\end{aligned}
\end{equation}
%
This \( b_i - \Ba^\T \Bx \) is an affine function of \( \Bx \) so it doesn't affect convexity.

Since \( \log \) is concave, \( -\log \) is convex.  Convex functions of affine function of \( \Bx \) is convex function of \( \Bx \).

\paragraph{Example:}
\begin{equation}\label{eqn:convexOptimizationLecture7:580}
F(\Bx) = \sup_{\By \in C} \Norm{ \Bx -  \By }
\end{equation}
%
%F3
\imageFigure{../figures/ece1505-convex-optimization/l7MaxFuncFig3}{Max length function}{fig:l7MaxFunc:l7MaxFuncFig3}{0.3}

Here \( C \subseteq \bbR^n \) is not necessarily convex.  We are using \( \sup \) here because the set \( C \) may be open.  This function is the length of the line from \( \Bx \) to the point in \( C \) that is furthest from \( \Bx \).

\begin{itemize}
\item \( \Bx - \By \) is linear in \( \Bx \)
\item \( g_\By(\Bx) = \Norm{\Bx - \By} \) is convex in \( \Bx \) since norms are convex functions.
\item \( F(\Bx) = \sup_{\By \in C} \Norm{ \Bx -  \By } \).  Each \( \By \) index is a convex function.  Taking max of those.
\end{itemize}

\paragraph{Example:}
%
\begin{equation}\label{eqn:convexOptimizationLecture7:600}
F(\Bx) = \inf_{\By \in C} \Norm{ \Bx -  \By }.
\end{equation}
%
Min and max of two convex functions are plotted in \cref{fig:l7minAndMax:l7minAndMaxFig4}.
\imageFigure{../figures/ece1505-convex-optimization/l7minAndMaxFig4}{Min and max}{fig:l7minAndMax:l7minAndMaxFig4}{0.3}

The max is observed to be convex, whereas the min is not necessarily so.
%Consider the 2d example of a min length function sketched in \cref{fig:l7MaxFunc:l7MaxFuncFig3}, where
%
\begin{equation}\label{eqn:convexOptimizationLecture7:800}
F(\Bz) = F(\theta \Bx + (1-\theta) \By) \ge \theta F(\Bx) + (1-\theta)F(\By).
\end{equation}
%
This is not necessarily convex for all sets \( C \subseteq \bbR^n \), because the \( \inf \) of a bunch of convex function is not necessarily convex.  However, if \( C \) is convex, then \( F(\Bx) \) is convex.

\paragraph{Consequences of convexity for differentiable functions}

\begin{itemize}
\item Think about unconstrained functions \( \dom F = \bbR^n \).
\item By first order condition \( F \) is convex iff the domain is convex and
\begin{equation}\label{eqn:convexOptimizationLecture7:620}
F(\Bx) \ge \lr{ \spacegrad F(\Bx)}^\T (\By - \Bx) \, \forall \Bx, \By \in \dom F.
\end{equation}
\end{itemize}

If \( F \) is convex and one can find an \( \Bx^\conj \in \dom F \) such that
%
\begin{equation}\label{eqn:convexOptimizationLecture7:640}
\spacegrad F(\Bx^\conj) = 0,
\end{equation}
%
then
%
\begin{equation}\label{eqn:convexOptimizationLecture7:660}
F(\By) \ge F(\Bx^\conj) \, \forall \By \in \dom F.
\end{equation}
%
If you can find the point where the gradient is zero (which can't always be found), then \( \Bx^\conj\) is a global minimum of \( F \).

Conversely, if \( \Bx^\conj \) is a global minimizer of \( F \), then \( \spacegrad F(\Bx^\conj) = 0 \) must hold.  If that were not the case, then you would be able to find a direction to move downhill, contracting the optimality of \( \Bx^\conj\).

\paragraph{Local vs Global optimum}

%\cref{fig:l7GlobalAndLocalMin:l7GlobalAndLocalMinFig6}.
\imageFigure{../figures/ece1505-convex-optimization/l7GlobalAndLocalMinFig6}{Global and local minimums}{fig:l7GlobalAndLocalMin:l7GlobalAndLocalMinFig6}{0.3}

\makedefinition{Local optimum.}{dfn:convexOptimizationLecture7:680}{
\( \Bx^\conj \) is a local optimum of \( F \) if \( \exists \epsilon > 0 \) such that \( \forall \Bx \), \( \Norm{\Bx - \Bx^\conj} < \epsilon \), we have
%
\begin{equation*}
F(\Bx^\conj) \le F(\Bx)
\end{equation*}
} % definition

%F6
%\cref{fig:l7MinFunc:l7MinFuncFig5}.
\imageFigure{../figures/ece1505-convex-optimization/l7MinFuncFig5}{min length function.}{fig:l7MinFunc:l7MinFuncFig5}{0.3}

\maketheorem{}{thm:convexOptimizationLecture7:700}{
Suppose \( F \) is twice continuously differentiable (not necessarily convex)

\begin{itemize}
\item
If \( \Bx^\conj\) is a local optimum then
%
\begin{equation*}
\begin{aligned}
\spacegrad F(\Bx^\conj) &= 0 \\
\spacegrad^2 F(\Bx^\conj) \ge 0
\end{aligned}
\end{equation*}
%
\item
If
\begin{equation*}
\begin{aligned}
\spacegrad F(\Bx^\conj) &= 0 \\
\spacegrad^2 F(\Bx^\conj) \ge 0
\end{aligned},
\end{equation*}
%
then \( \Bx^\conj\) is a local optimum.
\end{itemize}
} % theorem

Proof:

\begin{itemize}
\item Let \( \Bx^\conj \) be a local optimum.  Pick any \( \Bv \in \bbR^n \).
%
\begin{equation}\label{eqn:convexOptimizationLecture7:720}
\lim_{t \rightarrow 0} \frac{ F(\Bx^\conj + t \Bv) - F(\Bx^\conj)}{t}
= \lr{ \spacegrad F(\Bx^\conj) }^\T \Bv
\ge 0.
\end{equation}
\end{itemize}

Here the fraction is \( \ge 0 \) since \( \Bx^\conj \) is a local optimum.

Since the choice of \( \Bv \) is arbitrary, the only case that you can ensure that \( \ge 0, \forall \Bv \) is
%
\begin{equation}\label{eqn:convexOptimizationLecture7:740}
\spacegrad F = 0,
\end{equation}
%
( or else could pick \( \Bv = -\spacegrad F(\Bx^\conj) \).

This means that \( \spacegrad F(\Bx^\conj) = 0 \) if \( \Bx^\conj \) is a local optimum.

Consider the 2nd order derivative
%
\begin{equation}\label{eqn:convexOptimizationLecture7:760}
\begin{aligned}
\lim_{t \rightarrow 0} &\frac{ F(\Bx^\conj + t \Bv) - F(\Bx^\conj)}{t^2} \\
&=
\lim_{t \rightarrow 0} \inv{t^2}
\lr{
F(\Bx^\conj) + t \lr{ \spacegrad F(\Bx^\conj) }^\T \Bv + \inv{2} t^2 \Bv^\T \spacegrad^2 F(\Bx^\conj) \Bv + O(t^3)
- F(\Bx^\conj)
} \\
&=
\inv{2} \Bv^\T \spacegrad^2 F(\Bx^\conj) \Bv \\
&\ge 0.
\end{aligned}
\end{equation}
%
Here the \( \ge \) condition also comes from the fraction, based on the optimiality of \( \Bx^\conj \).  This is true for all choice of \( \Bv \), thus \( \spacegrad^2 F(\Bx^\conj) \).

% handout:
%Ps1: 1.6(b) plot solution is correct, but rest is for [3 0, 0, 3] not the problem specified.

%\EndArticle

      %
% Copyright � 2017 Peeter Joot.  All Rights Reserved.
% Licenced as described in the file LICENSE under the root directory of this GIT repository.
%
%\chapter{Local vs. Global, and composition of functions}
%%%\label{chap:convexOptimization8}
%%%
%%%\section{Disclaimer}
%%%
%%%Peeter's lecture notes from class.  These may be incoherent and rough.
%%%
%%%These are notes for the UofT course ECE1505H, Convex Optimization, taught by Prof. Stark Draper, from \citep{boyd2004convex}.
%%%
%%%\section{Today}
%%%
%%%\begin{itemize}
%%%\item Finish local vs global.
%%%\item Compositions of functions.
%%%\item Introduction to convex optimization problems.
%%%\end{itemize}
%%%
%\section{Continuing proof:}

Now we want to prove that if
%
\begin{equation*}
\begin{aligned}
\spacegrad F(\Bx^\conj) &= 0 \\
\spacegrad^2 F(\Bx^\conj) \ge 0
\end{aligned},
\end{equation*}
%
then \( \Bx^\conj\) is a local optimum.

Proof:

Again, using Taylor approximation
%
\begin{equation}\label{eqn:convexOptimizationLecture8:20}
F(\Bx^\conj + \Bv) = F(\Bx^\conj) + \lr{ \spacegrad F(\Bx^\conj)}^\T \Bv + \inv{2} \Bv^\T \spacegrad^2 F(\Bx^\conj) \Bv + o(\Norm{\Bv}^2)
\end{equation}
%
The linear term is zero by assumption, whereas the Hessian term is given as \( > 0 \).  Any direction that you move in, if your move is small enough, this is going uphill at a local optimum.

\section{Summarize:}

For twice continuously differentiable functions, at a local optimum \( \Bx^\conj \), then
%
\begin{equation}\label{eqn:convexOptimizationLecture8:40}
\begin{aligned}
\spacegrad F(\Bx^\conj) &= 0 \\
\spacegrad^2 F(\Bx^\conj) \ge 0
\end{aligned}
\end{equation}
%
If, in addition, \( F \) is convex, then \( \spacegrad F(\Bx^\conj) = 0 \) implies that \( \Bx^\conj \) is a global optimum.  i.e. for (unconstrained) convex functions, local and global optimums are equivalent.

\begin{itemize}
\item It is possible that a convex function does not have a global optimum.  Examples are \( F(x) = e^x \)
(\cref{fig:l8exponential:l8exponentialFig1})
, which has an \( \inf \), but no lowest point.

\imageFigure{../figures/ece1505-convex-optimization/l8exponentialFig1}{Exponential has no global optimum.}{fig:l8exponential:l8exponentialFig1}{0.2}

\item Our discussion has been for unconstrained functions.  For constrained problems (next topic) is not necessarily true that \( \spacegrad F(\Bx) = 0 \) implies that \( \Bx \) is a global optimum, even for \( F \) convex.

As an example of a constrained problem consider
\begin{equation}\label{eqn:convexOptimizationLecture8:n}
\begin{aligned}
\min &2 x^2 + y^2 \\
x &\ge 3 \\
y &\ge 5.
\end{aligned}
\end{equation}
%
The level sets of this objective function are plotted in \cref{fig:l8constrainedMin:l8constrainedMinFig2}.  The optimal point is at \( \Bx^\conj = (3,5) \), where \( \spacegrad F \ne 0 \).
\imageFigure{../figures/ece1505-convex-optimization/l8constrainedMinFig2}{Constrained problem with optimum not at the zero gradient point.}{fig:l8constrainedMin:l8constrainedMinFig2}{0.3}
\end{itemize}
%\EndArticle

      %\section{Problems}
   \mychapter{Compositions of functions.}
      %
% Copyright © 2017 Peeter Joot.  All Rights Reserved.
% Licenced as described in the file LICENSE under the root directory of this GIT repository.
%

%\chapter{Compositions of functions}
\section{Projection}

Given \( \Bx \in \bbR^n, \By \in \bbR^p \), if \( h(\Bx,\By) \) is convex in \( \Bx, \By \), then

\begin{equation}\label{eqn:convexOptimizationLecture8:60}
F(\Bx_0) = \inf_\By h(\Bx_0,\By)
\end{equation}

is convex in \( \Bx\), as sketched in \cref{fig:l8convexProjection:l8convexProjectionFig3}.

%F3 : epigraph of \( h \) is a filled in bowl.
\imageFigure{../figures/ece1505-convex-optimization/l8convexProjectionFig3}{Epigraph of \( h \) is a filled bowl.}{fig:l8convexProjection:l8convexProjectionFig3}{0.4}

The intuition here is that shining light on the (filled) ``bowl''.  That is, the image of \( \epi h \) on the \( \By = 0 \) screen which we will show is a convex set.

Proof:

Since \( h \) is convex in \( \begin{bmatrix} \Bx \\ \By \end{bmatrix} \in \dom h \), then

\begin{equation}\label{eqn:convexOptimizationLecture8:80}
\epi h = \setlr{ (\Bx,\By,t) | t \ge h(\Bx,\By), \begin{bmatrix} \Bx \\ \By \end{bmatrix} \in \dom h },
\end{equation}

is a convex set.

We also have to show that the domain of \( F \) is a convex set.  To show this note that

\begin{equation}\label{eqn:convexOptimizationLecture8:100}
\begin{aligned}
\dom F
&= \setlr{ \Bx | \exists \By s.t. \begin{bmatrix} \Bx \\ \By \end{bmatrix} \in \dom h } \\
&= \setlr{
\begin{bmatrix}
I_{n\times n} & 0_{n \times p}
\end{bmatrix}
\begin{bmatrix}
\Bx \\
\By
\end{bmatrix}
| \begin{bmatrix} \Bx \\ \By \end{bmatrix} \in \dom h
}.
\end{aligned}
\end{equation}

This is an affine map of a convex set.  Therefore \( \dom F \) is a convex set.

\begin{equation}\label{eqn:convexOptimizationLecture8:120}
\begin{aligned}
\epi F
&=
\setlr{ \begin{bmatrix} \Bx \\ \By \end{bmatrix} | t \ge \inf h(\Bx,\By), \Bx \in \dom F, \By: \begin{bmatrix} \Bx \\ \By \end{bmatrix} \in \dom h } \\
&=
\setlr{
\begin{bmatrix}
I & 0 & 0 \\
0 & 0 & 1
\end{bmatrix}
\begin{bmatrix}
\Bx \\
\By \\
t
\end{bmatrix}
|
t \ge h(\Bx,\By), \begin{bmatrix} \Bx \\ \By \end{bmatrix} \in \dom h
}.
\end{aligned}
\end{equation}

\paragraph{Example:}

The function

\begin{equation}\label{eqn:convexOptimizationLecture8:140}
F(\Bx) = \inf_{\By \in C} \Norm{ \Bx - \By },
\end{equation}

over \( \Bx \in \bbR^n, \By \in C \), ,is convex if \( C \) is a convex set.  Reason:

\begin{itemize}
\item \( \Bx - \By \) is linear in \((\Bx, \By)\).
\item \( \Norm{ \Bx - \By } \) is a convex function if the domain is a convex set
\item The domain is \( \bbR^n \times C \).  This will be a convex set if \( C \) is.
\item \( h(\Bx, \By) = \Norm{\Bx -\By} \) is a convex function if \( \dom h \) is a convex set.  By setting \( \dom h = \bbR^n \times C \), if \( C \) is convex, \( \dom h \) is a convex set.
\item \( F() \)
\end{itemize}

\section{Composition of functions}

Consider

\begin{equation}\label{eqn:convexOptimizationLecture8:160}
\begin{aligned}
F(\Bx) &= h(g(\Bx)) \\
\dom F &= \setlr{ \Bx \in \dom g | g(\Bx) \in \dom h } \\
F &: \bbR^n \rightarrow \bbR \\
g &: \bbR^n \rightarrow \bbR \\
h &: \bbR \rightarrow \bbR.
\end{aligned}
\end{equation}

Cases:
\begin{enumerate}[(a)]
\item \( g \) is convex, \( h \) is convex and non-decreasing.
\item \( g \) is convex, \( h \) is convex and non-increasing.
\end{enumerate}

Show for 1D case ( \( n = 1 \)).  Get to \( n > 1 \) by applying to all lines.

\begin{enumerate}[(a)]
\item
\begin{equation}\label{eqn:convexOptimizationLecture8:180}
\begin{aligned}
F'(x) &= h'(g(x)) g'(x) \\
F''(x) &=
h''(g(x)) g'(x) g'(x)
+
h'(g(x)) g''(x) \\
&=
h''(g(x)) (g'(x))^2
+
h'(g(x)) g''(x) \\
&= 
\lr{ \ge 0 } \cdot \lr{ \ge 0 }^2 + \lr{ \ge 0 } \cdot \lr{ \ge 0 },
\end{aligned}
\end{equation}

since \( h \) is respectively convex, and non-decreasing.
\item

\begin{equation}\label{eqn:convexOptimizationLecture8:180b}
F'(x) = 
\lr{ \ge 0 } \cdot \lr{ \ge 0 }^2 + \lr{ \le 0 } \cdot \lr{ \le 0 },
\end{equation}

since \( h \) is respectively convex, and non-increasing, and g is concave.
\end{enumerate}

\section{Extending to multiple dimensions}

\begin{equation}\label{eqn:convexOptimizationLecture8:200}
\begin{aligned}
F(\Bx)
&= h(g(\Bx))
= h( g_1(\Bx), g_2(\Bx), \cdots g_k(\Bx) ) \\
g &: \bbR^n \rightarrow \bbR \\
h &: \bbR^k \rightarrow \bbR.
\end{aligned}
\end{equation}

is convex if \( g_i \) is convex for each \( i \in [1,k] \) and \( h \) is convex and non-decreasing in each argument.

Proof:

again assume \( n = 1 \), without loss of generality,

\begin{equation}\label{eqn:convexOptimizationLecture8:220}
\begin{aligned}
g &: \bbR \rightarrow \bbR^k \\
h &: \bbR^k \rightarrow \bbR \\
\end{aligned}
\end{equation}

\begin{equation}\label{eqn:convexOptimizationLecture8:240}
F''(\Bx)
=
\begin{bmatrix}
g_1(\Bx) & g_2(\Bx) & \cdots & g_k(\Bx)
\end{bmatrix}
\spacegrad^2 h(g(\Bx))
\begin{bmatrix}
g_1'(\Bx) \\ g_2'(\Bx) \\ \vdots \\ g_k'(\Bx)
\end{bmatrix}
+
\lr{ \spacegrad h(g(x)) }^\T
\begin{bmatrix}
g_1''(\Bx) \\ g_2''(\Bx) \\ \vdots \\ g_k''(\Bx)
\end{bmatrix}
\end{equation}

The Hessian is PSD.

\paragraph{Example:}

\begin{equation}\label{eqn:convexOptimizationLecture8:260}
F(x) = \exp( g(x) ) = h( g(x) ),
\end{equation}

where \( g \) is convex is convex, and \( h(y) = e^y \).  This implies that \( F \) is a convex function.

\paragraph{Example:}

\begin{equation}\label{eqn:convexOptimizationLecture8:280}
F(x) = \inv{g(x)},
\end{equation}

is convex if \( g(x) \) is concave and positive.  The most simple such example of such a function is \( h(x) = 1/x, \dom h = \bbR_{++} \), which is plotted in \cref{fig:l8oneOverX:l8oneOverXFig6}.

\imageFigure{../figures/ece1505-convex-optimization/l8oneOverXFig6}{Inverse function is convex over positive domain.}{fig:l8oneOverX:l8oneOverXFig6}{0.2}

\paragraph{Example:}

\begin{equation}\label{eqn:convexOptimizationLecture8:300}
F(x) = - \sum_{i = 1}^n \log( -F_i(x) )
\end{equation}

is convex on \( \setlr{ x | F_i(x) < 0 \forall i } \) if all \( F_i \) are convex.

\begin{itemize}
\item Due to \( \dom F \), \( -F_i(x) > 0 \,\forall x \in \dom F \)
\item \( \log(x) \) concave on \( \bbR_{++} \) so \( -\log \) convex also non-increasing (\cref{fig:l8MinusLogX:l8MinusLogXFig7}).

\imageFigure{../figures/ece1505-convex-optimization/l8MinusLogXFig7}{Negative logarithm convex over positive domain.}{fig:l8MinusLogX:l8MinusLogXFig7}{0.2}

\end{itemize}

\begin{equation}\label{eqn:convexOptimizationLecture8:320}
F(x) = \sum h_i(x)
\end{equation}

but
\begin{equation}\label{eqn:convexOptimizationLecture8:340}
h_i(x) = -\log(-F_i(x)),
\end{equation}

which is a convex and non-increasing function (\(-\log\)), of a convex function \( -F_i(x) \).  Each 
\( h_i \) is convex, so this is a sum of convex functions, and is therefore convex.

\paragraph{Example:}

Over \( \dom F = S^n_{++} \)

\begin{equation}\label{eqn:convexOptimizationLecture8:360}
F(X) = \log \det X^{-1}
\end{equation}

To show that this is convex, check all lines in domain.  A line in \( S^n_{++} \) is a 1D family of matrices

\begin{equation}\label{eqn:convexOptimizationLecture8:380}
\tilde{F}(t) = \log \det( \lr{X_0 + t H}^{-1} ),
\end{equation}

where \( X_0 \in S^n_{++}, t \in \bbR, H \in S^n \).

F9

For \( t \) small enough,

\begin{equation}\label{eqn:convexOptimizationLecture8:400}
X_0 + t H \in S^n_{++}
\end{equation}

\begin{equation}\label{eqn:convexOptimizationLecture8:420}
\begin{aligned}
\tilde{F}(t)
&= \log \det( \lr{X_0 + t H}^{-1} ) \\
&= \log \det\lr{ X_0^{-1/2} \lr{I + t X_0^{-1/2} H X_0^{-1/2} }^{-1} X_0^{-1/2} } \\
&= \log \det\lr{ X_0^{-1} \lr{I + t X_0^{-1/2} H X_0^{-1/2} }^{-1} } \\
&= \log \det X_0^{-1} + \log\det \lr{I + t X_0^{-1/2} H X_0^{-1/2} }^{-1} \\
&= \log \det X_0^{-1} - \log\det \lr{I + t X_0^{-1/2} H X_0^{-1/2} } \\
&= \log \det X_0^{-1} - \log\det \lr{I + t M }.
\end{aligned}
\end{equation}

If \( \lambda_i \) are eigenvalues of \( M \), then \( 1 + t \lambda_i \) are eigenvalues of \( I + t M \).  i.e.:

\begin{equation}\label{eqn:convexOptimizationLecture8:440}
\begin{aligned}
(I + t M) \Bv
&= I \Bv + t \lambda_i \Bv \\
&= (1 + t \lambda_i) \Bv.
\end{aligned}
\end{equation}

This gives

\begin{equation}\label{eqn:convexOptimizationLecture8:460}
\tilde{F}(t)
= \log \det X_0^{-1} - \log \prod_{i = 1}^n (1 + t \lambda_i)
= \log \det X_0^{-1} - \sum_{i = 1}^n \log (1 + t \lambda_i)
\end{equation}

\begin{itemize}
\item \( 1 + t \lambda_i \) is linear in \( t \).
\item \( -\log \) is convex in its argument.
\item sum of convex function is convex.
\end{itemize}

\paragraph{Example:}

\begin{equation}\label{eqn:convexOptimizationLecture8:480}
F(X) = \lambda_\max(X),
\end{equation}

is convex on \( \dom F \in S^n \)

(a)
\begin{equation}\label{eqn:convexOptimizationLecture8:500}
\lambda_{\max} (X) = \sup_{\Norm{\Bv}_2 \le 1} \Bv^\T X \Bv,
\end{equation}

\begin{equation}\label{eqn:convexOptimizationLecture8:520}
\begin{bmatrix}
\lambda_1 &           &        & \\
          & \lambda_2 &        & \\
          &           & \ddots & \\
          &           &        & \lambda_n
\end{bmatrix}
\end{equation}


Recall that a decomposition

\begin{equation}\label{eqn:convexOptimizationLecture8:540}
\begin{aligned}
X &= Q \Lambda Q^\T \\
Q^\T Q = Q Q^\T = I
\end{aligned}
\end{equation}

can be used for any \( X \in S^n \).

(b)

Note that \( \Bv^\T X \Bv \) is linear in \( X \).  This is a max of a number of linear (and convex) functions, so it is convex.

Last example:

(non-symmetric matrices)

\begin{equation}\label{eqn:convexOptimizationLecture8:560}
F(X) = \sigma_\max(X),
\end{equation}

is convex on \( \dom F = \bbR^{m \times n} \).  Here

\begin{equation}\label{eqn:convexOptimizationLecture8:580}
\sigma_\max(X) = \sup_{\Norm{\Bv}_2 = 1} \Norm{X \Bv}_2
\end{equation}

This is called an operator norm of \( X \).  Using the SVD

\begin{equation}\label{eqn:convexOptimizationLecture8:600}
\begin{aligned}
X &= U \Sigma V^\T \\
U &= \bbR^{m \times r} \\
\Sigma &\in \diag \in \bbR{ r \times r } \\
V^T &\in \bbR^{r \times n}.
\end{aligned}
\end{equation}

Have

\begin{equation}\label{eqn:convexOptimizationLecture8:620}
\begin{aligned}
\Norm{X \Bv}_2^2
&= \Norm{ U \Sigma V^\T \Bv }_2^2 \\
&= \Bv^\T V \Sigma U^\T U \Sigma V^\T \Bv \\
&= \Bv^\T V \Sigma \Sigma V^\T \Bv \\
&= \Bv^\T V \Sigma^2 V^\T \Bv \\
&= \tilde{\Bv}^\T \Sigma^2 \tilde{\Bv},
\end{aligned}
\end{equation}

where \( \tilde{\Bv} = \Bv^\T V \), so

\begin{equation}\label{eqn:convexOptimizationLecture8:640}
\Norm{X \Bv}_2^2
=
\sum_{i = 1}^r \sigma_i^2 \Norm{\tilde{\Bv}}
\le \sigma_\max^2 \Norm{\tilde{\Bv}}^2,
\end{equation}

or
\begin{equation}\label{eqn:convexOptimizationLecture8:660}
\begin{aligned}
\Norm{X \Bv}_2
&\le \sqrt{ \sigma_\max^2 } \Norm{\tilde{\Bv}} \\
&\le \sigma_\max.
\end{aligned}
\end{equation}

Set \( \Bv \) to the right singular value of \( X \) to get equality.

   \mychapter{Problem set II (not attempted).}
      %
% Copyright � 2017 Peeter Joot.  All Rights Reserved.
% Licenced as described in the file LICENSE under the root directory of this GIT repository.
%
\makeoproblem{Identifying convexity}{convexOptimization:problemSet2:1}{\citep{boyd2004convex} pr. 3.16 (a)-(c)}{
For each of the following functions determine whether it is convex, concave, quasiconvex, or quasiconcave
\makesubproblem{}{convexOptimization:problemSet2:1a}
\( f(x) = e^x - 1 \, \mbox{on $\bbR$} \).
\makesubproblem{}{convexOptimization:problemSet2:1b}
\( f(x_1, x_2) = x_1 x_2 \, \, \mbox{on $\bbR^2_{++}$} \).
\makesubproblem{}{convexOptimization:problemSet2:1c}
\( f(x_1, x_2) = 1/(x_1 x_2) \, \mbox{on $\bbR^2_{++}$} \).
} % makeproblem

\makeanswer{convexOptimization:problemSet2:1}{
\withproblemsetsParagraph{
\makeSubAnswer{}{convexOptimization:problemSet2:1a}

TODO.
\makeSubAnswer{}{convexOptimization:problemSet2:1b}

TODO.
\makeSubAnswer{}{convexOptimization:problemSet2:1c}

TODO.
} % redaction
} % answer

      %
% Copyright � 2017 Peeter Joot.  All Rights Reserved.
% Licenced as described in the file LICENSE under the root directory of this GIT repository.
%
\makeoproblem{Products and ratios of convex functions}{convexOptimization:problemSet2:2}{\citep{boyd2004convex} pr. 3.32 (a)}{
In general the product or ration of two convex functions is not convex.  However, there are some results that apply to functions on \( \bbR \).  Prove the following
} % makeproblem

\makeanswer{convexOptimization:problemSet2:2}{
If \( f \) and \( g \) are convex, both nondecreasing (or nonincreasing), and positive functions on an interval, then \( f g \) is convex.

TODO.
}

      %
% Copyright � 2017 Peeter Joot.  All Rights Reserved.
% Licenced as described in the file LICENSE under the root directory of this GIT repository.
%
\makeoproblem{Convex-concave functions and saddle-points}{convexOptimization:problemSet2:3}{\citep{boyd2004convex} pr. 3.14}{
We way the function \( f : \bbR^n \times \bbR^m \rightarrow \bbR \) is convex-concave if \( f(\Bx,\Bz) \) is a concave function of \( \Bz \), for each fixed \( \Bx\), and a convex function of \( \Bx \), for each fixed \(\Bz\).  We also require its domain to have the product form \( \dom f = A \times B \), where \( A \subseteq \bbR^n \) and \( B \subseteq \bbR^m \) are convex.
\makesubproblem{}{convexOptimization:problemSet2:3a}
Give a second-order condition for a twice differentable function \( f : \bbR^n \times \bbR^m \rightarrow \bbR \) to be convex-concave, in terms of its Hessian \( \spacegrad^2 f(\Bx,\Bz) \).
\makesubproblem{}{convexOptimization:problemSet2:3b}
Suppose that \( f : \bbR^n \times \bbR^m \rightarrow \bbR \) is a convex-concave and differentiable, with \( \spacegrad f(\tilde{\Bx}, \tilde{\Bz}) = 0\).  Show that the saddle-point property holds: for all \( \Bx, \Bz\), we have

\begin{equation}\label{eqn:ProblemSet2Problem3:20}
f(\tilde{\Bx}, \Bz)  \le f(\tilde{\Bx}, \tilde{\Bz})  \le f(\Bx, \tilde{\Bz}) .
\end{equation}

Show that this implies that \( f \) satisfies the strong max-in property:

\begin{equation}\label{eqn:ProblemSet2Problem3:40}
\sup_\Bz \inf_\Bx f(\Bx, \Bz) = \inf_\Bx \sup_\Bz f(\Bx,\Bz)
\end{equation}

(and their common value is \( f(\tilde{\Bx}, \tilde{\Bz}) \)).

\makesubproblem{}{convexOptimization:problemSet2:3c}
Now suppose that \( f : \bbR^n \times \bbR^m \rightarrow \bbR \) is differentiable, but not necessarily convex-concave, and the saddle-point property holds at \( \tilde{\Bx}, \tilde{\Bz}\):

\begin{equation}\label{eqn:ProblemSet2Problem3:60}
f(\tilde{\Bx}, \Bz)  \le f(\tilde{\Bx}, \tilde{\Bz})  \le f(\Bx, \tilde{\Bz}),
\end{equation}

for all \( \Bx, \Bz\).  Show that \( \spacegrad  f(\tilde{\Bx}, \tilde{\Bz}) = 0\).
} % makeproblem

\makeanswer{convexOptimization:problemSet2:3}{
\makeSubAnswer{}{convexOptimization:problemSet2:3a}

TODO.
\makeSubAnswer{}{convexOptimization:problemSet2:3b}

TODO.
\makeSubAnswer{}{convexOptimization:problemSet2:3c}

TODO.
}

      %
% Copyright � 2017 Peeter Joot.  All Rights Reserved.
% Licenced as described in the file LICENSE under the root directory of this GIT repository.
%
\makeproblem{Parameterized convexity}{convexOptimization:problemSet2:4}{
Consider the function

\begin{equation}\label{eqn:ProblemSet2Problem4:20}
f(x, y) = x^2 + y^2 + \beta x y + x + 2 y.
\end{equation}

Find \( (x^\conj, y^\conj) \) for which \( \spacegrad f = 0 \).  Express your answer as a function of \( \beta \).  For which values of \( \beta \) is the \( (x^\conj, y^\conj) \) a global minimum of \( f(x,y) \)?
} % makeproblem

\makeanswer{convexOptimization:problemSet2:4}{

TODO.
}

      %
% Copyright � 2017 Peeter Joot.  All Rights Reserved.
% Licenced as described in the file LICENSE under the root directory of this GIT repository.
%
\makeproblem{Maximum likelyhood estimation.}{convexOptimization:problemSet2:5}{
In this problem, we are given a set of data points \((x_i,y_i)\), \( i = 1 \cdots 100\). We wish to fit a quadratic
model,

\begin{equation}\label{eqn:ProblemSet2Problem5:20}
y_i = a x_i^2 + b x_i + c + n_i,
\end{equation}
to the data. Here, \((a, b, c)\) are the parameters to be determined and
\( n_i \) is the unknown observation noise. The \((x_i, y_i)\) points are contained in a file \matlabText{dataForMLest.mat}
available on the course webpage.  You may load the data to \matlabText{MATLAB} using the command
\matlabText{load ps01data}
and view them using \( \matlabText{scatter(x,y,'+')}\). Please use the same data set and find the
maximum likelihood estimate of \((a, b, c)\) assuming \(n_i\)'s are i.i.d. when

\makesubproblem{}{convexOptimization:problemSet2:5a}
\( n_i \sim N(0, 1)\);
\makesubproblem{}{convexOptimization:problemSet2:5b}
\(n_i\) is always positive and \( p_{n_i}(z) = e^{-z} u(z)\) where \(u(\cdot)\) is the unit step function.

Please plot the data and the models on the same \matlabText{MATLAB} figure and submit the figgure as a part of
your solution.
(\matlabText{MATLAB} has built-in functions to solve many optimization problems. For example, \matlabText{linprog}
solves a linear programming problem, quadprog solves a quadratic programming problem. You
may use \matlabText{help linprog} to get more details.

\paragraph{Hint:} \partref{convexOptimization:problemSet2:5a} has an analytic solution.)
} % makeproblem

%\makeanswer{convexOptimization:problemSet2:5}{
%\withproblemsetsParagraph{
%\makeSubAnswer{}{convexOptimization:problemSet2:5a}
%
%TODO.
%\makeSubAnswer{}{convexOptimization:problemSet2:5b}
%
%TODO.
%} % redaction
%} % answer

      %
% Copyright � 2017 Peeter Joot.  All Rights Reserved.
% Licenced as described in the file LICENSE under the root directory of this GIT repository.
%
\makeproblem{First and second order conditions for convexity}{convexOptimization:problemSet2:6}{
In class we proved the first and second-order conditions for convexity of differentiable scalar functions. In particular, the first-order condition we showed is that a differentiable function \( f : \bbR \rightarrow \bbR \) is convex if and only if \( \dom f \) is a convex set and

\begin{equation}\label{eqn:ProblemSet2Problem6:20}
f(x) \ge f(x_0) + f'(x_0) (x - x_0),
\end{equation}

for all \( x, x_0 \in \dom f\).
The second-order condition we showed is that a twice differentiable function \( f : \bbR \rightarrow \bbR \)
is convex if and only if \( \dom f \) is a convex set and \( f''(x) \ge 0 \) for all \( x \in \dom f \).

In this problem you are asked to prove the two corresponding vector results:

\makesubproblem{}{convexOptimization:problemSet2:6a}
If \( f : \bbR^n \rightarrow \bbR \)
is differentiable then \( f \) is a convex function if and only if \( \dom f \) is a convex set
and

\begin{equation}\label{eqn:ProblemSet2Problem6:40}
f(\Bx) \ge f(\Bx_0) + \lr{ \spacegrad f(\Bx_0)}^\T (\Bx - \Bx_0),
\end{equation}

for all \( \Bx, \Bx_0 \in \dom f \).

\makesubproblem{}{convexOptimization:problemSet2:6b}
If \( f : \bbR^n \rightarrow \bbR \)
is twice differentiable then \( f \) is a convex function if and only if \( \dom f \) is a
convex set and

\begin{equation}\label{eqn:ProblemSet2Problem6:60}
\spacegrad^2 f(\Bx) \ge 0,
\end{equation}
for all \( \Bx \in \dom f \).

To prove the above two results, follow the approach followed in class. Namely, show that the
differentiable (twice differentiable) vector function
\( f : \bbR^n \rightarrow \bbR \)
is convex if and only if the first
order (second-order) scalar condition holds along all lines in the domain.
In other words, show that
\( f \) is convex if and only if \( \dom f \) is a convex set and the first-order (second-order) scalar condition
holds for \( f(\Bx_0 + t \Bv) \) for all \( \Bx_0, \Bv \in \bbR^n, t \in \bbR \) and \( \Bx_0 + t \Bv \in \dom f \).

To be clear, we already proved \partref{convexOptimization:problemSet2:6a}
in class. For \partref{convexOptimization:problemSet2:6a}
you are simply asked to reproduce
that proof to ensure you fully understand the proof method.
Clearly explain the overall logic and
the logic of each step.
Then, in \partref{convexOptimization:problemSet2:6b} you are asked to take the same (lines-based) approach to
show the second-order condition.
} % makeproblem
%\makeanswer{convexOptimization:problemSet2:6}{
%\withproblemsetsParagraph{
%\makeSubAnswer{}{convexOptimization:problemSet2:6a}
%
%TODO.
%\makeSubAnswer{}{convexOptimization:problemSet2:6b}
%
%TODO.
%} % redaction
%} % answer

      %
% Copyright � 2017 Peeter Joot.  All Rights Reserved.
% Licenced as described in the file LICENSE under the root directory of this GIT repository.
%
\makeoproblem{Kullback-Leibler divergence and the information inequality}{convexOptimization:problemSet2:7}{\citep{boyd2004convex} pr. 3.13}{
Let \( D_{kl} \) be the Kullback-Liebler divergence, as defined in (3.17).  Prove the information inequality:
%
\begin{equation}\label{eqn:ProblemSet2Problem7:20}
D_{kl}(\Bu,\Bv) \ge 0,
\end{equation}
%
for all \( \Bu, \Bv \in \bbR^n_{++} \).  Also show that \( D_{kl}(\Bu,\Bv) = 0 \) if and only if \( \Bu = \Bv \).

\paragraph{Hint:} The Kullback-Liebler divergence can be expressed as
%
\begin{equation}\label{eqn:ProblemSet2Problem7:40}
D_{kl}(\Bu,\Bv) = f(\Bu) - f(\Bv) - \lr{ \spacegrad f(\Bv)}^\T (\Bu - \Bv).
\end{equation}
} % makeproblem

%\makeanswer{convexOptimization:problemSet2:7}{
%\withproblemsetsParagraph{
%
%TODO.
%} % redaction
%} % answer

      %
% Copyright � 2017 Peeter Joot.  All Rights Reserved.
% Licenced as described in the file LICENSE under the root directory of this GIT repository.
%
\makeproblem{Examples of proving convexity}{convexOptimization:problemSet2:8}{
\makesubproblem{}{convexOptimization:problemSet2:8a}
Show that the following function \( f : \bbR^n \rightarrow \bbR \) is convex where

\begin{equation}\label{eqn:ProblemSet2Problem8:20}
f(\Bx) =
\left\{
\begin{array}{l l}
-(x_1 x_2 \cdots x_n)^{1/n} & \quad \mbox{if \( x_1 > 0, \cdots, x_n > 0\)} \\
\infty & \quad \mbox{otherwise}
\end{array}
\right.
\end{equation}

Prove the above result by computing the Hessian of the function. (Note, this is a special
case of
\citep{boyd2004convex} pr. 3.18(b) in which you are asked to show that
\( (\det X)^{1/n} \)
is concave on
\(\dom f = S^n_{++} \).
While
for this problem I ask you to compute the Hessian, in that problem you
may take any approach you wish.)

\paragraph{Hint:} it may prove useful to use the relation that for any real numbers \( \alpha_1, \cdots, \alpha_n\), \( (\sum_{i = 1}^n \alpha_i)^2 \le n (\sum_{i = 1}^n \alpha_i^2) \).
\makesubproblem{}{convexOptimization:problemSet2:8b}
Show that the following function \( f : \bbR^n \rightarrow \bbR \) is convex where

\begin{equation}\label{eqn:ProblemSet2Problem8:40}
f(\Bx) = \exp\lr{ \beta \Bx^\T A \Bx },
\end{equation}

where \( \Bx \in \bbR^n \), \( A \) is a positive semidefinite symmetric \( n \times n \) matrix, and \( \beta \) is a positive scalar.
} % makeproblem

%\makeanswer{convexOptimization:problemSet2:8}{
%\withproblemsetsParagraph{
%\makeSubAnswer{}{convexOptimization:problemSet2:8a}
%
%TODO.
%\makeSubAnswer{}{convexOptimization:problemSet2:8b}
%
%TODO.
%} % redaction
%} % answer

      %
% Copyright � 2017 Peeter Joot.  All Rights Reserved.
% Licenced as described in the file LICENSE under the root directory of this GIT repository.
%
\makeoproblem{\((\det X)^{1/n}\) is concave on \( \dom f = S^n_{++}\)}{convexOptimization:problemSet2:9}{\citep{boyd2004convex} pr. 3.18 (b)}{
Adapt the proof of concavity of the log-determinant function in \S 3.1.5 to show the following.

\begin{equation}\label{eqn:ProblemSet2Problem9:20}
f(X) = \tr (X^{-1}),
\end{equation}

is convex on \( \dom f = S^n_{++} \).
} % makeproblem

\makeanswer{convexOptimization:problemSet2:9}{
\withproblemsetsParagraph{

TODO.
} % redaction
} % answer

      %
% Copyright � 2017 Peeter Joot.  All Rights Reserved.
% Licenced as described in the file LICENSE under the root directory of this GIT repository.
%
\makeoproblem{Some functions on the probability simplex.}{convexOptimization:problemSet2:10}{\citep{boyd2004convex} pr. 3.24 (a)-(e)}{
Let \( x \) be a real-valued random variable which takes values \( \setlr{ a_1, \cdots, a_n } \) where \( a_1 < a_2 < \cdots < a_n \), with \( \prob( x = a_i) = p_i\, i \in [1,n] \).  For each of the following functions of \( p \) (on the probability simplex \( \setlr{ p \in \bbR^n_{+} | \BOne^\T p = 1 }\)), determine if the function is convex, concave, quasiconvex, or quasiconcave.
\makesubproblem{}{convexOptimization:problemSet2:10a}
\( \BE x\).
\makesubproblem{}{convexOptimization:problemSet2:10b}
\( \prob( x \ge \alpha ) \).
\makesubproblem{}{convexOptimization:problemSet2:10c}
\( \prob( \alpha \le x \le \beta ) \).
\makesubproblem{}{convexOptimization:problemSet2:10d}
\( \sum_{i = 1}^n p_i \log p_i\), the negative entropy of the distribution.
\makesubproblem{}{convexOptimization:problemSet2:10e}
\( \var x = \BE(x - \BE x)^2\).
} % makeproblem

\makeanswer{convexOptimization:problemSet2:10}{
\withproblemsetsParagraph{
\makeSubAnswer{}{convexOptimization:problemSet2:10a}

TODO.
\makeSubAnswer{}{convexOptimization:problemSet2:10b}

TODO.
\makeSubAnswer{}{convexOptimization:problemSet2:10c}

TODO.
\makeSubAnswer{}{convexOptimization:problemSet2:10d}

TODO.
\makeSubAnswer{}{convexOptimization:problemSet2:10e}

TODO.
} % redaction
} % answer

